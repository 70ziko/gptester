\RequirePackage{ifluatex}
\documentclass[12pt,aspectratio=169,ignorenonframetext]{beamer}
\usepackage[polish]{babel}
\keywords{Politechnika, Wrocławska, szablon prezentacji, LaTeX, beamer, księga logotypu, system identyfikacji wizualnej}
\subject{System identyfikacji wizualnej Politechniki Wrocławskiej}
\ifluatex
 \usepackage{fontspec}
 \setsansfont{Carlito}[Numbers=OldStyle,
  %	UprightFont=*-Regular,
  %	BoldFont=*-Bold,
  %	ItalicFont=Cyklop-Italic,
  %	BoldItalicFont=*-BoldOblique]
 ]
\else
 \usepackage[utf8]{inputenc}
 \usepackage[T1]{fontenc}
 \usepackage[]{carlito}
\fi
\usepackage{polski}

\usepackage{hologo}
\usepackage{listings}
\lstloadlanguages{TeX}
\lstdefinestyle{beamer}{
 language=[LaTeX]TeX,
 morekeywords={documentclass, usetheme, subtitle, institute, titlegraphics, keywords, subject, sfdefault, trans, insertpagenumber, insertframenumber, insertframestartpage, insertframeendpage, insertdocumentendpage, pgfpagesuselayout, setsansfont}
}
\lstset{style=beamer}
\mode<article>{%
 \usepackage{lmodern}
 \usepackage{graphicx}
}

\mode<presentation>
{
 %\usepackage{carlito}
 %\usepackage{lmodern}
}

\usepackage{pifont}

\newcommand{\cmark}{\ding{51}}%
\newcommand{\xmark}{\ding{55}}%
\newcommand{\done}{\rlap{$\square$}{\raisebox{2pt}{\large\hspace{1pt}\cmark}}%
 \hspace{-2.5pt}}
\newcommand{\wontfix}{\rlap{$\square$}{\large\hspace{1pt}\xmark}}
\newcommand{\emptybox}{$\square$}


\usepackage{hyperref}

\title{Szablon prezentacji beamer zgodny\\ z~księgą logotypu Politechniki Wrocławskiej:\\ krótka dokumentacja}
\subtitle{Wersja: \VCRevisionMod}
\author{Wojciech Myszka}
\institute{Katedra Mechaniki i~Inżynierii Materiałowej}
\date{\BZRBuildDate\\ kompilowane: \today}

\RequirePackage{ifluatex}
\documentclass[12pt,aspectratio=169,ignorenonframetext]{beamer}
\usepackage[polish]{babel}
\keywords{Politechnika, Wrocławska, szablon prezentacji, LaTeX, beamer, księga logotypu, system identyfikacji wizualnej}
\subject{System identyfikacji wizualnej Politechniki Wrocławskiej}
\ifluatex
 \usepackage{fontspec}
 \setsansfont{Carlito}[Numbers=OldStyle,
  %	UprightFont=*-Regular,
  %	BoldFont=*-Bold,
  %	ItalicFont=Cyklop-Italic,
  %	BoldItalicFont=*-BoldOblique]
 ]
\else
 \usepackage[utf8]{inputenc}
 \usepackage[T1]{fontenc}
 \usepackage[]{carlito}
\fi
\usepackage{polski}

\usepackage{hologo}
\usepackage{listings}
\lstloadlanguages{TeX}
\lstdefinestyle{beamer}{
 language=[LaTeX]TeX,
 morekeywords={documentclass, usetheme, subtitle, institute, titlegraphics, keywords, subject, sfdefault, trans, insertpagenumber, insertframenumber, insertframestartpage, insertframeendpage, insertdocumentendpage, pgfpagesuselayout, setsansfont}
}
\lstset{style=beamer}
\mode<article>{%
 \usepackage{lmodern}
 \usepackage{graphicx}
}

\mode<presentation>
{
 %\usepackage{carlito}
 %\usepackage{lmodern}
}

\usepackage{pifont}

\newcommand{\cmark}{\ding{51}}%
\newcommand{\xmark}{\ding{55}}%
\newcommand{\done}{\rlap{$\square$}{\raisebox{2pt}{\large\hspace{1pt}\cmark}}%
 \hspace{-2.5pt}}
\newcommand{\wontfix}{\rlap{$\square$}{\large\hspace{1pt}\xmark}}
\newcommand{\emptybox}{$\square$}


\usepackage{hyperref}

\title{Szablon prezentacji beamer zgodny\\ z~księgą logotypu Politechniki Wrocławskiej:\\ krótka dokumentacja}
\subtitle{Wersja: \VCRevisionMod}
\author{Wojciech Myszka}
\institute{Katedra Mechaniki i~Inżynierii Materiałowej}
\date{\BZRBuildDate\\ kompilowane: \today}

\RequirePackage{ifluatex}
\documentclass[12pt,aspectratio=169,ignorenonframetext]{beamer}
\usepackage[polish]{babel}
\keywords{Politechnika, Wrocławska, szablon prezentacji, LaTeX, beamer, księga logotypu, system identyfikacji wizualnej}
\subject{System identyfikacji wizualnej Politechniki Wrocławskiej}
\ifluatex
 \usepackage{fontspec}
 \setsansfont{Carlito}[Numbers=OldStyle,
  %	UprightFont=*-Regular,
  %	BoldFont=*-Bold,
  %	ItalicFont=Cyklop-Italic,
  %	BoldItalicFont=*-BoldOblique]
 ]
\else
 \usepackage[utf8]{inputenc}
 \usepackage[T1]{fontenc}
 \usepackage[]{carlito}
\fi
\usepackage{polski}

\usepackage{hologo}
\usepackage{listings}
\lstloadlanguages{TeX}
\lstdefinestyle{beamer}{
 language=[LaTeX]TeX,
 morekeywords={documentclass, usetheme, subtitle, institute, titlegraphics, keywords, subject, sfdefault, trans, insertpagenumber, insertframenumber, insertframestartpage, insertframeendpage, insertdocumentendpage, pgfpagesuselayout, setsansfont}
}
\lstset{style=beamer}
\mode<article>{%
 \usepackage{lmodern}
 \usepackage{graphicx}
}

\mode<presentation>
{
 %\usepackage{carlito}
 %\usepackage{lmodern}
}

\usepackage{pifont}

\newcommand{\cmark}{\ding{51}}%
\newcommand{\xmark}{\ding{55}}%
\newcommand{\done}{\rlap{$\square$}{\raisebox{2pt}{\large\hspace{1pt}\cmark}}%
 \hspace{-2.5pt}}
\newcommand{\wontfix}{\rlap{$\square$}{\large\hspace{1pt}\xmark}}
\newcommand{\emptybox}{$\square$}


\usepackage{hyperref}

\title{Szablon prezentacji beamer zgodny\\ z~księgą logotypu Politechniki Wrocławskiej:\\ krótka dokumentacja}
\subtitle{Wersja: \VCRevisionMod}
\author{Wojciech Myszka}
\institute{Katedra Mechaniki i~Inżynierii Materiałowej}
\date{\BZRBuildDate\\ kompilowane: \today}

\RequirePackage{ifluatex}
\documentclass[12pt,aspectratio=169,ignorenonframetext]{beamer}
\usepackage[polish]{babel}
\keywords{Politechnika, Wrocławska, szablon prezentacji, LaTeX, beamer, księga logotypu, system identyfikacji wizualnej}
\subject{System identyfikacji wizualnej Politechniki Wrocławskiej}
\ifluatex
 \usepackage{fontspec}
 \setsansfont{Carlito}[Numbers=OldStyle,
  %	UprightFont=*-Regular,
  %	BoldFont=*-Bold,
  %	ItalicFont=Cyklop-Italic,
  %	BoldItalicFont=*-BoldOblique]
 ]
\else
 \usepackage[utf8]{inputenc}
 \usepackage[T1]{fontenc}
 \usepackage[]{carlito}
\fi
\usepackage{polski}

\usepackage{hologo}
\usepackage{listings}
\lstloadlanguages{TeX}
\lstdefinestyle{beamer}{
 language=[LaTeX]TeX,
 morekeywords={documentclass, usetheme, subtitle, institute, titlegraphics, keywords, subject, sfdefault, trans, insertpagenumber, insertframenumber, insertframestartpage, insertframeendpage, insertdocumentendpage, pgfpagesuselayout, setsansfont}
}
\lstset{style=beamer}
\mode<article>{%
 \usepackage{lmodern}
 \usepackage{graphicx}
}

\mode<presentation>
{
 %\usepackage{carlito}
 %\usepackage{lmodern}
}

\usepackage{pifont}

\newcommand{\cmark}{\ding{51}}%
\newcommand{\xmark}{\ding{55}}%
\newcommand{\done}{\rlap{$\square$}{\raisebox{2pt}{\large\hspace{1pt}\cmark}}%
 \hspace{-2.5pt}}
\newcommand{\wontfix}{\rlap{$\square$}{\large\hspace{1pt}\xmark}}
\newcommand{\emptybox}{$\square$}


\usepackage{hyperref}

\title{Szablon prezentacji beamer zgodny\\ z~księgą logotypu Politechniki Wrocławskiej:\\ krótka dokumentacja}
\subtitle{Wersja: \VCRevisionMod}
\author{Wojciech Myszka}
\institute{Katedra Mechaniki i~Inżynierii Materiałowej}
\date{\BZRBuildDate\\ kompilowane: \today}

\input{nowy_szablon_doc.ltx}

\begin{document}

\begin{frame}[plain]
 \maketitle
\end{frame}

\ifpdf
 \tableofcontents
\fi

\section{Terminologia}

Jest pewien problem z~terminologią, a~zwłaszcza z~bardzo sobie bliskimi pojęciami. W~dokumentacji beamera używane są pojęcia:
\begin{itemize}
 \item
       frame,
 \item
       slide,
 \item
       overlay,
 \item
       page,
 \item
       hangout.
\end{itemize}
Trzeba to jakoś przetłumaczyć na Polski.

\begin{frame}[fragile]
 \frametitle<presentation>{Terminologia}
 \begin{itemize}
  \item
        strona (\emph{page}) — plik PDF składa się z~wielu stron i~widzimy tylko jedną z~nich na ekranie;
  \item
        slajd (\emph{frame}) — podstawowa jednostka logiczna informacji wyświetlanej na ekranie; składać się na nią \textbf{może} kilka \textbf{nakładek} (\textbf{warstw}),
  \item
        nakładka/warstwa (\emph{overlay}) — z~punktu widzenia pliku PDF będzie to jedna \textbf{strona} zawierająca, najczęściej, przyrostowo dodawane informacje składające się na \textbf{slajd}.

        \mode<article>{Instrukcje przyrostowego dodawania informacji do slajdu opisuje dokumentacja pakietu beamer \cite[Rozdział~9]{Beamer}.}
 \end{itemize}
 Występuje tez pojęcie „folia” (\emph{transparency}).

 Pojęcie \emph{hangout} tłumaczę jako „materiały informacyjne”.
\end{frame}
Materiały informacyjne to, po prostu, wszystkie slajdy wydrukowane po 2, 4,… na kartce papieru.

Rozróżnienie to będzie zwłaszcza istotne w~przypadku produkowaniu materiałów informacyjnych czy drukowania folii.

\section{Wprowadzenie}

Od chwili opracowania pierwszej wersji logotypu Politechniki Wrocławskiej \cite{siw2004} upłynęło sporo czasu — \difftoday{2004}{10}{21}. Prawdę mówiąc nie przez wszystkich został on przyjęty jednakowo entuzjastycznie. Bardzo wiele osób krytykowało pewne jego „przesztywnienie”. I~o~ile bardzo restrykcyjne zasady używania znaku Politechniki  Wrocławskiej uznać należy za zasadne, to już prezentacja nie dawała żadnych możliwości wykazania się inwencją. Z~drugiej strony — ciągle uważam, że znacznie ważniejsza od formy jest treść.

Pierwszym „objawem” buntu było pojawienie się „rektorskiej” (jak się dowiedziałem) wersji prezentacji. Pierwotnie używana ona była, między innymi, do prezentowania rocznych Sprawozdań Rektora (rys.~\ref{rektorska}).

\begin{frame}
 \frametitle<presentation>{Szablon „rektorski”}
 \begin{figure}
  \mode<presentation>{
   \includegraphics[width=\textwidth,height=.7\textheight,keepaspectratio]{rektorska}
  }
  \mode<article>{
   \centering\includegraphics[width=.5\textwidth]{rektorska}
  }
  \caption{Wygląd prezentacji z~„rektorską” wersją szablonu}\label{rektorska}
 \end{figure}
\end{frame}

Szablon ten nigdy nie był oficjalnie ogłoszony, a~dziś jako „poprzednia wersję szablonu”\footnote{\url{https://pwr.edu.pl/uczelnia/informacje-ogolne/materialy-promocyjne/logotyp}} wskazywana jest wersja oryginalna. Natomiast dało się zauważyć, że wiele osób z~tej wersji szablonu korzystało (a nawet ciągle korzysta — co i~rusz pojawiają się takie prezentacje).

Z drugiej strony daje się również zauważyć, że całkiem oficjalne prezentacje (nawet osób z~Kierownictwa) nie zawsze korzystają z~obowiązującego szablonu. Ale Kierownictwu wolno więcej.

W roku 2016 została ogłoszona nowa wersja Systemu Identyfikacji Wizualnej \cite{siw2016}. Prezentowana jest ona na stronach Uczelni \cite{Logotyp}. Podstawowa zmiana wiąże się ze zmianą angielskiej nazwy Politechniki Wrocławskiej z~\emph{Wrocław Uniwersity of Technology} na \emph{Wrocław University of Science and Technology}.

Kolejne zmiany związane były z~uzyskaniem przez Politechnikę Wrocławską prawa do używania logo „HR Excellence in Research”\footnote{\url{https://pwr.edu.pl/uczelnia/europejska-karta-naukowca/}.}. Logo trafiło zarówno na papier firmowy jak i~na stronę tytułową prezentacji.

\begin{frame}{Logo HR Excellence in Research}
 \includegraphics<presentation>[width=\textwidth,height=.7\textheight,keepaspectratio]{Hr_p1.pdf}
 \includegraphics<article>[width=.2\textwidth]{Hr_p1}
\end{frame}

Zmieniły się też szablony prezentacji. Wybrano niezbyt łatwy do odwzorowania w~\LaTeX{}u szablon: tło nie jest jednolicie białe, tylko w~kratkę, która zmienia swoje natężenie w~różnych miejscach. Ale zasadniczo szablon jest podobny do ,,rektorskiego'', tylko występuje w~dwu wariantach.

Jeden nazywać będę „poziomym” (poziomy pasek z~logo uczelni), a~drugi pionowym.

\begin{frame}
 \frametitle<presentation>{Tła nowego szablonu w~wersji polskiej i~angielskiej}
 \mode<article>{
  \begin{figure}
   \includegraphics[width=.24\textwidth]{image1_p1_pl}
   \includegraphics[width=.24\textwidth]{image1_p2_pl}
   \includegraphics[width=.24\textwidth]{image1_p1_en}
   \includegraphics[width=.24\textwidth]{image1_p2_en}

   \includegraphics[width=.24\textwidth]{image2_p1_pl}
   \includegraphics[width=.24\textwidth]{image2_p2_pl}
   \includegraphics[width=.24\textwidth]{image2_p1_en}
   \includegraphics[width=.24\textwidth]{image2_p2_en}
   \caption{Tła strony tytułowej (pierwszy rząd) i~kolejnych stron nowego szablonu}
  \end{figure}
 }
 \mode<presentation>{
  \framesubtitle<1>{tytułowa, horizontal, polski}
  \framesubtitle<2>{tytułowa, vertical, polski}
  \framesubtitle<3>{tytułowa, horizontal, angielski}
  \framesubtitle<4>{tytułowa, vertical, angielski}
  \framesubtitle<5>{normalna, horizontal, polski}
  \framesubtitle<6>{normalna, vertical, polski}
  \framesubtitle<7>{normalna, horizontal, angielski}
  \framesubtitle<8>{normalna,vertical, angielski}
  \transduration<1>{1}
  \transdissolve<1>[duration=1]
  \transduration<2>{1}
  \transdissolve<2>[duration=1]
  \transduration<3>{1}
  \transdissolve<3>[duration=1]
  \transduration<4>{1}
  \transdissolve<4>[duration=1]
  \transduration<5>{1}
  \transdissolve<5>[duration=1]
  \transduration<6>{1}
  \transdissolve<6>[duration=1]
  \transduration<7>{1}
  \transdissolve<7>[duration=1]
  \only<1>{\includegraphics[width=\textwidth,height=.7\textheight,keepaspectratio]{image1_p1_pl}}
  \only<2>{\includegraphics[width=\textwidth,height=.7\textheight,keepaspectratio]{image1_p2_pl}}
  \only<3>{\includegraphics[width=\textwidth,height=.7\textheight,keepaspectratio]{image1_p1_en}}
  \only<4>{\includegraphics[width=\textwidth,height=.7\textheight,keepaspectratio]{image1_p2_en}}
  \only<5>{\includegraphics[width=\textwidth,height=.7\textheight,keepaspectratio]{image2_p1_pl}}
  \only<6>{\includegraphics[width=\textwidth,height=.7\textheight,keepaspectratio]{image2_p2_pl}}
  \only<7>{\includegraphics[width=\textwidth,height=.7\textheight,keepaspectratio]{image2_p1_en}}
  \only<8>{\includegraphics[width=\textwidth,height=.7\textheight,keepaspectratio]{image2_p2_en}}
 }
\end{frame}

\section{Jak instalować}\label{instalacja}

Najlepiej by było, gdyby pakiet stał się częścią archiwum \href{https://ctan.org/}{CTAN}. Nie bardzo mam ochotę próbować dostosowywać pakiet do (wysokich) standardów tam panujących. Pozostaniemy zatem (przynajmniej na razie) przy instalacji ręcznej.

W każdym razie plik archiwum zgodny jest ze strukturą TDS~\cite{tds} i~wystarczy pliki rozpakować.

\begin{frame}[fragile]
 \frametitle<presentation>{Jak instalować}
 \only<presentation>{Należy pliki ręcznie skopiować we \textbf{właściwe} miejsce.}

 Właściwe miejsce zdefiniowane jest przez zmienne:
 \begin{itemize}
  \item
        TEXMFLOCAL (dla plików dostępnych dla wszystkich użytkowników)
        \begin{itemize}
         \item \path{/usr/local/texlive/texmf-local} (standardowo Unix/Linux)
         \item \path{%SystemDrive%\texlive\texmf-local} (Windows)
        \end{itemize}
        oraz
  \item
        TEXMFHOME (dla plików dostępnych dla aktualnego użytkownika); kartoteka texmf w~kartotece \path{$HOME} (Unix, Linux) lub \path (Windows).
  \item
        Jeszcze inaczej jest w~przypadku MiKTeXa. Opisuje to dokumentacja\only<article>{~\cite{miktex}}. Kluczową aplikacją będzie Miktex Options (w~skrócie mo).
 \end{itemize}
\end{frame}

Można też wszystkie pliki z~podkartoteki \lstinline|tex/latex/pwr| skopiować do kartoteki, w~której tworzony jest dokument. Tej metody jednak nie polecam.

\begin{frame}[fragile]
 \frametitle{Instalacja}
 \begin{enumerate}
  \item Linux: Aby ściągnąć plik archiwum trzeba sięgnąć pod \href{https://kmim.wm.pwr.edu.pl/myszka/logotyp/nowy_szablon.tar.xz}{ten adres}. Najprościej będzie wykonać następujące polecenia:
        \ifpdf
         \begin{lstlisting}[language=bash]
cd /tmp
wget https://kmim.wm.pwr.edu.pl/myszka/logotyp/nowy_szablon.tar.xz
mkdir -p ~/texmf
cd ~/texmf
tar xJf /tmp/nowy_szablon.tar.xz
\end{lstlisting}
        \else
         \begin{verbatim}
cd /tmp
wget https://kmim.wm.pwr.edu.pl/myszka/logotyp/nowy_szablon.tar.xz
mkdir -p ~/texmf
cd ~/texmf
tar xJf /tmp/nowy_szablon.tar.xz
\end{verbatim}
        \fi
  \item Windows: Aby ściągnąć plik archiwum trzeba sięgnąć pod \href{https://kmim.wm.pwr.edu.pl/myszka/logotyp/nowy_szablon.zip}{ten adres}. Później postępujemy zgodnie z~wymaganiami używanego do dekompresji i~rozpakowywania programu. Kiedyś\only<article>{\footnote{Ale ja od dawna nie używam Windows i~nawet nie mam komputera, żeby to sprawdzić!}} po kliknięciu prawym klawiszem myszy na archiwum zip pojawiało się coś podobnego do „Wyodrębnij tutaj”.
 \end{enumerate}
\end{frame}

\subsection{Wersja oszczędna}

Oprócz tego istnieje „wersja oszczędna” szablonu zawierająca wszystko za wyjątkiem plików źródłowych i~dokumentacji szablonu. Plik można pobrać \href{https://kmim.wm.pwr.edu.pl/myszka/logotyp/nowy_szablon_maly.tar.xz}{tu (wyłącznie wersja tar.xz)}.

\subsection{Używane pakiety}

Potrzebne będą „standardowe” pakiety potrzebne do pracy z~beamerem. Są to:
\begin{enumerate}
 \item
       ifxetex
 \item
       oberdiek
 \item
       pgf
 \item
       amsfonts
 \item
       amsmath
 \item
       beamer
 \item
       etoolbox
 \item
       geometry
 \item
       graphics
 \item
       hyperref
 \item
       enumerate
 \item
       xcolor
 \item
       extsizes (gdy żądamy innych niż standardowe rozmiary fontów).
\end{enumerate}
\begin{frame}
 \frametitle{Dodatkowe pakiety}

 Oprócz pakietów standardowych \only<presentation>{(niezbędnych do pracy z~beamerem)} nie są potrzebne żadne pakiety dodatkowe.

 W~przypadku gdy zechcemy użyć innych niż standardowe fontów potrzebne mogą być pakiety je instalujące, na przykład:
 \begin{itemize}
  \item
        carlito — gdy chcemy naśladować standardowy font PowerPointa: Calibri\only<article>{ (patrz również rozdział~\ref{lab:fonty})}.
 \end{itemize}
\end{frame}

\section{Jak używać}

\hologo{pdfLaTeX}:

\begin{frame}[fragile]
 \frametitle<presentation>{Jak używać?}
 \framesubtitle{\hologo{pdfLaTeX}}
 Użycie szablonu jest bardzo proste:
 \ifpdf
  \begin{lstlisting}  
\documentclass[ ]{beamer}
\usepackage[utf8]{inputenc}
\usepackage[T1]{fontenc}
\usepackage{carlito}
\usetheme[horizontal=true]{NewPwr}
\end{lstlisting}
 \else
  \begin{verbatim}  
\documentclass[ ]{beamer}
\usepackage[utf8]{inputenc}
\usepackage[T1]{fontenc}
\usepackage{carlito}
\usetheme[horizontal=true]{NewPwr}
\end{verbatim}
 \fi
\end{frame}

\hologo{LuaLaTeX}:

\begin{frame}[fragile]
 \frametitle<presentation>{Jak używać?}
 \framesubtitle{\hologo{LuaLaTeX}}
 Użycie szablonu jest równie proste:
 \ifpdf
  \begin{lstlisting}  
\documentclass[ ]{beamer}
\usepackage{fontspec}
\setsansfont{Carlito}[Numbers=OldStyle]
\usetheme[horizontal=true]{NewPwr}
\end{lstlisting}
 \else
  \begin{verbatim}  
\documentclass[ ]{beamer}
\usepackage{fontspec}
\setsansfont{Carlito}[Numbers=OldStyle]
\usetheme[horizontal=true]{NewPwr}
\end{verbatim}
 \fi
\end{frame}

W przypadku \hologo{LuaLaTeX}a trzeba jednak dokładnie przestudować dokumentację tego programu, żeby poprawnie korzystać z~fontów (i wszystkiego co one oferują).

Nie powinno być wększych problemów podczas stosowana \hologo{XeLaTeX}a.  Ale (w tej wersji dokumentacji) nie będę o~tym pisał.

%\end{document}

\begin{frame}[fragile,allowframebreaks]
 \frametitle<presentation>{Jak używać}
 \framesubtitle{Opcje szablonu PwrNew}
 W~poleceniu \lstinline|\usetheme[ ]{NewPwr}|, w~nawiasach kwadratowych podajemy opcje szablonu. Można wybrać język i~tło.
 \begin{enumerate}
  \item
        Język wybieramy podając opcję \lstinline|lang=en| lub \lstinline|lang=pl|, gdy jej nie podamy — wybrany będzie język polski.
  \item
        Rodzaj tła (pasek poziomy lub pasek pionowy) podajemy deklarując \lstinline|horizontal=true| albo \lstinline|vertical=true|; można skrócić do samego \hypertarget{pasekpoziomy}{\lstinline|horizontal|}, \hypertarget{pasekpionowy}{\lstinline|vertical|}.
        Gdy nie podamy — będzie \lstinline|vertical|.
        %        \mode<article>{
        %        \begin{figure}
        %        \caption{aaaa}
        %        \end{figure}
        %        }
  \item
        Ponieważ gdzieniegdzie funkcjonuje jeszcze wersja ,,rektorska'' szablonu, można użyć (niezalecanej) opcji \lstinline|rektor=true| lub, po prostu \lstinline|rektor|\only<article>{\footnote{Uwaga! Nie znalazłem nigdzie angielskiej wersji „Rektorskiego” szablonu/prezentacji.}}. Szablon występuje wyłącznie w~wersji polskojęzycznej i~dla ekranów o~\lstinline|aspectratio=43|.
  \item
        Uważam, że nie zawsze musi być prezentowane logo HR Excellence in Research. Władze uczelni są innego zdania%
        \only<article>{\footnote{W piśmie z~dnia 31 lipca 2017, Rektor Jasieńko pisze tak: „Proszę również o~umieszczanie logo HR […] we wszystkich prezentacjach […]”.}}%
        \ i~w~związku z~tym domyślnie logo \textbf{jest umieszczane}.   Można je wyłączyć używając opcji \lstinline|hr=false|.
        \only<article>{\end{enumerate}}
 \mode<article>{
  „Standardowe” szablony, na każdej stronie — z~wyjątkiem tytułowej — zawierają numer strony. Niektórzy tego nie lubią. Inni się tego domagają. Osobiście nie mam zdania. Zatem powinno to trafić do opcji.
  
  Natomiast w beamerze jest pewien problem z numeracją: Ze względu na to, że każdy slajd zbudowany może być z kilku warstw (nakładek) numer strony PDF zazwyczaj różni się od numeru kolejnego slajdu.
 }
 \only<article>{\begin{enumerate}[resume]}
  \item Opcja \lstinline|pagenumbers=frame| włącza numerowanie slajdów; na każdym slajdzie będzie umieszczony jego numer. Opcja \lstinline|pagenumbers=page|  lub \lstinline|pagenumbers=true| lub  lub \lstinline|pagenumbers| umieszcza na slajdzie numer strony. (Standardowo numerowanie jest wyłączone!) Strona tytułowa prezentacji nie jest numerowana!
  \item Od „zawsze”, standardowo, na dole każdego slajdu beamer umieszczał symbole nawigacji. Ponieważ, mało kto z~tego korzysta — są one wyłączone. Można je włączyć umieszczając wśród opcji szablonu \lstinline|navigation| lub \lstinline|navigation=true|.
 \end{enumerate}
\end{frame}

Tak na marginesie, beamer oferuje możliwość włączenia następujących informacji (\cite[Rozdział~8.2.1]{Beamer}): 
\begin{itemize}
 \item
       numer strony (polecenie \lstinline|\insertpagenumber|),
 \item
       numer „ramki” (\emph{frame}) \lstinline|\insertframenumber|,
 \item
       numer pierwszej strony aktualnej ramki \lstinline|\insertframestartpage|,
 \item
       numer ostatniej strony aktualnej ramki \lstinline|\insertframeendpage|,
 \item
       numer ostatniej strony dokumentu \lstinline|\insertdocumentendpage|.
\end{itemize}

\begin{frame}
 \frametitle<presentation>{Jak używać}
 \framesubtitle{Opcje klasy beamer}
 Warto również korzystać z~dodatkowych parametrów klasy beamer. Najważniejsze z~nich to:
 \begin{enumerate}
  \item
        \hypertarget{fontsize}{Rozmiar czcionki} definiowany jako:  \alert{8pt}, \alert{9pt}, 10pt, 11pt, 12pt, \alert{14pt}, \alert{17pt}, \alert{20pt}. Rozmiar 17pt to standardowy rozmiar prezentacji PowerPoint i~Impress.
        \mode<article>{Rozmiary oznaczone  kursywą wymagają zainstalowania pakietu extsize.}
  \item
        \hypertarget{proporcje}{Proporcje} ekranu ustala opcja \lstinline|aspectratio|. Standardowo \lstinline|aspectratio=43| (co oznacza proporcje 4:3). Inne dostępne wartości to 1610, 169, 149, 141, 54, 32.

        \mode<article>{
         Znaczenie parametrów jest następujące:
         \begin{description}
          \item [1610] ekran o~proporcjach $16\times10$,
          \item [169] ekran o~proporcjach $16\times9$,
          \item [149] proporcje $14\times9$,
          \item [141] ekran o~proporcjach $\sqrt{2}\times1$ czyli $1{,}41\times1$
          \item [54] ekran o~proporcjach $5\times4$,
          \item [43] (\textbf{standardowo!}) ekran o~proporcjach $4\times3$,
          \item [32] proporcje $3\times2$.
         \end{description}
        }
  \item
        Normalnie zawartość slajdu jest centrowana pionowo. Można to zmienić globalnie używając opcji \lstinline|c| (standard, centrowane) lub \lstinline|t| — umieszczana od góry slajdu.
        \mode<article>{
         Sposób rozmieszczania zawartości można ustalać odrębnie dla każdego slajdu, korzystając z~opcji środowiska \lstinline|frame|: \lstinline|t|, \lstinline|c| lub \lstinline|b|.
        }
  \item
        Inne, warte uwagi parametry, to: \hyperlink{sec:handout}{handout} i~\hyperlink{sec:trans}{trans}.
 \end{enumerate}
\end{frame}

\subsection{Personalizacja}

Pakiet beamer pozwala na bardzo „intensywną” personalizację slajdów i~wyglądu prezentacji. Warto zapoznać się z~dokumentacją \cite{Beamer} (nawet jeżeli liczy ona 250 stron).

\begin{frame}
 \frametitle<presentation>{Personalizacja}
 \begin{enumerate}
  \item
        „Pasek” \hyperlink{pasekpionowy}{pionowy}/\hyperlink{pasekpoziomy}{poziomy},
  \item
        \hyperlink{fonty}{Fonty},
  \item
        Podstawowa \hyperlink{fontsize}{wielkość liter},
  \item
        \hyperlink{efekty}{Efekty przejścia},
  \item
        \hyperlink{proporcje}{Proporcje obrazu},
  \item
        \hyperlink{tytul}{Slajd tytułowy}.
 \end{enumerate}
\end{frame}

\subsection{Fonty}\label{lab:fonty}

Szablon sugeruje użycie fontu Calibri\footnote{Nie jest to nigdzie napisane, ale tak jest ustawiony font domyślny prezentacji (co, zapewne, jest cechą wbudowaną programu PowePoint).}. Najbliższym jego odpowiednikiem w~\LaTeX{}u jest font Carlito.
Zatem możemy zainstalować pakiet \href{https://www.ctan.org/pkg/carlito}{carlito} i~użyć go w~prezentacji.
Alternatywą może być użycie fontu Iwona lub Kurier\footnote{\url{http://jmn.pl/kurier-i-iwona/}}.

Warto też wspomnieć, że w przypadku gdy font Calibri jest zainstalowany w systemie\footnote{Niestety pakiet ttf-mscorefonts-installer nie zawiera tego fontu.} (co zapewne jest standardem systemu Windows) można z niego korzystać kompilując ptrezentację z użyciem LuaLaTeXa.

\begin{frame}[fragile,allowframebreaks]
 \frametitle<presentation>{Fonty}
 \hypertarget{fonty}{\null}
 \begin{enumerate}
  \item Standard.
        Gdy nie zdefiniujemy nic, użyty zostanie standardowy font bezszeryfowy. Jest on bardzo jasny.\\<all>
        \includegraphics[width=.8\textwidth]{default}
  \item Latin Modern.
        \begin{lstlisting}
  \usepackage{lmodern}
  \end{lstlisting}
        W~zasadzie to samo co standardowy, ale wygląda znacznie lepiej.\\<all>
        \includegraphics[width=.8\textwidth]{lmodern}
  \item Carlito \only<presentation>{(darmowy odpowiednik fontu Calibri)}
        \begin{lstlisting}
 \usepackage{carlito}
\end{lstlisting}
        \includegraphics[width=.8\textwidth]{carlito}
        \begin{lstlisting}
 \usepackage[lining]{carlito}
\end{lstlisting}
        \includegraphics[width=.8\textwidth]{carlito_lining}

        \mode<article>{Font Carlito standardowo oferuje cyfry \textbf{nautyczne} (zwane czasami „zepsutymi”).
         Jeżeli zdecydowanie nam nie pasują — należy w~poleceniu dodać dodatkowy parametr (w nawiasach kwadratowych) \lstinline|lining|.}
  \item Font Iwona
        \begin{lstlisting}
 \usepackage{iwona}
\end{lstlisting}
        \includegraphics[width=.8\textwidth]{iwona}
        \only<presentation>{\pagebreak}
  \item Font Kurier
        \begin{lstlisting}
 \usepackage{kurier}
\end{lstlisting}
        \includegraphics[width=.8\textwidth]{kurier}
  \item Trebuchet (był to podstawowy font poprzedniego szablonu).
        Jeżeli ktoś ma go zainstalowanego (i korzystał z~poprzedniego szablonu bez problemów), można użyć go poleceniem:
        \begin{lstlisting}
\renewcommand{\sfdefault}{jtrr}
\end{lstlisting}
        \includegraphics[width=.8\textwidth]{trebuchet}
 \end{enumerate}
 Ale nie są to jedyne możliwości.
\end{frame}

Ja, osobiście bardzo lubię fonty \href{https://www.ctan.org/pkg/iwona}{Iwona} i~\href{https://www.ctan.org/pkg/kurier}{Kurier}, więc bardzo często zamiast fontu Carlito używam właśnie ich.

Doświadczenie uczy, że w~prezentacjach lepiej sprawdzają się fonty bezszeryfowe (\emph{sans serif}) można użyć praktycznie każdego fontu dostępnego w~systemie \hologo{LaTeX}. Na stronie \url{http://www.tug.dk/FontCatalogue/sansseriffonts.html} znajduje się katalog dostępnych fontów. Jeżeli zrezygnować z~\hologo{pdfLaTeX}a na rzecz \hologo{LuaLaTeX}a lub \hologo{XeLaTeX}a — można użyć praktycznie każdego współczesnego fontu dostępnego w~systemie. Warto jednak poczytać dokumentację pakietu \href{https://ctan.org/pkg/fontspec}{fontspec} \cite{Fontspec}.

\subsection{Efekty „przejścia”}

Efekt „przejścia” (\emph{transition effect}) slajdów są (według mnie) czymś złym.
Zajmują czas (nawet jak tylko dwie sekundy), podczas procesu znikania jednego slajdu i~pojawiania się drugiego zawartość jest niedostępna lub nieczytelna, odwracają wreszcie uwagę od zasadniczego tematu.
No, ale są.
Również w~beamerze można z~nich korzystać. Definiuje się je na poziomie warstw lub slajdów.

\begin{frame}
 \frametitle<presentation>{Efekty ,,przejścia''}
 \framesubtitle{Transition effects}
 \transduration<1->{1}
 \transblindshorizontal<2>[duration=1]
 \transblindsvertical<3>[duration=1]
 \transboxin<4>[duration=1]
 \transboxout<5>[duration=1]
 \transcover<6>[duration=1]
 \transdissolve<7>[duration=1]
 \transfade<8>[duration=1]
 \transglitter<9>[duration=1]
 \hypertarget{efekty}{Beamer} dostarcza następujący zestaw efektów:
 \begin{itemize}
  \item<2>
        \textbf{blindshorizontal}
        Show the slide as if horizontal blinds were pulled away.
  \item<3>
        \textbf{blindsvertical}
        Show the slide as if vertical blinds were pulled away.
  \item<4>
        \textbf{boxin}
        Show the slide by moving to the center from all four sides.
  \item<5>
        \textbf{boxout}
        Show the slide by showing more and more of a~rectangular area that is centered on the slide center.
  \item<6>
        \textbf{cover}
        Show the slide by covering the content that was shown before.
  \item<7>
        \textbf{dissolve}
        Show the slide by slowly dissolving what was shown before.
  \item<8>
        \textbf{fade}
        Show the slide by slowly fading what was shown before.
  \item<9>
        \textbf{glitter}
        Show the slide with a~glitter effect that sweeps in the specified direction.
 \end{itemize}
\end{frame}
\begin{frame}
 \frametitle<presentation>{Efekty ,,przejścia'' cd}
 \transduration<1->{1}
 \transpush<1>[duration=1]
 \transsplitverticalin<2>[duration=1]
 \transsplitverticalout<3>[duration=1]
 \transsplithorizontalin<4>[duration=1]
 \transsplithorizontalout<5>[duration=1]
 \transwipe<6>[duration=1]
 %\transreplace<3>[duration=1]
 \begin{itemize}
  \item<1>
        \textbf{push}
        Show the slide by pushing what was shown before off the screen using the new content.
  \item<2>
        \textbf{replace}
        Replace the previous slide directly (default behaviour).
  \item<3>
        \textbf{splitverticalin}
        Show the slide by sweeping two vertical lines from the sides inward.
  \item<4>
        \textbf{splitverticalout}
        Show the slide by sweeping two vertical lines from the center outward.
  \item<5>
        \textbf{splithorizontalin}
        Show the slide by sweeping two horizontal lines from the sides inward.
  \item<6>
        \textbf{splithorizontalout}
        Show the slide by sweeping two horizontal lines from the center outward.
  \item<7>
        \textbf{wipe}
        Show the slide by sweeping a~single line in the specified direction, thereby “wiping out” the previous contents.
 \end{itemize}
\end{frame}

% \begin{frame}
% \frametitle<beamer>{Efekty ,,przejścia''}
% \only<beamer>{Beamer dostarcza następujący zestaw efektów:}
% \begin{itemize}
%\item
% \textbf{splitverticalin} 
%Show the slide by sweeping two vertical lines from the sides inward.
%\item
% \textbf{splitverticalout} 
%Show the slide by sweeping two vertical lines from the center outward.
%\item
% \textbf{splithorizontalin} 
%Show the slide by sweeping two horizontal lines from the sides inward.
%\item
% \textbf{splithorizontalout} 
%Show the slide by sweeping two horizontal lines from the center outward.
%\item
% \textbf{wipe} 
% Show the slide by sweeping a~single line in the specified direction, thereby “wiping out” the previous contents.
%\end{itemize}
%\end{frame}

\begin{frame}[fragile]
 \frametitle{Efekty przejścia}
 \framesubtitle{Uwagi}
 \begin{itemize}
  \item
        Pamiętać natomiast trzeba, że nie każdy program używany do wyświetlania plików PDF wszystkie efekty realizuje.
        Można mieć pewność chyba tylko w~przypadku oryginalnego Adobe Acrobat Readera.
  \item
        Efekt przejścia został zaimplementowany w~taki sposób, że można go wskazać dla wybranych nakładek. Polecenia włączające efekt przejścia dokładnie opisane są w~dokumentacji pakietu beamer\only<article>{~\cite{Beamer}}.
        Tworzy się je przez dodanie do nazwy efektu przedrostka: \lstinline|\trans|. Całe polecenie ma postać:\\
        \lstinline|\trans|\emph{efekt}\lstinline|<|\emph{specyfikacja slajdów}\lstinline|>[|\emph{dodatkowe parametry}\lstinline|]|.
  \item
        \emph{dodatkowe parametry} to, na przykład czas trwanie efektu: \lstinline|duration=1|
 \end{itemize}
\end{frame}

\subsection{Uwagi}

Można \href{https://pwr.edu.pl/uczelnia/informacje-ogolne/materialy-promocyjne/logotyp/}{znaleźć} przygotowany (przez \href{https://pwr.edu.pl/uczelnia/informacje-ogolne/organizacja-uczelni/index,dzial-informacji-i-promocji.html}{Dział Informacji i~Promocji}) materiał zawierający \href{https://kmim.wm.pwr.edu.pl/myszka/wp-content/uploads/sites/2/2017/12/prezentacje_instrukcje.pdf}{przykładowe slajdy} i~możliwości rozmieszczenia tekstu oraz grafik  w~nowych szablonach.
Niektóre efekty bardzo łatwo jest osiągnąć w~beamerze, inne są trudniejsze do uzyskania. W~szczególności modyfikacje slajdu tytułowego wymagają modyfikacji szablonu.

\subsection{Modyfikacje slajdu tytułowego}

\begin{frame}[fragile]
 \frametitle<presentation>{Slajd tytułowy}
 \hypertarget{tytul}{Modyfikacje} slajdu tytułowego są bardzo ograniczone. Standardowo mogą się tam pojawić następujące pola:
 \begin{columns}
  \begin{column}{.5\textwidth}
   \begin{itemize}
    \item
          autor (\lstinline|\author{}|),
    \item
          tytuł (\lstinline|\title{}|),
    \item
          podtytuł (\lstinline|\subtitle{}|),
    \item
          data (\lstinline|\date{}|),
    \item
          instytucja (\lstinline|\institute{}|),
    \item
          grafika (\lstinline|\titlegraphics{}|),
   \end{itemize}
   \only<article>{Slajd tytułowy przedstawia rys.~\ref{title}.}
  \end{column}
  \begin{column}{.48\textwidth}
   \only<article>{\begin{figure}}
     \includegraphics[width=.6\textwidth]{metadane}

     \only<article>{
     \caption{Wygląd slajdu tytułowego}\label{title}
    \end{figure}
   }
  \end{column}
 \end{columns}
\end{frame}



\subsection{Metadane}

\begin{frame}[fragile]
 \frametitle<presentation>{Metadane}
 Zawartość pól:
 \begin{itemize}
  \item
        \lstinline|\title{}| oraz \lstinline|\subtitle{}|,
  \item
        \lstinline|\author{}|,
  \item
        \lstinline|\keywords{}|,
  \item
        \lstinline|\subject{}|,
 \end{itemize}
 automatycznie trafia do odpowiednich pól pliku PDF zawierających metadane. Generalnie metadanie mogą ułatwić zadanie wyszukiwarkom czy poprawnie zdefiniowć autora tekstów umieszczonych w~Internecie.

 Z~umieszczania metadanych w~pliku PDF można zrezygnować:
 \begin{lstlisting}
\documentclass[usepdftitle=false]{beamer}
\end{lstlisting}
\end{frame}

Tak na marginesie: pakiet \hyperref{https://ctan.org/pkg/pdfprivacy}{pdfprivacy} pozwala ma usunięcie wielu innych metadanych zapisywanych standardowo do pliku PDF.

\section{Dokumentacja}

Całą dokumentacja znajduje się w~podkatalogu (patrz rozdział~\ref{instalacja}) \path{doc/pwr}. Składa się na nią:
\begin{frame}
 \frametitle<presentation>{Dokumentacja}
 \only<presentation>{Na dokumentację składa się:}
 \begin{itemize}
  \item
        plik \href{https://kmim.wm.pwr.edu.pl/myszka/wp-content/uploads/sites/2/2017/12/nowy_szablon_doc_article.pdf}{{nowy\_szablon\_doc\_article.pdf}}\only<article>{ (zawierający treść tej strony)};
  \item
        plik \href{https://kmim.wm.pwr.edu.pl/myszka/wp-content/uploads/sites/2/2017/12/nowy_szablon_doc_beamer.pdf}{nowy\_szablon\_doc\_beamer.pdf} będący wersją „prezentacyjną” tej dokumentacji\only<presentation>{ (czyli właśnie to)};
  \item
        pliki źródłowe dokumentacji;
  \item
        przykładowa, prosta prezentacja \href{https://kmim.wm.pwr.edu.pl/myszka/wp-content/uploads/sites/2/2017/12/NewPwr-example.pdf}{NewPwr-example.pdf} wraz ze źródłami;
        \only<presentation>{\item dodatkowym bonusem jest plik \lstinline|oficyna_url.bst| czyli szablon  „\textbf{prawie} zgodny” z~wymaganiami Oficyny Wydawniczej PWr.}
 \end{itemize}
 \only<presentation>{Dokumentacja znajduje się w~podkatalogu \path{doc/pwr}.}
\end{frame}


\section{Materiały informacyjne (handout)}

Aby przygotować materiały informacyjne (tzw. handout) wystarczy wybrać ich format (dwie albo cztery slajdy na kartkę) i~zmodyfikować nieco źródło.
\begin{frame}[fragile]
 \frametitle<presentation>{Materiały informacyjne}
 \framesubtitle<presentation>{(handout)}
 \hypertarget{sec:handout}{}
 \begin{lstlisting}
\documentclass[handout]{beamer}
\usepackage{pgfpages}
\pgfpagesuselayout{2 on 1}[a4paper,
                           border shrink=5mm]
\end{lstlisting}
 może też być
 \begin{lstlisting}
\pgfpagesuselayout{4 on 1}[a4paper,landscape,
                           border shrink=5mm]
\end{lstlisting}
 Standardowo materiały informacyjne zostaną wygenerowane używając „pełnej wersji” slajdów (po uwzględnieniu wszystkich informacji „przyrostowych”).
\end{frame}
Można eksperymentować z~innymi opcjami oszczędnościowymi i~umieszczać więcej slajdów na jednej stronie.

Generalna zasada podczas tworzenia materiałów informacyjnych jest taka, że zostaje umieszczona w~nich „ostateczne” wersja slajdu (czyli po uwzględnieniu wszystkich przyrostów). Można jednak zdecydować inaczej — używając parametru handout w~definicji przyrostu.

\section{Folie}

Pakiet beamer może również być użyty do tworzenia folii (transparencies). Nie bardzo wiem w~jakiej sytuacji może się to dziś przydać — ale istnieje taka możliwość.

\begin{frame}[fragile]
 \frametitle<presentation>{Folie}
 \framesubtitle{trans}
 \hypertarget{sec:trans}{}
 \begin{lstlisting}
\documentclass[trans,17pt,aspectratio=43]{beamer}
\end{lstlisting}
 \begin{itemize}
  \item
        Standardowo folie zostaną wygenerowane używając „pełnej wersji” slajdów (po uwzględnieniu wszystkich informacji „przyrostowych”).
  \item
        Mądrze korzystając ze wszystkich możliwości oferowanych przez „kompilację warunkową”  można w~ten sposób wygenerować „lżejszą” (mniej atramentożerną) wersję prezentacji.
 \end{itemize}

\end{frame}

Działanie opcji \lstinline|trans| jest zbliżone do działania opcji \lstinline|hangout|: użyte będą „ostateczne” wersje slajdów (czyli po uwzględnieniu wszystkich przyrostów).

\input{beamer_uwagi}

\section{Do zrobienia}

\begin{frame}[fragile,allowframebreaks]
 \frametitle<presentation>{Do zrobienia}
 \begin{enumerate}
  \item
        Szczegółowe:
        \begin{itemize}
         \item[\done]
               Numery slajdów
         \item[\done]
               Numer strony na stronie tytułowej
         \item[\emptybox]
               Poprawne pozycjonowanie ilustracji wstawianej poleceniem \lstinline|\titlegraphics| dla wersji pionowej
         \item[\emptybox]
               Personalizacja strony tytułowej
         \item[\emptybox]
               Wygląd bloków (zwłaszcza alertblock) i~twierdzeń, dowodów,… % powiązać z~kolorkiem szarym z~szablonu (dla poziomego?)
         \item[\done]
               Slajdy plain (bez tła)
         \item[\emptybox]
               \emph{Progress bar?}
               \item [\emptybox] Zoom
               \item [\emptybox] Kompresja slajdów
        \end{itemize}
  \item Ogólne:
        \begin{itemize}
         \item[\emptybox]
               Notatki
         \item[\emptybox]
               Drugi ekran
         \item[\emptybox]
               Narzędzia (Linux)
               \item[\emptybox] 
               Animacje — ogólne uwagi i przykłady
        \end{itemize}
        \mode<presentation>{\pagebreak}
  \item Zgłoszone na kursie uwagi/błędy:
        \begin{itemize}
         \item[\emptybox] Kolorystyka spisu treści (Agenda)
         \item[\done] Numeracja sekcji w spisie treści
          \item[\done] Raczej numer slajdu niż numer strony PDF
        \end{itemize}
  \item Zauważone błędy (niedoróbki):
        \begin{itemize}
         \item[\done]
               Numeracja slajdów (slajd 11???)
        \end{itemize}
 \end{enumerate}
\end{frame}

%\subsection{Komentarze}
%
%W maju byłem uczestnikiem organizowanego na Poltechnice Wrocławskiej w ramach projektu Innowacyjna Uczelnia --- Innowacyjny Nauczyciel szkolenia \LaTeX{} tworzenie profesjonalnej prezentacji. Zgłoszono mi tam pewne propozycje ulepszenia szablonu.

\section{Uwagi}

\subsection{Struktura archiwum/plików w~pakiecie}

{
 \footnotesize
 \lstinputlisting{tree.txt}
}

\subsection{Pliki dodatkowe}

\subsubsection{Szablon bibliografii zgodny z~wymaganiami Oficyny Wydawniczej PWr}

W pakiecie znajduje się również styl bibliografii \path{oficyna\_url.bst}. Powstał om podczas składania różnych publikacji wydawanych przez Oficynę Wydawniczą PWr i~\textbf{znacznym stopniu} spełnia wymagania edytorskie. (Choć te, w~zależności od Redaktora potrafią się trochę różnić, czy może inaczej: różni się odporność Redaktorów na drobne niezgodności.)

Plik można pobrać również \href{http://kmim.wm.pwr.edu.pl/myszka/wp-content/uploads/sites/2/2018/11/oficyna_url.bst}{osobno}.

\subsubsection{Testy}

W pod-kartotece z~dokumentacją znajduje się pod-kartoteka beamer-benchmark. W~niej znajdują się pliki zaczerpnięte z~repozytorium \href{https://github.com/louisstuart96/beamer-benchmark}{beamer-benchmark} \href{https://github.com/louisstuart96}{Luisa Stuarta}. Pozwalają one „testować” różne elementy szablonu.

\subsection{Stara wersja szablonu…}

…jest ciągle \href{https://kmim.wm.pwr.edu.pl/myszka/projekty/szablon-prezentacji-pwr/szablon-prezentacji-zgodny-z-ksiega-logotypu/}{dostępna}.

%\subsection{}

%\end{document}





\begin{document}

\begin{frame}[plain]
 \maketitle
\end{frame}

\ifpdf
 \tableofcontents
\fi

\section{Terminologia}

Jest pewien problem z~terminologią, a~zwłaszcza z~bardzo sobie bliskimi pojęciami. W~dokumentacji beamera używane są pojęcia:
\begin{itemize}
 \item
       frame,
 \item
       slide,
 \item
       overlay,
 \item
       page,
 \item
       hangout.
\end{itemize}
Trzeba to jakoś przetłumaczyć na Polski.

\begin{frame}[fragile]
 \frametitle<presentation>{Terminologia}
 \begin{itemize}
  \item
        strona (\emph{page}) — plik PDF składa się z~wielu stron i~widzimy tylko jedną z~nich na ekranie;
  \item
        slajd (\emph{frame}) — podstawowa jednostka logiczna informacji wyświetlanej na ekranie; składać się na nią \textbf{może} kilka \textbf{nakładek} (\textbf{warstw}),
  \item
        nakładka/warstwa (\emph{overlay}) — z~punktu widzenia pliku PDF będzie to jedna \textbf{strona} zawierająca, najczęściej, przyrostowo dodawane informacje składające się na \textbf{slajd}.

        \mode<article>{Instrukcje przyrostowego dodawania informacji do slajdu opisuje dokumentacja pakietu beamer \cite[Rozdział~9]{Beamer}.}
 \end{itemize}
 Występuje tez pojęcie „folia” (\emph{transparency}).

 Pojęcie \emph{hangout} tłumaczę jako „materiały informacyjne”.
\end{frame}
Materiały informacyjne to, po prostu, wszystkie slajdy wydrukowane po 2, 4,… na kartce papieru.

Rozróżnienie to będzie zwłaszcza istotne w~przypadku produkowaniu materiałów informacyjnych czy drukowania folii.

\section{Wprowadzenie}

Od chwili opracowania pierwszej wersji logotypu Politechniki Wrocławskiej \cite{siw2004} upłynęło sporo czasu — \difftoday{2004}{10}{21}. Prawdę mówiąc nie przez wszystkich został on przyjęty jednakowo entuzjastycznie. Bardzo wiele osób krytykowało pewne jego „przesztywnienie”. I~o~ile bardzo restrykcyjne zasady używania znaku Politechniki  Wrocławskiej uznać należy za zasadne, to już prezentacja nie dawała żadnych możliwości wykazania się inwencją. Z~drugiej strony — ciągle uważam, że znacznie ważniejsza od formy jest treść.

Pierwszym „objawem” buntu było pojawienie się „rektorskiej” (jak się dowiedziałem) wersji prezentacji. Pierwotnie używana ona była, między innymi, do prezentowania rocznych Sprawozdań Rektora (rys.~\ref{rektorska}).

\begin{frame}
 \frametitle<presentation>{Szablon „rektorski”}
 \begin{figure}
  \mode<presentation>{
   \includegraphics[width=\textwidth,height=.7\textheight,keepaspectratio]{rektorska}
  }
  \mode<article>{
   \centering\includegraphics[width=.5\textwidth]{rektorska}
  }
  \caption{Wygląd prezentacji z~„rektorską” wersją szablonu}\label{rektorska}
 \end{figure}
\end{frame}

Szablon ten nigdy nie był oficjalnie ogłoszony, a~dziś jako „poprzednia wersję szablonu”\footnote{\url{https://pwr.edu.pl/uczelnia/informacje-ogolne/materialy-promocyjne/logotyp}} wskazywana jest wersja oryginalna. Natomiast dało się zauważyć, że wiele osób z~tej wersji szablonu korzystało (a nawet ciągle korzysta — co i~rusz pojawiają się takie prezentacje).

Z drugiej strony daje się również zauważyć, że całkiem oficjalne prezentacje (nawet osób z~Kierownictwa) nie zawsze korzystają z~obowiązującego szablonu. Ale Kierownictwu wolno więcej.

W roku 2016 została ogłoszona nowa wersja Systemu Identyfikacji Wizualnej \cite{siw2016}. Prezentowana jest ona na stronach Uczelni \cite{Logotyp}. Podstawowa zmiana wiąże się ze zmianą angielskiej nazwy Politechniki Wrocławskiej z~\emph{Wrocław Uniwersity of Technology} na \emph{Wrocław University of Science and Technology}.

Kolejne zmiany związane były z~uzyskaniem przez Politechnikę Wrocławską prawa do używania logo „HR Excellence in Research”\footnote{\url{https://pwr.edu.pl/uczelnia/europejska-karta-naukowca/}.}. Logo trafiło zarówno na papier firmowy jak i~na stronę tytułową prezentacji.

\begin{frame}{Logo HR Excellence in Research}
 \includegraphics<presentation>[width=\textwidth,height=.7\textheight,keepaspectratio]{Hr_p1.pdf}
 \includegraphics<article>[width=.2\textwidth]{Hr_p1}
\end{frame}

Zmieniły się też szablony prezentacji. Wybrano niezbyt łatwy do odwzorowania w~\LaTeX{}u szablon: tło nie jest jednolicie białe, tylko w~kratkę, która zmienia swoje natężenie w~różnych miejscach. Ale zasadniczo szablon jest podobny do ,,rektorskiego'', tylko występuje w~dwu wariantach.

Jeden nazywać będę „poziomym” (poziomy pasek z~logo uczelni), a~drugi pionowym.

\begin{frame}
 \frametitle<presentation>{Tła nowego szablonu w~wersji polskiej i~angielskiej}
 \mode<article>{
  \begin{figure}
   \includegraphics[width=.24\textwidth]{image1_p1_pl}
   \includegraphics[width=.24\textwidth]{image1_p2_pl}
   \includegraphics[width=.24\textwidth]{image1_p1_en}
   \includegraphics[width=.24\textwidth]{image1_p2_en}

   \includegraphics[width=.24\textwidth]{image2_p1_pl}
   \includegraphics[width=.24\textwidth]{image2_p2_pl}
   \includegraphics[width=.24\textwidth]{image2_p1_en}
   \includegraphics[width=.24\textwidth]{image2_p2_en}
   \caption{Tła strony tytułowej (pierwszy rząd) i~kolejnych stron nowego szablonu}
  \end{figure}
 }
 \mode<presentation>{
  \framesubtitle<1>{tytułowa, horizontal, polski}
  \framesubtitle<2>{tytułowa, vertical, polski}
  \framesubtitle<3>{tytułowa, horizontal, angielski}
  \framesubtitle<4>{tytułowa, vertical, angielski}
  \framesubtitle<5>{normalna, horizontal, polski}
  \framesubtitle<6>{normalna, vertical, polski}
  \framesubtitle<7>{normalna, horizontal, angielski}
  \framesubtitle<8>{normalna,vertical, angielski}
  \transduration<1>{1}
  \transdissolve<1>[duration=1]
  \transduration<2>{1}
  \transdissolve<2>[duration=1]
  \transduration<3>{1}
  \transdissolve<3>[duration=1]
  \transduration<4>{1}
  \transdissolve<4>[duration=1]
  \transduration<5>{1}
  \transdissolve<5>[duration=1]
  \transduration<6>{1}
  \transdissolve<6>[duration=1]
  \transduration<7>{1}
  \transdissolve<7>[duration=1]
  \only<1>{\includegraphics[width=\textwidth,height=.7\textheight,keepaspectratio]{image1_p1_pl}}
  \only<2>{\includegraphics[width=\textwidth,height=.7\textheight,keepaspectratio]{image1_p2_pl}}
  \only<3>{\includegraphics[width=\textwidth,height=.7\textheight,keepaspectratio]{image1_p1_en}}
  \only<4>{\includegraphics[width=\textwidth,height=.7\textheight,keepaspectratio]{image1_p2_en}}
  \only<5>{\includegraphics[width=\textwidth,height=.7\textheight,keepaspectratio]{image2_p1_pl}}
  \only<6>{\includegraphics[width=\textwidth,height=.7\textheight,keepaspectratio]{image2_p2_pl}}
  \only<7>{\includegraphics[width=\textwidth,height=.7\textheight,keepaspectratio]{image2_p1_en}}
  \only<8>{\includegraphics[width=\textwidth,height=.7\textheight,keepaspectratio]{image2_p2_en}}
 }
\end{frame}

\section{Jak instalować}\label{instalacja}

Najlepiej by było, gdyby pakiet stał się częścią archiwum \href{https://ctan.org/}{CTAN}. Nie bardzo mam ochotę próbować dostosowywać pakiet do (wysokich) standardów tam panujących. Pozostaniemy zatem (przynajmniej na razie) przy instalacji ręcznej.

W każdym razie plik archiwum zgodny jest ze strukturą TDS~\cite{tds} i~wystarczy pliki rozpakować.

\begin{frame}[fragile]
 \frametitle<presentation>{Jak instalować}
 \only<presentation>{Należy pliki ręcznie skopiować we \textbf{właściwe} miejsce.}

 Właściwe miejsce zdefiniowane jest przez zmienne:
 \begin{itemize}
  \item
        TEXMFLOCAL (dla plików dostępnych dla wszystkich użytkowników)
        \begin{itemize}
         \item \path{/usr/local/texlive/texmf-local} (standardowo Unix/Linux)
         \item \path{%SystemDrive%\texlive\texmf-local} (Windows)
        \end{itemize}
        oraz
  \item
        TEXMFHOME (dla plików dostępnych dla aktualnego użytkownika); kartoteka texmf w~kartotece \path{$HOME} (Unix, Linux) lub \path (Windows).
  \item
        Jeszcze inaczej jest w~przypadku MiKTeXa. Opisuje to dokumentacja\only<article>{~\cite{miktex}}. Kluczową aplikacją będzie Miktex Options (w~skrócie mo).
 \end{itemize}
\end{frame}

Można też wszystkie pliki z~podkartoteki \lstinline|tex/latex/pwr| skopiować do kartoteki, w~której tworzony jest dokument. Tej metody jednak nie polecam.

\begin{frame}[fragile]
 \frametitle{Instalacja}
 \begin{enumerate}
  \item Linux: Aby ściągnąć plik archiwum trzeba sięgnąć pod \href{https://kmim.wm.pwr.edu.pl/myszka/logotyp/nowy_szablon.tar.xz}{ten adres}. Najprościej będzie wykonać następujące polecenia:
        \ifpdf
         \begin{lstlisting}[language=bash]
cd /tmp
wget https://kmim.wm.pwr.edu.pl/myszka/logotyp/nowy_szablon.tar.xz
mkdir -p ~/texmf
cd ~/texmf
tar xJf /tmp/nowy_szablon.tar.xz
\end{lstlisting}
        \else
         \begin{verbatim}
cd /tmp
wget https://kmim.wm.pwr.edu.pl/myszka/logotyp/nowy_szablon.tar.xz
mkdir -p ~/texmf
cd ~/texmf
tar xJf /tmp/nowy_szablon.tar.xz
\end{verbatim}
        \fi
  \item Windows: Aby ściągnąć plik archiwum trzeba sięgnąć pod \href{https://kmim.wm.pwr.edu.pl/myszka/logotyp/nowy_szablon.zip}{ten adres}. Później postępujemy zgodnie z~wymaganiami używanego do dekompresji i~rozpakowywania programu. Kiedyś\only<article>{\footnote{Ale ja od dawna nie używam Windows i~nawet nie mam komputera, żeby to sprawdzić!}} po kliknięciu prawym klawiszem myszy na archiwum zip pojawiało się coś podobnego do „Wyodrębnij tutaj”.
 \end{enumerate}
\end{frame}

\subsection{Wersja oszczędna}

Oprócz tego istnieje „wersja oszczędna” szablonu zawierająca wszystko za wyjątkiem plików źródłowych i~dokumentacji szablonu. Plik można pobrać \href{https://kmim.wm.pwr.edu.pl/myszka/logotyp/nowy_szablon_maly.tar.xz}{tu (wyłącznie wersja tar.xz)}.

\subsection{Używane pakiety}

Potrzebne będą „standardowe” pakiety potrzebne do pracy z~beamerem. Są to:
\begin{enumerate}
 \item
       ifxetex
 \item
       oberdiek
 \item
       pgf
 \item
       amsfonts
 \item
       amsmath
 \item
       beamer
 \item
       etoolbox
 \item
       geometry
 \item
       graphics
 \item
       hyperref
 \item
       enumerate
 \item
       xcolor
 \item
       extsizes (gdy żądamy innych niż standardowe rozmiary fontów).
\end{enumerate}
\begin{frame}
 \frametitle{Dodatkowe pakiety}

 Oprócz pakietów standardowych \only<presentation>{(niezbędnych do pracy z~beamerem)} nie są potrzebne żadne pakiety dodatkowe.

 W~przypadku gdy zechcemy użyć innych niż standardowe fontów potrzebne mogą być pakiety je instalujące, na przykład:
 \begin{itemize}
  \item
        carlito — gdy chcemy naśladować standardowy font PowerPointa: Calibri\only<article>{ (patrz również rozdział~\ref{lab:fonty})}.
 \end{itemize}
\end{frame}

\section{Jak używać}

\hologo{pdfLaTeX}:

\begin{frame}[fragile]
 \frametitle<presentation>{Jak używać?}
 \framesubtitle{\hologo{pdfLaTeX}}
 Użycie szablonu jest bardzo proste:
 \ifpdf
  \begin{lstlisting}  
\documentclass[ ]{beamer}
\usepackage[utf8]{inputenc}
\usepackage[T1]{fontenc}
\usepackage{carlito}
\usetheme[horizontal=true]{NewPwr}
\end{lstlisting}
 \else
  \begin{verbatim}  
\documentclass[ ]{beamer}
\usepackage[utf8]{inputenc}
\usepackage[T1]{fontenc}
\usepackage{carlito}
\usetheme[horizontal=true]{NewPwr}
\end{verbatim}
 \fi
\end{frame}

\hologo{LuaLaTeX}:

\begin{frame}[fragile]
 \frametitle<presentation>{Jak używać?}
 \framesubtitle{\hologo{LuaLaTeX}}
 Użycie szablonu jest równie proste:
 \ifpdf
  \begin{lstlisting}  
\documentclass[ ]{beamer}
\usepackage{fontspec}
\setsansfont{Carlito}[Numbers=OldStyle]
\usetheme[horizontal=true]{NewPwr}
\end{lstlisting}
 \else
  \begin{verbatim}  
\documentclass[ ]{beamer}
\usepackage{fontspec}
\setsansfont{Carlito}[Numbers=OldStyle]
\usetheme[horizontal=true]{NewPwr}
\end{verbatim}
 \fi
\end{frame}

W przypadku \hologo{LuaLaTeX}a trzeba jednak dokładnie przestudować dokumentację tego programu, żeby poprawnie korzystać z~fontów (i wszystkiego co one oferują).

Nie powinno być wększych problemów podczas stosowana \hologo{XeLaTeX}a.  Ale (w tej wersji dokumentacji) nie będę o~tym pisał.

%\end{document}

\begin{frame}[fragile,allowframebreaks]
 \frametitle<presentation>{Jak używać}
 \framesubtitle{Opcje szablonu PwrNew}
 W~poleceniu \lstinline|\usetheme[ ]{NewPwr}|, w~nawiasach kwadratowych podajemy opcje szablonu. Można wybrać język i~tło.
 \begin{enumerate}
  \item
        Język wybieramy podając opcję \lstinline|lang=en| lub \lstinline|lang=pl|, gdy jej nie podamy — wybrany będzie język polski.
  \item
        Rodzaj tła (pasek poziomy lub pasek pionowy) podajemy deklarując \lstinline|horizontal=true| albo \lstinline|vertical=true|; można skrócić do samego \hypertarget{pasekpoziomy}{\lstinline|horizontal|}, \hypertarget{pasekpionowy}{\lstinline|vertical|}.
        Gdy nie podamy — będzie \lstinline|vertical|.
        %        \mode<article>{
        %        \begin{figure}
        %        \caption{aaaa}
        %        \end{figure}
        %        }
  \item
        Ponieważ gdzieniegdzie funkcjonuje jeszcze wersja ,,rektorska'' szablonu, można użyć (niezalecanej) opcji \lstinline|rektor=true| lub, po prostu \lstinline|rektor|\only<article>{\footnote{Uwaga! Nie znalazłem nigdzie angielskiej wersji „Rektorskiego” szablonu/prezentacji.}}. Szablon występuje wyłącznie w~wersji polskojęzycznej i~dla ekranów o~\lstinline|aspectratio=43|.
  \item
        Uważam, że nie zawsze musi być prezentowane logo HR Excellence in Research. Władze uczelni są innego zdania%
        \only<article>{\footnote{W piśmie z~dnia 31 lipca 2017, Rektor Jasieńko pisze tak: „Proszę również o~umieszczanie logo HR […] we wszystkich prezentacjach […]”.}}%
        \ i~w~związku z~tym domyślnie logo \textbf{jest umieszczane}.   Można je wyłączyć używając opcji \lstinline|hr=false|.
        \only<article>{\end{enumerate}}
 \mode<article>{
  „Standardowe” szablony, na każdej stronie — z~wyjątkiem tytułowej — zawierają numer strony. Niektórzy tego nie lubią. Inni się tego domagają. Osobiście nie mam zdania. Zatem powinno to trafić do opcji.
  
  Natomiast w beamerze jest pewien problem z numeracją: Ze względu na to, że każdy slajd zbudowany może być z kilku warstw (nakładek) numer strony PDF zazwyczaj różni się od numeru kolejnego slajdu.
 }
 \only<article>{\begin{enumerate}[resume]}
  \item Opcja \lstinline|pagenumbers=frame| włącza numerowanie slajdów; na każdym slajdzie będzie umieszczony jego numer. Opcja \lstinline|pagenumbers=page|  lub \lstinline|pagenumbers=true| lub  lub \lstinline|pagenumbers| umieszcza na slajdzie numer strony. (Standardowo numerowanie jest wyłączone!) Strona tytułowa prezentacji nie jest numerowana!
  \item Od „zawsze”, standardowo, na dole każdego slajdu beamer umieszczał symbole nawigacji. Ponieważ, mało kto z~tego korzysta — są one wyłączone. Można je włączyć umieszczając wśród opcji szablonu \lstinline|navigation| lub \lstinline|navigation=true|.
 \end{enumerate}
\end{frame}

Tak na marginesie, beamer oferuje możliwość włączenia następujących informacji (\cite[Rozdział~8.2.1]{Beamer}): 
\begin{itemize}
 \item
       numer strony (polecenie \lstinline|\insertpagenumber|),
 \item
       numer „ramki” (\emph{frame}) \lstinline|\insertframenumber|,
 \item
       numer pierwszej strony aktualnej ramki \lstinline|\insertframestartpage|,
 \item
       numer ostatniej strony aktualnej ramki \lstinline|\insertframeendpage|,
 \item
       numer ostatniej strony dokumentu \lstinline|\insertdocumentendpage|.
\end{itemize}

\begin{frame}
 \frametitle<presentation>{Jak używać}
 \framesubtitle{Opcje klasy beamer}
 Warto również korzystać z~dodatkowych parametrów klasy beamer. Najważniejsze z~nich to:
 \begin{enumerate}
  \item
        \hypertarget{fontsize}{Rozmiar czcionki} definiowany jako:  \alert{8pt}, \alert{9pt}, 10pt, 11pt, 12pt, \alert{14pt}, \alert{17pt}, \alert{20pt}. Rozmiar 17pt to standardowy rozmiar prezentacji PowerPoint i~Impress.
        \mode<article>{Rozmiary oznaczone  kursywą wymagają zainstalowania pakietu extsize.}
  \item
        \hypertarget{proporcje}{Proporcje} ekranu ustala opcja \lstinline|aspectratio|. Standardowo \lstinline|aspectratio=43| (co oznacza proporcje 4:3). Inne dostępne wartości to 1610, 169, 149, 141, 54, 32.

        \mode<article>{
         Znaczenie parametrów jest następujące:
         \begin{description}
          \item [1610] ekran o~proporcjach $16\times10$,
          \item [169] ekran o~proporcjach $16\times9$,
          \item [149] proporcje $14\times9$,
          \item [141] ekran o~proporcjach $\sqrt{2}\times1$ czyli $1{,}41\times1$
          \item [54] ekran o~proporcjach $5\times4$,
          \item [43] (\textbf{standardowo!}) ekran o~proporcjach $4\times3$,
          \item [32] proporcje $3\times2$.
         \end{description}
        }
  \item
        Normalnie zawartość slajdu jest centrowana pionowo. Można to zmienić globalnie używając opcji \lstinline|c| (standard, centrowane) lub \lstinline|t| — umieszczana od góry slajdu.
        \mode<article>{
         Sposób rozmieszczania zawartości można ustalać odrębnie dla każdego slajdu, korzystając z~opcji środowiska \lstinline|frame|: \lstinline|t|, \lstinline|c| lub \lstinline|b|.
        }
  \item
        Inne, warte uwagi parametry, to: \hyperlink{sec:handout}{handout} i~\hyperlink{sec:trans}{trans}.
 \end{enumerate}
\end{frame}

\subsection{Personalizacja}

Pakiet beamer pozwala na bardzo „intensywną” personalizację slajdów i~wyglądu prezentacji. Warto zapoznać się z~dokumentacją \cite{Beamer} (nawet jeżeli liczy ona 250 stron).

\begin{frame}
 \frametitle<presentation>{Personalizacja}
 \begin{enumerate}
  \item
        „Pasek” \hyperlink{pasekpionowy}{pionowy}/\hyperlink{pasekpoziomy}{poziomy},
  \item
        \hyperlink{fonty}{Fonty},
  \item
        Podstawowa \hyperlink{fontsize}{wielkość liter},
  \item
        \hyperlink{efekty}{Efekty przejścia},
  \item
        \hyperlink{proporcje}{Proporcje obrazu},
  \item
        \hyperlink{tytul}{Slajd tytułowy}.
 \end{enumerate}
\end{frame}

\subsection{Fonty}\label{lab:fonty}

Szablon sugeruje użycie fontu Calibri\footnote{Nie jest to nigdzie napisane, ale tak jest ustawiony font domyślny prezentacji (co, zapewne, jest cechą wbudowaną programu PowePoint).}. Najbliższym jego odpowiednikiem w~\LaTeX{}u jest font Carlito.
Zatem możemy zainstalować pakiet \href{https://www.ctan.org/pkg/carlito}{carlito} i~użyć go w~prezentacji.
Alternatywą może być użycie fontu Iwona lub Kurier\footnote{\url{http://jmn.pl/kurier-i-iwona/}}.

Warto też wspomnieć, że w przypadku gdy font Calibri jest zainstalowany w systemie\footnote{Niestety pakiet ttf-mscorefonts-installer nie zawiera tego fontu.} (co zapewne jest standardem systemu Windows) można z niego korzystać kompilując ptrezentację z użyciem LuaLaTeXa.

\begin{frame}[fragile,allowframebreaks]
 \frametitle<presentation>{Fonty}
 \hypertarget{fonty}{\null}
 \begin{enumerate}
  \item Standard.
        Gdy nie zdefiniujemy nic, użyty zostanie standardowy font bezszeryfowy. Jest on bardzo jasny.\\<all>
        \includegraphics[width=.8\textwidth]{default}
  \item Latin Modern.
        \begin{lstlisting}
  \usepackage{lmodern}
  \end{lstlisting}
        W~zasadzie to samo co standardowy, ale wygląda znacznie lepiej.\\<all>
        \includegraphics[width=.8\textwidth]{lmodern}
  \item Carlito \only<presentation>{(darmowy odpowiednik fontu Calibri)}
        \begin{lstlisting}
 \usepackage{carlito}
\end{lstlisting}
        \includegraphics[width=.8\textwidth]{carlito}
        \begin{lstlisting}
 \usepackage[lining]{carlito}
\end{lstlisting}
        \includegraphics[width=.8\textwidth]{carlito_lining}

        \mode<article>{Font Carlito standardowo oferuje cyfry \textbf{nautyczne} (zwane czasami „zepsutymi”).
         Jeżeli zdecydowanie nam nie pasują — należy w~poleceniu dodać dodatkowy parametr (w nawiasach kwadratowych) \lstinline|lining|.}
  \item Font Iwona
        \begin{lstlisting}
 \usepackage{iwona}
\end{lstlisting}
        \includegraphics[width=.8\textwidth]{iwona}
        \only<presentation>{\pagebreak}
  \item Font Kurier
        \begin{lstlisting}
 \usepackage{kurier}
\end{lstlisting}
        \includegraphics[width=.8\textwidth]{kurier}
  \item Trebuchet (był to podstawowy font poprzedniego szablonu).
        Jeżeli ktoś ma go zainstalowanego (i korzystał z~poprzedniego szablonu bez problemów), można użyć go poleceniem:
        \begin{lstlisting}
\renewcommand{\sfdefault}{jtrr}
\end{lstlisting}
        \includegraphics[width=.8\textwidth]{trebuchet}
 \end{enumerate}
 Ale nie są to jedyne możliwości.
\end{frame}

Ja, osobiście bardzo lubię fonty \href{https://www.ctan.org/pkg/iwona}{Iwona} i~\href{https://www.ctan.org/pkg/kurier}{Kurier}, więc bardzo często zamiast fontu Carlito używam właśnie ich.

Doświadczenie uczy, że w~prezentacjach lepiej sprawdzają się fonty bezszeryfowe (\emph{sans serif}) można użyć praktycznie każdego fontu dostępnego w~systemie \hologo{LaTeX}. Na stronie \url{http://www.tug.dk/FontCatalogue/sansseriffonts.html} znajduje się katalog dostępnych fontów. Jeżeli zrezygnować z~\hologo{pdfLaTeX}a na rzecz \hologo{LuaLaTeX}a lub \hologo{XeLaTeX}a — można użyć praktycznie każdego współczesnego fontu dostępnego w~systemie. Warto jednak poczytać dokumentację pakietu \href{https://ctan.org/pkg/fontspec}{fontspec} \cite{Fontspec}.

\subsection{Efekty „przejścia”}

Efekt „przejścia” (\emph{transition effect}) slajdów są (według mnie) czymś złym.
Zajmują czas (nawet jak tylko dwie sekundy), podczas procesu znikania jednego slajdu i~pojawiania się drugiego zawartość jest niedostępna lub nieczytelna, odwracają wreszcie uwagę od zasadniczego tematu.
No, ale są.
Również w~beamerze można z~nich korzystać. Definiuje się je na poziomie warstw lub slajdów.

\begin{frame}
 \frametitle<presentation>{Efekty ,,przejścia''}
 \framesubtitle{Transition effects}
 \transduration<1->{1}
 \transblindshorizontal<2>[duration=1]
 \transblindsvertical<3>[duration=1]
 \transboxin<4>[duration=1]
 \transboxout<5>[duration=1]
 \transcover<6>[duration=1]
 \transdissolve<7>[duration=1]
 \transfade<8>[duration=1]
 \transglitter<9>[duration=1]
 \hypertarget{efekty}{Beamer} dostarcza następujący zestaw efektów:
 \begin{itemize}
  \item<2>
        \textbf{blindshorizontal}
        Show the slide as if horizontal blinds were pulled away.
  \item<3>
        \textbf{blindsvertical}
        Show the slide as if vertical blinds were pulled away.
  \item<4>
        \textbf{boxin}
        Show the slide by moving to the center from all four sides.
  \item<5>
        \textbf{boxout}
        Show the slide by showing more and more of a~rectangular area that is centered on the slide center.
  \item<6>
        \textbf{cover}
        Show the slide by covering the content that was shown before.
  \item<7>
        \textbf{dissolve}
        Show the slide by slowly dissolving what was shown before.
  \item<8>
        \textbf{fade}
        Show the slide by slowly fading what was shown before.
  \item<9>
        \textbf{glitter}
        Show the slide with a~glitter effect that sweeps in the specified direction.
 \end{itemize}
\end{frame}
\begin{frame}
 \frametitle<presentation>{Efekty ,,przejścia'' cd}
 \transduration<1->{1}
 \transpush<1>[duration=1]
 \transsplitverticalin<2>[duration=1]
 \transsplitverticalout<3>[duration=1]
 \transsplithorizontalin<4>[duration=1]
 \transsplithorizontalout<5>[duration=1]
 \transwipe<6>[duration=1]
 %\transreplace<3>[duration=1]
 \begin{itemize}
  \item<1>
        \textbf{push}
        Show the slide by pushing what was shown before off the screen using the new content.
  \item<2>
        \textbf{replace}
        Replace the previous slide directly (default behaviour).
  \item<3>
        \textbf{splitverticalin}
        Show the slide by sweeping two vertical lines from the sides inward.
  \item<4>
        \textbf{splitverticalout}
        Show the slide by sweeping two vertical lines from the center outward.
  \item<5>
        \textbf{splithorizontalin}
        Show the slide by sweeping two horizontal lines from the sides inward.
  \item<6>
        \textbf{splithorizontalout}
        Show the slide by sweeping two horizontal lines from the center outward.
  \item<7>
        \textbf{wipe}
        Show the slide by sweeping a~single line in the specified direction, thereby “wiping out” the previous contents.
 \end{itemize}
\end{frame}

% \begin{frame}
% \frametitle<beamer>{Efekty ,,przejścia''}
% \only<beamer>{Beamer dostarcza następujący zestaw efektów:}
% \begin{itemize}
%\item
% \textbf{splitverticalin} 
%Show the slide by sweeping two vertical lines from the sides inward.
%\item
% \textbf{splitverticalout} 
%Show the slide by sweeping two vertical lines from the center outward.
%\item
% \textbf{splithorizontalin} 
%Show the slide by sweeping two horizontal lines from the sides inward.
%\item
% \textbf{splithorizontalout} 
%Show the slide by sweeping two horizontal lines from the center outward.
%\item
% \textbf{wipe} 
% Show the slide by sweeping a~single line in the specified direction, thereby “wiping out” the previous contents.
%\end{itemize}
%\end{frame}

\begin{frame}[fragile]
 \frametitle{Efekty przejścia}
 \framesubtitle{Uwagi}
 \begin{itemize}
  \item
        Pamiętać natomiast trzeba, że nie każdy program używany do wyświetlania plików PDF wszystkie efekty realizuje.
        Można mieć pewność chyba tylko w~przypadku oryginalnego Adobe Acrobat Readera.
  \item
        Efekt przejścia został zaimplementowany w~taki sposób, że można go wskazać dla wybranych nakładek. Polecenia włączające efekt przejścia dokładnie opisane są w~dokumentacji pakietu beamer\only<article>{~\cite{Beamer}}.
        Tworzy się je przez dodanie do nazwy efektu przedrostka: \lstinline|\trans|. Całe polecenie ma postać:\\
        \lstinline|\trans|\emph{efekt}\lstinline|<|\emph{specyfikacja slajdów}\lstinline|>[|\emph{dodatkowe parametry}\lstinline|]|.
  \item
        \emph{dodatkowe parametry} to, na przykład czas trwanie efektu: \lstinline|duration=1|
 \end{itemize}
\end{frame}

\subsection{Uwagi}

Można \href{https://pwr.edu.pl/uczelnia/informacje-ogolne/materialy-promocyjne/logotyp/}{znaleźć} przygotowany (przez \href{https://pwr.edu.pl/uczelnia/informacje-ogolne/organizacja-uczelni/index,dzial-informacji-i-promocji.html}{Dział Informacji i~Promocji}) materiał zawierający \href{https://kmim.wm.pwr.edu.pl/myszka/wp-content/uploads/sites/2/2017/12/prezentacje_instrukcje.pdf}{przykładowe slajdy} i~możliwości rozmieszczenia tekstu oraz grafik  w~nowych szablonach.
Niektóre efekty bardzo łatwo jest osiągnąć w~beamerze, inne są trudniejsze do uzyskania. W~szczególności modyfikacje slajdu tytułowego wymagają modyfikacji szablonu.

\subsection{Modyfikacje slajdu tytułowego}

\begin{frame}[fragile]
 \frametitle<presentation>{Slajd tytułowy}
 \hypertarget{tytul}{Modyfikacje} slajdu tytułowego są bardzo ograniczone. Standardowo mogą się tam pojawić następujące pola:
 \begin{columns}
  \begin{column}{.5\textwidth}
   \begin{itemize}
    \item
          autor (\lstinline|\author{}|),
    \item
          tytuł (\lstinline|\title{}|),
    \item
          podtytuł (\lstinline|\subtitle{}|),
    \item
          data (\lstinline|\date{}|),
    \item
          instytucja (\lstinline|\institute{}|),
    \item
          grafika (\lstinline|\titlegraphics{}|),
   \end{itemize}
   \only<article>{Slajd tytułowy przedstawia rys.~\ref{title}.}
  \end{column}
  \begin{column}{.48\textwidth}
   \only<article>{\begin{figure}}
     \includegraphics[width=.6\textwidth]{metadane}

     \only<article>{
     \caption{Wygląd slajdu tytułowego}\label{title}
    \end{figure}
   }
  \end{column}
 \end{columns}
\end{frame}



\subsection{Metadane}

\begin{frame}[fragile]
 \frametitle<presentation>{Metadane}
 Zawartość pól:
 \begin{itemize}
  \item
        \lstinline|\title{}| oraz \lstinline|\subtitle{}|,
  \item
        \lstinline|\author{}|,
  \item
        \lstinline|\keywords{}|,
  \item
        \lstinline|\subject{}|,
 \end{itemize}
 automatycznie trafia do odpowiednich pól pliku PDF zawierających metadane. Generalnie metadanie mogą ułatwić zadanie wyszukiwarkom czy poprawnie zdefiniowć autora tekstów umieszczonych w~Internecie.

 Z~umieszczania metadanych w~pliku PDF można zrezygnować:
 \begin{lstlisting}
\documentclass[usepdftitle=false]{beamer}
\end{lstlisting}
\end{frame}

Tak na marginesie: pakiet \hyperref{https://ctan.org/pkg/pdfprivacy}{pdfprivacy} pozwala ma usunięcie wielu innych metadanych zapisywanych standardowo do pliku PDF.

\section{Dokumentacja}

Całą dokumentacja znajduje się w~podkatalogu (patrz rozdział~\ref{instalacja}) \path{doc/pwr}. Składa się na nią:
\begin{frame}
 \frametitle<presentation>{Dokumentacja}
 \only<presentation>{Na dokumentację składa się:}
 \begin{itemize}
  \item
        plik \href{https://kmim.wm.pwr.edu.pl/myszka/wp-content/uploads/sites/2/2017/12/nowy_szablon_doc_article.pdf}{{nowy\_szablon\_doc\_article.pdf}}\only<article>{ (zawierający treść tej strony)};
  \item
        plik \href{https://kmim.wm.pwr.edu.pl/myszka/wp-content/uploads/sites/2/2017/12/nowy_szablon_doc_beamer.pdf}{nowy\_szablon\_doc\_beamer.pdf} będący wersją „prezentacyjną” tej dokumentacji\only<presentation>{ (czyli właśnie to)};
  \item
        pliki źródłowe dokumentacji;
  \item
        przykładowa, prosta prezentacja \href{https://kmim.wm.pwr.edu.pl/myszka/wp-content/uploads/sites/2/2017/12/NewPwr-example.pdf}{NewPwr-example.pdf} wraz ze źródłami;
        \only<presentation>{\item dodatkowym bonusem jest plik \lstinline|oficyna_url.bst| czyli szablon  „\textbf{prawie} zgodny” z~wymaganiami Oficyny Wydawniczej PWr.}
 \end{itemize}
 \only<presentation>{Dokumentacja znajduje się w~podkatalogu \path{doc/pwr}.}
\end{frame}


\section{Materiały informacyjne (handout)}

Aby przygotować materiały informacyjne (tzw. handout) wystarczy wybrać ich format (dwie albo cztery slajdy na kartkę) i~zmodyfikować nieco źródło.
\begin{frame}[fragile]
 \frametitle<presentation>{Materiały informacyjne}
 \framesubtitle<presentation>{(handout)}
 \hypertarget{sec:handout}{}
 \begin{lstlisting}
\documentclass[handout]{beamer}
\usepackage{pgfpages}
\pgfpagesuselayout{2 on 1}[a4paper,
                           border shrink=5mm]
\end{lstlisting}
 może też być
 \begin{lstlisting}
\pgfpagesuselayout{4 on 1}[a4paper,landscape,
                           border shrink=5mm]
\end{lstlisting}
 Standardowo materiały informacyjne zostaną wygenerowane używając „pełnej wersji” slajdów (po uwzględnieniu wszystkich informacji „przyrostowych”).
\end{frame}
Można eksperymentować z~innymi opcjami oszczędnościowymi i~umieszczać więcej slajdów na jednej stronie.

Generalna zasada podczas tworzenia materiałów informacyjnych jest taka, że zostaje umieszczona w~nich „ostateczne” wersja slajdu (czyli po uwzględnieniu wszystkich przyrostów). Można jednak zdecydować inaczej — używając parametru handout w~definicji przyrostu.

\section{Folie}

Pakiet beamer może również być użyty do tworzenia folii (transparencies). Nie bardzo wiem w~jakiej sytuacji może się to dziś przydać — ale istnieje taka możliwość.

\begin{frame}[fragile]
 \frametitle<presentation>{Folie}
 \framesubtitle{trans}
 \hypertarget{sec:trans}{}
 \begin{lstlisting}
\documentclass[trans,17pt,aspectratio=43]{beamer}
\end{lstlisting}
 \begin{itemize}
  \item
        Standardowo folie zostaną wygenerowane używając „pełnej wersji” slajdów (po uwzględnieniu wszystkich informacji „przyrostowych”).
  \item
        Mądrze korzystając ze wszystkich możliwości oferowanych przez „kompilację warunkową”  można w~ten sposób wygenerować „lżejszą” (mniej atramentożerną) wersję prezentacji.
 \end{itemize}

\end{frame}

Działanie opcji \lstinline|trans| jest zbliżone do działania opcji \lstinline|hangout|: użyte będą „ostateczne” wersje slajdów (czyli po uwzględnieniu wszystkich przyrostów).

%\section{Drugi ekran}
%
%Second screen

\section{Tłumaczenia}

Pakiet beamer już od pewnego czasu używa narzędzia bardzo podobnego do narzędzi używanych w „profesjonalnych” pakietach oprogramowania, a służącego do łatwiejszego tworzenia dokumentów wielojęzycznych.

Pakiet nazywa się \href{https://ctan.org/pkg/translator}{translator}. Osobiście nie wiem, czy to dobry pomysł, czy zły, ale tak jest.

W szczególności korzystanie ze środowisk: \lstinline|example|, \lstinline|theorem|, \lstinline|definition|, czy poleceń \lstinline|\partpage|, \lstinline|\sectionpage|, \lstinline|\subsectionpage|,…

%
%\section{Narzędzia pomocnicze}
%
%\subsection{pdfpc}
%
%\url{https://pdfpc.github.io/}
%
%\subsection{pympress}
%
%\url{https://github.com/Cimbali/pympress}
%
%\subsection{PDF Presenter}
%
%\url{http://pdfpresenter.sourceforge.net/}
%




\section{Do zrobienia}

\begin{frame}[fragile,allowframebreaks]
 \frametitle<presentation>{Do zrobienia}
 \begin{enumerate}
  \item
        Szczegółowe:
        \begin{itemize}
         \item[\done]
               Numery slajdów
         \item[\done]
               Numer strony na stronie tytułowej
         \item[\emptybox]
               Poprawne pozycjonowanie ilustracji wstawianej poleceniem \lstinline|\titlegraphics| dla wersji pionowej
         \item[\emptybox]
               Personalizacja strony tytułowej
         \item[\emptybox]
               Wygląd bloków (zwłaszcza alertblock) i~twierdzeń, dowodów,… % powiązać z~kolorkiem szarym z~szablonu (dla poziomego?)
         \item[\done]
               Slajdy plain (bez tła)
         \item[\emptybox]
               \emph{Progress bar?}
               \item [\emptybox] Zoom
               \item [\emptybox] Kompresja slajdów
        \end{itemize}
  \item Ogólne:
        \begin{itemize}
         \item[\emptybox]
               Notatki
         \item[\emptybox]
               Drugi ekran
         \item[\emptybox]
               Narzędzia (Linux)
               \item[\emptybox] 
               Animacje — ogólne uwagi i przykłady
        \end{itemize}
        \mode<presentation>{\pagebreak}
  \item Zgłoszone na kursie uwagi/błędy:
        \begin{itemize}
         \item[\emptybox] Kolorystyka spisu treści (Agenda)
         \item[\done] Numeracja sekcji w spisie treści
          \item[\done] Raczej numer slajdu niż numer strony PDF
        \end{itemize}
  \item Zauważone błędy (niedoróbki):
        \begin{itemize}
         \item[\done]
               Numeracja slajdów (slajd 11???)
        \end{itemize}
 \end{enumerate}
\end{frame}

%\subsection{Komentarze}
%
%W maju byłem uczestnikiem organizowanego na Poltechnice Wrocławskiej w ramach projektu Innowacyjna Uczelnia --- Innowacyjny Nauczyciel szkolenia \LaTeX{} tworzenie profesjonalnej prezentacji. Zgłoszono mi tam pewne propozycje ulepszenia szablonu.

\section{Uwagi}

\subsection{Struktura archiwum/plików w~pakiecie}

{
 \footnotesize
 \lstinputlisting{tree.txt}
}

\subsection{Pliki dodatkowe}

\subsubsection{Szablon bibliografii zgodny z~wymaganiami Oficyny Wydawniczej PWr}

W pakiecie znajduje się również styl bibliografii \path{oficyna\_url.bst}. Powstał om podczas składania różnych publikacji wydawanych przez Oficynę Wydawniczą PWr i~\textbf{znacznym stopniu} spełnia wymagania edytorskie. (Choć te, w~zależności od Redaktora potrafią się trochę różnić, czy może inaczej: różni się odporność Redaktorów na drobne niezgodności.)

Plik można pobrać również \href{http://kmim.wm.pwr.edu.pl/myszka/wp-content/uploads/sites/2/2018/11/oficyna_url.bst}{osobno}.

\subsubsection{Testy}

W pod-kartotece z~dokumentacją znajduje się pod-kartoteka beamer-benchmark. W~niej znajdują się pliki zaczerpnięte z~repozytorium \href{https://github.com/louisstuart96/beamer-benchmark}{beamer-benchmark} \href{https://github.com/louisstuart96}{Luisa Stuarta}. Pozwalają one „testować” różne elementy szablonu.

\subsection{Stara wersja szablonu…}

…jest ciągle \href{https://kmim.wm.pwr.edu.pl/myszka/projekty/szablon-prezentacji-pwr/szablon-prezentacji-zgodny-z-ksiega-logotypu/}{dostępna}.

%\subsection{}

%\end{document}





\begin{document}

\begin{frame}[plain]
 \maketitle
\end{frame}

\ifpdf
 \tableofcontents
\fi

\section{Terminologia}

Jest pewien problem z~terminologią, a~zwłaszcza z~bardzo sobie bliskimi pojęciami. W~dokumentacji beamera używane są pojęcia:
\begin{itemize}
 \item
       frame,
 \item
       slide,
 \item
       overlay,
 \item
       page,
 \item
       hangout.
\end{itemize}
Trzeba to jakoś przetłumaczyć na Polski.

\begin{frame}[fragile]
 \frametitle<presentation>{Terminologia}
 \begin{itemize}
  \item
        strona (\emph{page}) — plik PDF składa się z~wielu stron i~widzimy tylko jedną z~nich na ekranie;
  \item
        slajd (\emph{frame}) — podstawowa jednostka logiczna informacji wyświetlanej na ekranie; składać się na nią \textbf{może} kilka \textbf{nakładek} (\textbf{warstw}),
  \item
        nakładka/warstwa (\emph{overlay}) — z~punktu widzenia pliku PDF będzie to jedna \textbf{strona} zawierająca, najczęściej, przyrostowo dodawane informacje składające się na \textbf{slajd}.

        \mode<article>{Instrukcje przyrostowego dodawania informacji do slajdu opisuje dokumentacja pakietu beamer \cite[Rozdział~9]{Beamer}.}
 \end{itemize}
 Występuje tez pojęcie „folia” (\emph{transparency}).

 Pojęcie \emph{hangout} tłumaczę jako „materiały informacyjne”.
\end{frame}
Materiały informacyjne to, po prostu, wszystkie slajdy wydrukowane po 2, 4,… na kartce papieru.

Rozróżnienie to będzie zwłaszcza istotne w~przypadku produkowaniu materiałów informacyjnych czy drukowania folii.

\section{Wprowadzenie}

Od chwili opracowania pierwszej wersji logotypu Politechniki Wrocławskiej \cite{siw2004} upłynęło sporo czasu — \difftoday{2004}{10}{21}. Prawdę mówiąc nie przez wszystkich został on przyjęty jednakowo entuzjastycznie. Bardzo wiele osób krytykowało pewne jego „przesztywnienie”. I~o~ile bardzo restrykcyjne zasady używania znaku Politechniki  Wrocławskiej uznać należy za zasadne, to już prezentacja nie dawała żadnych możliwości wykazania się inwencją. Z~drugiej strony — ciągle uważam, że znacznie ważniejsza od formy jest treść.

Pierwszym „objawem” buntu było pojawienie się „rektorskiej” (jak się dowiedziałem) wersji prezentacji. Pierwotnie używana ona była, między innymi, do prezentowania rocznych Sprawozdań Rektora (rys.~\ref{rektorska}).

\begin{frame}
 \frametitle<presentation>{Szablon „rektorski”}
 \begin{figure}
  \mode<presentation>{
   \includegraphics[width=\textwidth,height=.7\textheight,keepaspectratio]{rektorska}
  }
  \mode<article>{
   \centering\includegraphics[width=.5\textwidth]{rektorska}
  }
  \caption{Wygląd prezentacji z~„rektorską” wersją szablonu}\label{rektorska}
 \end{figure}
\end{frame}

Szablon ten nigdy nie był oficjalnie ogłoszony, a~dziś jako „poprzednia wersję szablonu”\footnote{\url{https://pwr.edu.pl/uczelnia/informacje-ogolne/materialy-promocyjne/logotyp}} wskazywana jest wersja oryginalna. Natomiast dało się zauważyć, że wiele osób z~tej wersji szablonu korzystało (a nawet ciągle korzysta — co i~rusz pojawiają się takie prezentacje).

Z drugiej strony daje się również zauważyć, że całkiem oficjalne prezentacje (nawet osób z~Kierownictwa) nie zawsze korzystają z~obowiązującego szablonu. Ale Kierownictwu wolno więcej.

W roku 2016 została ogłoszona nowa wersja Systemu Identyfikacji Wizualnej \cite{siw2016}. Prezentowana jest ona na stronach Uczelni \cite{Logotyp}. Podstawowa zmiana wiąże się ze zmianą angielskiej nazwy Politechniki Wrocławskiej z~\emph{Wrocław Uniwersity of Technology} na \emph{Wrocław University of Science and Technology}.

Kolejne zmiany związane były z~uzyskaniem przez Politechnikę Wrocławską prawa do używania logo „HR Excellence in Research”\footnote{\url{https://pwr.edu.pl/uczelnia/europejska-karta-naukowca/}.}. Logo trafiło zarówno na papier firmowy jak i~na stronę tytułową prezentacji.

\begin{frame}{Logo HR Excellence in Research}
 \includegraphics<presentation>[width=\textwidth,height=.7\textheight,keepaspectratio]{Hr_p1.pdf}
 \includegraphics<article>[width=.2\textwidth]{Hr_p1}
\end{frame}

Zmieniły się też szablony prezentacji. Wybrano niezbyt łatwy do odwzorowania w~\LaTeX{}u szablon: tło nie jest jednolicie białe, tylko w~kratkę, która zmienia swoje natężenie w~różnych miejscach. Ale zasadniczo szablon jest podobny do ,,rektorskiego'', tylko występuje w~dwu wariantach.

Jeden nazywać będę „poziomym” (poziomy pasek z~logo uczelni), a~drugi pionowym.

\begin{frame}
 \frametitle<presentation>{Tła nowego szablonu w~wersji polskiej i~angielskiej}
 \mode<article>{
  \begin{figure}
   \includegraphics[width=.24\textwidth]{image1_p1_pl}
   \includegraphics[width=.24\textwidth]{image1_p2_pl}
   \includegraphics[width=.24\textwidth]{image1_p1_en}
   \includegraphics[width=.24\textwidth]{image1_p2_en}

   \includegraphics[width=.24\textwidth]{image2_p1_pl}
   \includegraphics[width=.24\textwidth]{image2_p2_pl}
   \includegraphics[width=.24\textwidth]{image2_p1_en}
   \includegraphics[width=.24\textwidth]{image2_p2_en}
   \caption{Tła strony tytułowej (pierwszy rząd) i~kolejnych stron nowego szablonu}
  \end{figure}
 }
 \mode<presentation>{
  \framesubtitle<1>{tytułowa, horizontal, polski}
  \framesubtitle<2>{tytułowa, vertical, polski}
  \framesubtitle<3>{tytułowa, horizontal, angielski}
  \framesubtitle<4>{tytułowa, vertical, angielski}
  \framesubtitle<5>{normalna, horizontal, polski}
  \framesubtitle<6>{normalna, vertical, polski}
  \framesubtitle<7>{normalna, horizontal, angielski}
  \framesubtitle<8>{normalna,vertical, angielski}
  \transduration<1>{1}
  \transdissolve<1>[duration=1]
  \transduration<2>{1}
  \transdissolve<2>[duration=1]
  \transduration<3>{1}
  \transdissolve<3>[duration=1]
  \transduration<4>{1}
  \transdissolve<4>[duration=1]
  \transduration<5>{1}
  \transdissolve<5>[duration=1]
  \transduration<6>{1}
  \transdissolve<6>[duration=1]
  \transduration<7>{1}
  \transdissolve<7>[duration=1]
  \only<1>{\includegraphics[width=\textwidth,height=.7\textheight,keepaspectratio]{image1_p1_pl}}
  \only<2>{\includegraphics[width=\textwidth,height=.7\textheight,keepaspectratio]{image1_p2_pl}}
  \only<3>{\includegraphics[width=\textwidth,height=.7\textheight,keepaspectratio]{image1_p1_en}}
  \only<4>{\includegraphics[width=\textwidth,height=.7\textheight,keepaspectratio]{image1_p2_en}}
  \only<5>{\includegraphics[width=\textwidth,height=.7\textheight,keepaspectratio]{image2_p1_pl}}
  \only<6>{\includegraphics[width=\textwidth,height=.7\textheight,keepaspectratio]{image2_p2_pl}}
  \only<7>{\includegraphics[width=\textwidth,height=.7\textheight,keepaspectratio]{image2_p1_en}}
  \only<8>{\includegraphics[width=\textwidth,height=.7\textheight,keepaspectratio]{image2_p2_en}}
 }
\end{frame}

\section{Jak instalować}\label{instalacja}

Najlepiej by było, gdyby pakiet stał się częścią archiwum \href{https://ctan.org/}{CTAN}. Nie bardzo mam ochotę próbować dostosowywać pakiet do (wysokich) standardów tam panujących. Pozostaniemy zatem (przynajmniej na razie) przy instalacji ręcznej.

W każdym razie plik archiwum zgodny jest ze strukturą TDS~\cite{tds} i~wystarczy pliki rozpakować.

\begin{frame}[fragile]
 \frametitle<presentation>{Jak instalować}
 \only<presentation>{Należy pliki ręcznie skopiować we \textbf{właściwe} miejsce.}

 Właściwe miejsce zdefiniowane jest przez zmienne:
 \begin{itemize}
  \item
        TEXMFLOCAL (dla plików dostępnych dla wszystkich użytkowników)
        \begin{itemize}
         \item \path{/usr/local/texlive/texmf-local} (standardowo Unix/Linux)
         \item \path{%SystemDrive%\texlive\texmf-local} (Windows)
        \end{itemize}
        oraz
  \item
        TEXMFHOME (dla plików dostępnych dla aktualnego użytkownika); kartoteka texmf w~kartotece \path{$HOME} (Unix, Linux) lub \path (Windows).
  \item
        Jeszcze inaczej jest w~przypadku MiKTeXa. Opisuje to dokumentacja\only<article>{~\cite{miktex}}. Kluczową aplikacją będzie Miktex Options (w~skrócie mo).
 \end{itemize}
\end{frame}

Można też wszystkie pliki z~podkartoteki \lstinline|tex/latex/pwr| skopiować do kartoteki, w~której tworzony jest dokument. Tej metody jednak nie polecam.

\begin{frame}[fragile]
 \frametitle{Instalacja}
 \begin{enumerate}
  \item Linux: Aby ściągnąć plik archiwum trzeba sięgnąć pod \href{https://kmim.wm.pwr.edu.pl/myszka/logotyp/nowy_szablon.tar.xz}{ten adres}. Najprościej będzie wykonać następujące polecenia:
        \ifpdf
         \begin{lstlisting}[language=bash]
cd /tmp
wget https://kmim.wm.pwr.edu.pl/myszka/logotyp/nowy_szablon.tar.xz
mkdir -p ~/texmf
cd ~/texmf
tar xJf /tmp/nowy_szablon.tar.xz
\end{lstlisting}
        \else
         \begin{verbatim}
cd /tmp
wget https://kmim.wm.pwr.edu.pl/myszka/logotyp/nowy_szablon.tar.xz
mkdir -p ~/texmf
cd ~/texmf
tar xJf /tmp/nowy_szablon.tar.xz
\end{verbatim}
        \fi
  \item Windows: Aby ściągnąć plik archiwum trzeba sięgnąć pod \href{https://kmim.wm.pwr.edu.pl/myszka/logotyp/nowy_szablon.zip}{ten adres}. Później postępujemy zgodnie z~wymaganiami używanego do dekompresji i~rozpakowywania programu. Kiedyś\only<article>{\footnote{Ale ja od dawna nie używam Windows i~nawet nie mam komputera, żeby to sprawdzić!}} po kliknięciu prawym klawiszem myszy na archiwum zip pojawiało się coś podobnego do „Wyodrębnij tutaj”.
 \end{enumerate}
\end{frame}

\subsection{Wersja oszczędna}

Oprócz tego istnieje „wersja oszczędna” szablonu zawierająca wszystko za wyjątkiem plików źródłowych i~dokumentacji szablonu. Plik można pobrać \href{https://kmim.wm.pwr.edu.pl/myszka/logotyp/nowy_szablon_maly.tar.xz}{tu (wyłącznie wersja tar.xz)}.

\subsection{Używane pakiety}

Potrzebne będą „standardowe” pakiety potrzebne do pracy z~beamerem. Są to:
\begin{enumerate}
 \item
       ifxetex
 \item
       oberdiek
 \item
       pgf
 \item
       amsfonts
 \item
       amsmath
 \item
       beamer
 \item
       etoolbox
 \item
       geometry
 \item
       graphics
 \item
       hyperref
 \item
       enumerate
 \item
       xcolor
 \item
       extsizes (gdy żądamy innych niż standardowe rozmiary fontów).
\end{enumerate}
\begin{frame}
 \frametitle{Dodatkowe pakiety}

 Oprócz pakietów standardowych \only<presentation>{(niezbędnych do pracy z~beamerem)} nie są potrzebne żadne pakiety dodatkowe.

 W~przypadku gdy zechcemy użyć innych niż standardowe fontów potrzebne mogą być pakiety je instalujące, na przykład:
 \begin{itemize}
  \item
        carlito — gdy chcemy naśladować standardowy font PowerPointa: Calibri\only<article>{ (patrz również rozdział~\ref{lab:fonty})}.
 \end{itemize}
\end{frame}

\section{Jak używać}

\hologo{pdfLaTeX}:

\begin{frame}[fragile]
 \frametitle<presentation>{Jak używać?}
 \framesubtitle{\hologo{pdfLaTeX}}
 Użycie szablonu jest bardzo proste:
 \ifpdf
  \begin{lstlisting}  
\documentclass[ ]{beamer}
\usepackage[utf8]{inputenc}
\usepackage[T1]{fontenc}
\usepackage{carlito}
\usetheme[horizontal=true]{NewPwr}
\end{lstlisting}
 \else
  \begin{verbatim}  
\documentclass[ ]{beamer}
\usepackage[utf8]{inputenc}
\usepackage[T1]{fontenc}
\usepackage{carlito}
\usetheme[horizontal=true]{NewPwr}
\end{verbatim}
 \fi
\end{frame}

\hologo{LuaLaTeX}:

\begin{frame}[fragile]
 \frametitle<presentation>{Jak używać?}
 \framesubtitle{\hologo{LuaLaTeX}}
 Użycie szablonu jest równie proste:
 \ifpdf
  \begin{lstlisting}  
\documentclass[ ]{beamer}
\usepackage{fontspec}
\setsansfont{Carlito}[Numbers=OldStyle]
\usetheme[horizontal=true]{NewPwr}
\end{lstlisting}
 \else
  \begin{verbatim}  
\documentclass[ ]{beamer}
\usepackage{fontspec}
\setsansfont{Carlito}[Numbers=OldStyle]
\usetheme[horizontal=true]{NewPwr}
\end{verbatim}
 \fi
\end{frame}

W przypadku \hologo{LuaLaTeX}a trzeba jednak dokładnie przestudować dokumentację tego programu, żeby poprawnie korzystać z~fontów (i wszystkiego co one oferują).

Nie powinno być wększych problemów podczas stosowana \hologo{XeLaTeX}a.  Ale (w tej wersji dokumentacji) nie będę o~tym pisał.

%\end{document}

\begin{frame}[fragile,allowframebreaks]
 \frametitle<presentation>{Jak używać}
 \framesubtitle{Opcje szablonu PwrNew}
 W~poleceniu \lstinline|\usetheme[ ]{NewPwr}|, w~nawiasach kwadratowych podajemy opcje szablonu. Można wybrać język i~tło.
 \begin{enumerate}
  \item
        Język wybieramy podając opcję \lstinline|lang=en| lub \lstinline|lang=pl|, gdy jej nie podamy — wybrany będzie język polski.
  \item
        Rodzaj tła (pasek poziomy lub pasek pionowy) podajemy deklarując \lstinline|horizontal=true| albo \lstinline|vertical=true|; można skrócić do samego \hypertarget{pasekpoziomy}{\lstinline|horizontal|}, \hypertarget{pasekpionowy}{\lstinline|vertical|}.
        Gdy nie podamy — będzie \lstinline|vertical|.
        %        \mode<article>{
        %        \begin{figure}
        %        \caption{aaaa}
        %        \end{figure}
        %        }
  \item
        Ponieważ gdzieniegdzie funkcjonuje jeszcze wersja ,,rektorska'' szablonu, można użyć (niezalecanej) opcji \lstinline|rektor=true| lub, po prostu \lstinline|rektor|\only<article>{\footnote{Uwaga! Nie znalazłem nigdzie angielskiej wersji „Rektorskiego” szablonu/prezentacji.}}. Szablon występuje wyłącznie w~wersji polskojęzycznej i~dla ekranów o~\lstinline|aspectratio=43|.
  \item
        Uważam, że nie zawsze musi być prezentowane logo HR Excellence in Research. Władze uczelni są innego zdania%
        \only<article>{\footnote{W piśmie z~dnia 31 lipca 2017, Rektor Jasieńko pisze tak: „Proszę również o~umieszczanie logo HR […] we wszystkich prezentacjach […]”.}}%
        \ i~w~związku z~tym domyślnie logo \textbf{jest umieszczane}.   Można je wyłączyć używając opcji \lstinline|hr=false|.
        \only<article>{\end{enumerate}}
 \mode<article>{
  „Standardowe” szablony, na każdej stronie — z~wyjątkiem tytułowej — zawierają numer strony. Niektórzy tego nie lubią. Inni się tego domagają. Osobiście nie mam zdania. Zatem powinno to trafić do opcji.
  
  Natomiast w beamerze jest pewien problem z numeracją: Ze względu na to, że każdy slajd zbudowany może być z kilku warstw (nakładek) numer strony PDF zazwyczaj różni się od numeru kolejnego slajdu.
 }
 \only<article>{\begin{enumerate}[resume]}
  \item Opcja \lstinline|pagenumbers=frame| włącza numerowanie slajdów; na każdym slajdzie będzie umieszczony jego numer. Opcja \lstinline|pagenumbers=page|  lub \lstinline|pagenumbers=true| lub  lub \lstinline|pagenumbers| umieszcza na slajdzie numer strony. (Standardowo numerowanie jest wyłączone!) Strona tytułowa prezentacji nie jest numerowana!
  \item Od „zawsze”, standardowo, na dole każdego slajdu beamer umieszczał symbole nawigacji. Ponieważ, mało kto z~tego korzysta — są one wyłączone. Można je włączyć umieszczając wśród opcji szablonu \lstinline|navigation| lub \lstinline|navigation=true|.
 \end{enumerate}
\end{frame}

Tak na marginesie, beamer oferuje możliwość włączenia następujących informacji (\cite[Rozdział~8.2.1]{Beamer}): 
\begin{itemize}
 \item
       numer strony (polecenie \lstinline|\insertpagenumber|),
 \item
       numer „ramki” (\emph{frame}) \lstinline|\insertframenumber|,
 \item
       numer pierwszej strony aktualnej ramki \lstinline|\insertframestartpage|,
 \item
       numer ostatniej strony aktualnej ramki \lstinline|\insertframeendpage|,
 \item
       numer ostatniej strony dokumentu \lstinline|\insertdocumentendpage|.
\end{itemize}

\begin{frame}
 \frametitle<presentation>{Jak używać}
 \framesubtitle{Opcje klasy beamer}
 Warto również korzystać z~dodatkowych parametrów klasy beamer. Najważniejsze z~nich to:
 \begin{enumerate}
  \item
        \hypertarget{fontsize}{Rozmiar czcionki} definiowany jako:  \alert{8pt}, \alert{9pt}, 10pt, 11pt, 12pt, \alert{14pt}, \alert{17pt}, \alert{20pt}. Rozmiar 17pt to standardowy rozmiar prezentacji PowerPoint i~Impress.
        \mode<article>{Rozmiary oznaczone  kursywą wymagają zainstalowania pakietu extsize.}
  \item
        \hypertarget{proporcje}{Proporcje} ekranu ustala opcja \lstinline|aspectratio|. Standardowo \lstinline|aspectratio=43| (co oznacza proporcje 4:3). Inne dostępne wartości to 1610, 169, 149, 141, 54, 32.

        \mode<article>{
         Znaczenie parametrów jest następujące:
         \begin{description}
          \item [1610] ekran o~proporcjach $16\times10$,
          \item [169] ekran o~proporcjach $16\times9$,
          \item [149] proporcje $14\times9$,
          \item [141] ekran o~proporcjach $\sqrt{2}\times1$ czyli $1{,}41\times1$
          \item [54] ekran o~proporcjach $5\times4$,
          \item [43] (\textbf{standardowo!}) ekran o~proporcjach $4\times3$,
          \item [32] proporcje $3\times2$.
         \end{description}
        }
  \item
        Normalnie zawartość slajdu jest centrowana pionowo. Można to zmienić globalnie używając opcji \lstinline|c| (standard, centrowane) lub \lstinline|t| — umieszczana od góry slajdu.
        \mode<article>{
         Sposób rozmieszczania zawartości można ustalać odrębnie dla każdego slajdu, korzystając z~opcji środowiska \lstinline|frame|: \lstinline|t|, \lstinline|c| lub \lstinline|b|.
        }
  \item
        Inne, warte uwagi parametry, to: \hyperlink{sec:handout}{handout} i~\hyperlink{sec:trans}{trans}.
 \end{enumerate}
\end{frame}

\subsection{Personalizacja}

Pakiet beamer pozwala na bardzo „intensywną” personalizację slajdów i~wyglądu prezentacji. Warto zapoznać się z~dokumentacją \cite{Beamer} (nawet jeżeli liczy ona 250 stron).

\begin{frame}
 \frametitle<presentation>{Personalizacja}
 \begin{enumerate}
  \item
        „Pasek” \hyperlink{pasekpionowy}{pionowy}/\hyperlink{pasekpoziomy}{poziomy},
  \item
        \hyperlink{fonty}{Fonty},
  \item
        Podstawowa \hyperlink{fontsize}{wielkość liter},
  \item
        \hyperlink{efekty}{Efekty przejścia},
  \item
        \hyperlink{proporcje}{Proporcje obrazu},
  \item
        \hyperlink{tytul}{Slajd tytułowy}.
 \end{enumerate}
\end{frame}

\subsection{Fonty}\label{lab:fonty}

Szablon sugeruje użycie fontu Calibri\footnote{Nie jest to nigdzie napisane, ale tak jest ustawiony font domyślny prezentacji (co, zapewne, jest cechą wbudowaną programu PowePoint).}. Najbliższym jego odpowiednikiem w~\LaTeX{}u jest font Carlito.
Zatem możemy zainstalować pakiet \href{https://www.ctan.org/pkg/carlito}{carlito} i~użyć go w~prezentacji.
Alternatywą może być użycie fontu Iwona lub Kurier\footnote{\url{http://jmn.pl/kurier-i-iwona/}}.

Warto też wspomnieć, że w przypadku gdy font Calibri jest zainstalowany w systemie\footnote{Niestety pakiet ttf-mscorefonts-installer nie zawiera tego fontu.} (co zapewne jest standardem systemu Windows) można z niego korzystać kompilując ptrezentację z użyciem LuaLaTeXa.

\begin{frame}[fragile,allowframebreaks]
 \frametitle<presentation>{Fonty}
 \hypertarget{fonty}{\null}
 \begin{enumerate}
  \item Standard.
        Gdy nie zdefiniujemy nic, użyty zostanie standardowy font bezszeryfowy. Jest on bardzo jasny.\\<all>
        \includegraphics[width=.8\textwidth]{default}
  \item Latin Modern.
        \begin{lstlisting}
  \usepackage{lmodern}
  \end{lstlisting}
        W~zasadzie to samo co standardowy, ale wygląda znacznie lepiej.\\<all>
        \includegraphics[width=.8\textwidth]{lmodern}
  \item Carlito \only<presentation>{(darmowy odpowiednik fontu Calibri)}
        \begin{lstlisting}
 \usepackage{carlito}
\end{lstlisting}
        \includegraphics[width=.8\textwidth]{carlito}
        \begin{lstlisting}
 \usepackage[lining]{carlito}
\end{lstlisting}
        \includegraphics[width=.8\textwidth]{carlito_lining}

        \mode<article>{Font Carlito standardowo oferuje cyfry \textbf{nautyczne} (zwane czasami „zepsutymi”).
         Jeżeli zdecydowanie nam nie pasują — należy w~poleceniu dodać dodatkowy parametr (w nawiasach kwadratowych) \lstinline|lining|.}
  \item Font Iwona
        \begin{lstlisting}
 \usepackage{iwona}
\end{lstlisting}
        \includegraphics[width=.8\textwidth]{iwona}
        \only<presentation>{\pagebreak}
  \item Font Kurier
        \begin{lstlisting}
 \usepackage{kurier}
\end{lstlisting}
        \includegraphics[width=.8\textwidth]{kurier}
  \item Trebuchet (był to podstawowy font poprzedniego szablonu).
        Jeżeli ktoś ma go zainstalowanego (i korzystał z~poprzedniego szablonu bez problemów), można użyć go poleceniem:
        \begin{lstlisting}
\renewcommand{\sfdefault}{jtrr}
\end{lstlisting}
        \includegraphics[width=.8\textwidth]{trebuchet}
 \end{enumerate}
 Ale nie są to jedyne możliwości.
\end{frame}

Ja, osobiście bardzo lubię fonty \href{https://www.ctan.org/pkg/iwona}{Iwona} i~\href{https://www.ctan.org/pkg/kurier}{Kurier}, więc bardzo często zamiast fontu Carlito używam właśnie ich.

Doświadczenie uczy, że w~prezentacjach lepiej sprawdzają się fonty bezszeryfowe (\emph{sans serif}) można użyć praktycznie każdego fontu dostępnego w~systemie \hologo{LaTeX}. Na stronie \url{http://www.tug.dk/FontCatalogue/sansseriffonts.html} znajduje się katalog dostępnych fontów. Jeżeli zrezygnować z~\hologo{pdfLaTeX}a na rzecz \hologo{LuaLaTeX}a lub \hologo{XeLaTeX}a — można użyć praktycznie każdego współczesnego fontu dostępnego w~systemie. Warto jednak poczytać dokumentację pakietu \href{https://ctan.org/pkg/fontspec}{fontspec} \cite{Fontspec}.

\subsection{Efekty „przejścia”}

Efekt „przejścia” (\emph{transition effect}) slajdów są (według mnie) czymś złym.
Zajmują czas (nawet jak tylko dwie sekundy), podczas procesu znikania jednego slajdu i~pojawiania się drugiego zawartość jest niedostępna lub nieczytelna, odwracają wreszcie uwagę od zasadniczego tematu.
No, ale są.
Również w~beamerze można z~nich korzystać. Definiuje się je na poziomie warstw lub slajdów.

\begin{frame}
 \frametitle<presentation>{Efekty ,,przejścia''}
 \framesubtitle{Transition effects}
 \transduration<1->{1}
 \transblindshorizontal<2>[duration=1]
 \transblindsvertical<3>[duration=1]
 \transboxin<4>[duration=1]
 \transboxout<5>[duration=1]
 \transcover<6>[duration=1]
 \transdissolve<7>[duration=1]
 \transfade<8>[duration=1]
 \transglitter<9>[duration=1]
 \hypertarget{efekty}{Beamer} dostarcza następujący zestaw efektów:
 \begin{itemize}
  \item<2>
        \textbf{blindshorizontal}
        Show the slide as if horizontal blinds were pulled away.
  \item<3>
        \textbf{blindsvertical}
        Show the slide as if vertical blinds were pulled away.
  \item<4>
        \textbf{boxin}
        Show the slide by moving to the center from all four sides.
  \item<5>
        \textbf{boxout}
        Show the slide by showing more and more of a~rectangular area that is centered on the slide center.
  \item<6>
        \textbf{cover}
        Show the slide by covering the content that was shown before.
  \item<7>
        \textbf{dissolve}
        Show the slide by slowly dissolving what was shown before.
  \item<8>
        \textbf{fade}
        Show the slide by slowly fading what was shown before.
  \item<9>
        \textbf{glitter}
        Show the slide with a~glitter effect that sweeps in the specified direction.
 \end{itemize}
\end{frame}
\begin{frame}
 \frametitle<presentation>{Efekty ,,przejścia'' cd}
 \transduration<1->{1}
 \transpush<1>[duration=1]
 \transsplitverticalin<2>[duration=1]
 \transsplitverticalout<3>[duration=1]
 \transsplithorizontalin<4>[duration=1]
 \transsplithorizontalout<5>[duration=1]
 \transwipe<6>[duration=1]
 %\transreplace<3>[duration=1]
 \begin{itemize}
  \item<1>
        \textbf{push}
        Show the slide by pushing what was shown before off the screen using the new content.
  \item<2>
        \textbf{replace}
        Replace the previous slide directly (default behaviour).
  \item<3>
        \textbf{splitverticalin}
        Show the slide by sweeping two vertical lines from the sides inward.
  \item<4>
        \textbf{splitverticalout}
        Show the slide by sweeping two vertical lines from the center outward.
  \item<5>
        \textbf{splithorizontalin}
        Show the slide by sweeping two horizontal lines from the sides inward.
  \item<6>
        \textbf{splithorizontalout}
        Show the slide by sweeping two horizontal lines from the center outward.
  \item<7>
        \textbf{wipe}
        Show the slide by sweeping a~single line in the specified direction, thereby “wiping out” the previous contents.
 \end{itemize}
\end{frame}

% \begin{frame}
% \frametitle<beamer>{Efekty ,,przejścia''}
% \only<beamer>{Beamer dostarcza następujący zestaw efektów:}
% \begin{itemize}
%\item
% \textbf{splitverticalin} 
%Show the slide by sweeping two vertical lines from the sides inward.
%\item
% \textbf{splitverticalout} 
%Show the slide by sweeping two vertical lines from the center outward.
%\item
% \textbf{splithorizontalin} 
%Show the slide by sweeping two horizontal lines from the sides inward.
%\item
% \textbf{splithorizontalout} 
%Show the slide by sweeping two horizontal lines from the center outward.
%\item
% \textbf{wipe} 
% Show the slide by sweeping a~single line in the specified direction, thereby “wiping out” the previous contents.
%\end{itemize}
%\end{frame}

\begin{frame}[fragile]
 \frametitle{Efekty przejścia}
 \framesubtitle{Uwagi}
 \begin{itemize}
  \item
        Pamiętać natomiast trzeba, że nie każdy program używany do wyświetlania plików PDF wszystkie efekty realizuje.
        Można mieć pewność chyba tylko w~przypadku oryginalnego Adobe Acrobat Readera.
  \item
        Efekt przejścia został zaimplementowany w~taki sposób, że można go wskazać dla wybranych nakładek. Polecenia włączające efekt przejścia dokładnie opisane są w~dokumentacji pakietu beamer\only<article>{~\cite{Beamer}}.
        Tworzy się je przez dodanie do nazwy efektu przedrostka: \lstinline|\trans|. Całe polecenie ma postać:\\
        \lstinline|\trans|\emph{efekt}\lstinline|<|\emph{specyfikacja slajdów}\lstinline|>[|\emph{dodatkowe parametry}\lstinline|]|.
  \item
        \emph{dodatkowe parametry} to, na przykład czas trwanie efektu: \lstinline|duration=1|
 \end{itemize}
\end{frame}

\subsection{Uwagi}

Można \href{https://pwr.edu.pl/uczelnia/informacje-ogolne/materialy-promocyjne/logotyp/}{znaleźć} przygotowany (przez \href{https://pwr.edu.pl/uczelnia/informacje-ogolne/organizacja-uczelni/index,dzial-informacji-i-promocji.html}{Dział Informacji i~Promocji}) materiał zawierający \href{https://kmim.wm.pwr.edu.pl/myszka/wp-content/uploads/sites/2/2017/12/prezentacje_instrukcje.pdf}{przykładowe slajdy} i~możliwości rozmieszczenia tekstu oraz grafik  w~nowych szablonach.
Niektóre efekty bardzo łatwo jest osiągnąć w~beamerze, inne są trudniejsze do uzyskania. W~szczególności modyfikacje slajdu tytułowego wymagają modyfikacji szablonu.

\subsection{Modyfikacje slajdu tytułowego}

\begin{frame}[fragile]
 \frametitle<presentation>{Slajd tytułowy}
 \hypertarget{tytul}{Modyfikacje} slajdu tytułowego są bardzo ograniczone. Standardowo mogą się tam pojawić następujące pola:
 \begin{columns}
  \begin{column}{.5\textwidth}
   \begin{itemize}
    \item
          autor (\lstinline|\author{}|),
    \item
          tytuł (\lstinline|\title{}|),
    \item
          podtytuł (\lstinline|\subtitle{}|),
    \item
          data (\lstinline|\date{}|),
    \item
          instytucja (\lstinline|\institute{}|),
    \item
          grafika (\lstinline|\titlegraphics{}|),
   \end{itemize}
   \only<article>{Slajd tytułowy przedstawia rys.~\ref{title}.}
  \end{column}
  \begin{column}{.48\textwidth}
   \only<article>{\begin{figure}}
     \includegraphics[width=.6\textwidth]{metadane}

     \only<article>{
     \caption{Wygląd slajdu tytułowego}\label{title}
    \end{figure}
   }
  \end{column}
 \end{columns}
\end{frame}



\subsection{Metadane}

\begin{frame}[fragile]
 \frametitle<presentation>{Metadane}
 Zawartość pól:
 \begin{itemize}
  \item
        \lstinline|\title{}| oraz \lstinline|\subtitle{}|,
  \item
        \lstinline|\author{}|,
  \item
        \lstinline|\keywords{}|,
  \item
        \lstinline|\subject{}|,
 \end{itemize}
 automatycznie trafia do odpowiednich pól pliku PDF zawierających metadane. Generalnie metadanie mogą ułatwić zadanie wyszukiwarkom czy poprawnie zdefiniowć autora tekstów umieszczonych w~Internecie.

 Z~umieszczania metadanych w~pliku PDF można zrezygnować:
 \begin{lstlisting}
\documentclass[usepdftitle=false]{beamer}
\end{lstlisting}
\end{frame}

Tak na marginesie: pakiet \hyperref{https://ctan.org/pkg/pdfprivacy}{pdfprivacy} pozwala ma usunięcie wielu innych metadanych zapisywanych standardowo do pliku PDF.

\section{Dokumentacja}

Całą dokumentacja znajduje się w~podkatalogu (patrz rozdział~\ref{instalacja}) \path{doc/pwr}. Składa się na nią:
\begin{frame}
 \frametitle<presentation>{Dokumentacja}
 \only<presentation>{Na dokumentację składa się:}
 \begin{itemize}
  \item
        plik \href{https://kmim.wm.pwr.edu.pl/myszka/wp-content/uploads/sites/2/2017/12/nowy_szablon_doc_article.pdf}{{nowy\_szablon\_doc\_article.pdf}}\only<article>{ (zawierający treść tej strony)};
  \item
        plik \href{https://kmim.wm.pwr.edu.pl/myszka/wp-content/uploads/sites/2/2017/12/nowy_szablon_doc_beamer.pdf}{nowy\_szablon\_doc\_beamer.pdf} będący wersją „prezentacyjną” tej dokumentacji\only<presentation>{ (czyli właśnie to)};
  \item
        pliki źródłowe dokumentacji;
  \item
        przykładowa, prosta prezentacja \href{https://kmim.wm.pwr.edu.pl/myszka/wp-content/uploads/sites/2/2017/12/NewPwr-example.pdf}{NewPwr-example.pdf} wraz ze źródłami;
        \only<presentation>{\item dodatkowym bonusem jest plik \lstinline|oficyna_url.bst| czyli szablon  „\textbf{prawie} zgodny” z~wymaganiami Oficyny Wydawniczej PWr.}
 \end{itemize}
 \only<presentation>{Dokumentacja znajduje się w~podkatalogu \path{doc/pwr}.}
\end{frame}


\section{Materiały informacyjne (handout)}

Aby przygotować materiały informacyjne (tzw. handout) wystarczy wybrać ich format (dwie albo cztery slajdy na kartkę) i~zmodyfikować nieco źródło.
\begin{frame}[fragile]
 \frametitle<presentation>{Materiały informacyjne}
 \framesubtitle<presentation>{(handout)}
 \hypertarget{sec:handout}{}
 \begin{lstlisting}
\documentclass[handout]{beamer}
\usepackage{pgfpages}
\pgfpagesuselayout{2 on 1}[a4paper,
                           border shrink=5mm]
\end{lstlisting}
 może też być
 \begin{lstlisting}
\pgfpagesuselayout{4 on 1}[a4paper,landscape,
                           border shrink=5mm]
\end{lstlisting}
 Standardowo materiały informacyjne zostaną wygenerowane używając „pełnej wersji” slajdów (po uwzględnieniu wszystkich informacji „przyrostowych”).
\end{frame}
Można eksperymentować z~innymi opcjami oszczędnościowymi i~umieszczać więcej slajdów na jednej stronie.

Generalna zasada podczas tworzenia materiałów informacyjnych jest taka, że zostaje umieszczona w~nich „ostateczne” wersja slajdu (czyli po uwzględnieniu wszystkich przyrostów). Można jednak zdecydować inaczej — używając parametru handout w~definicji przyrostu.

\section{Folie}

Pakiet beamer może również być użyty do tworzenia folii (transparencies). Nie bardzo wiem w~jakiej sytuacji może się to dziś przydać — ale istnieje taka możliwość.

\begin{frame}[fragile]
 \frametitle<presentation>{Folie}
 \framesubtitle{trans}
 \hypertarget{sec:trans}{}
 \begin{lstlisting}
\documentclass[trans,17pt,aspectratio=43]{beamer}
\end{lstlisting}
 \begin{itemize}
  \item
        Standardowo folie zostaną wygenerowane używając „pełnej wersji” slajdów (po uwzględnieniu wszystkich informacji „przyrostowych”).
  \item
        Mądrze korzystając ze wszystkich możliwości oferowanych przez „kompilację warunkową”  można w~ten sposób wygenerować „lżejszą” (mniej atramentożerną) wersję prezentacji.
 \end{itemize}

\end{frame}

Działanie opcji \lstinline|trans| jest zbliżone do działania opcji \lstinline|hangout|: użyte będą „ostateczne” wersje slajdów (czyli po uwzględnieniu wszystkich przyrostów).

%\section{Drugi ekran}
%
%Second screen

\section{Tłumaczenia}

Pakiet beamer już od pewnego czasu używa narzędzia bardzo podobnego do narzędzi używanych w „profesjonalnych” pakietach oprogramowania, a służącego do łatwiejszego tworzenia dokumentów wielojęzycznych.

Pakiet nazywa się \href{https://ctan.org/pkg/translator}{translator}. Osobiście nie wiem, czy to dobry pomysł, czy zły, ale tak jest.

W szczególności korzystanie ze środowisk: \lstinline|example|, \lstinline|theorem|, \lstinline|definition|, czy poleceń \lstinline|\partpage|, \lstinline|\sectionpage|, \lstinline|\subsectionpage|,…

%
%\section{Narzędzia pomocnicze}
%
%\subsection{pdfpc}
%
%\url{https://pdfpc.github.io/}
%
%\subsection{pympress}
%
%\url{https://github.com/Cimbali/pympress}
%
%\subsection{PDF Presenter}
%
%\url{http://pdfpresenter.sourceforge.net/}
%




\section{Do zrobienia}

\begin{frame}[fragile,allowframebreaks]
 \frametitle<presentation>{Do zrobienia}
 \begin{enumerate}
  \item
        Szczegółowe:
        \begin{itemize}
         \item[\done]
               Numery slajdów
         \item[\done]
               Numer strony na stronie tytułowej
         \item[\emptybox]
               Poprawne pozycjonowanie ilustracji wstawianej poleceniem \lstinline|\titlegraphics| dla wersji pionowej
         \item[\emptybox]
               Personalizacja strony tytułowej
         \item[\emptybox]
               Wygląd bloków (zwłaszcza alertblock) i~twierdzeń, dowodów,… % powiązać z~kolorkiem szarym z~szablonu (dla poziomego?)
         \item[\done]
               Slajdy plain (bez tła)
         \item[\emptybox]
               \emph{Progress bar?}
               \item [\emptybox] Zoom
               \item [\emptybox] Kompresja slajdów
        \end{itemize}
  \item Ogólne:
        \begin{itemize}
         \item[\emptybox]
               Notatki
         \item[\emptybox]
               Drugi ekran
         \item[\emptybox]
               Narzędzia (Linux)
               \item[\emptybox] 
               Animacje — ogólne uwagi i przykłady
        \end{itemize}
        \mode<presentation>{\pagebreak}
  \item Zgłoszone na kursie uwagi/błędy:
        \begin{itemize}
         \item[\emptybox] Kolorystyka spisu treści (Agenda)
         \item[\done] Numeracja sekcji w spisie treści
          \item[\done] Raczej numer slajdu niż numer strony PDF
        \end{itemize}
  \item Zauważone błędy (niedoróbki):
        \begin{itemize}
         \item[\done]
               Numeracja slajdów (slajd 11???)
        \end{itemize}
 \end{enumerate}
\end{frame}

%\subsection{Komentarze}
%
%W maju byłem uczestnikiem organizowanego na Poltechnice Wrocławskiej w ramach projektu Innowacyjna Uczelnia --- Innowacyjny Nauczyciel szkolenia \LaTeX{} tworzenie profesjonalnej prezentacji. Zgłoszono mi tam pewne propozycje ulepszenia szablonu.

\section{Uwagi}

\subsection{Struktura archiwum/plików w~pakiecie}

{
 \footnotesize
 \lstinputlisting{tree.txt}
}

\subsection{Pliki dodatkowe}

\subsubsection{Szablon bibliografii zgodny z~wymaganiami Oficyny Wydawniczej PWr}

W pakiecie znajduje się również styl bibliografii \path{oficyna\_url.bst}. Powstał om podczas składania różnych publikacji wydawanych przez Oficynę Wydawniczą PWr i~\textbf{znacznym stopniu} spełnia wymagania edytorskie. (Choć te, w~zależności od Redaktora potrafią się trochę różnić, czy może inaczej: różni się odporność Redaktorów na drobne niezgodności.)

Plik można pobrać również \href{http://kmim.wm.pwr.edu.pl/myszka/wp-content/uploads/sites/2/2018/11/oficyna_url.bst}{osobno}.

\subsubsection{Testy}

W pod-kartotece z~dokumentacją znajduje się pod-kartoteka beamer-benchmark. W~niej znajdują się pliki zaczerpnięte z~repozytorium \href{https://github.com/louisstuart96/beamer-benchmark}{beamer-benchmark} \href{https://github.com/louisstuart96}{Luisa Stuarta}. Pozwalają one „testować” różne elementy szablonu.

\subsection{Stara wersja szablonu…}

…jest ciągle \href{https://kmim.wm.pwr.edu.pl/myszka/projekty/szablon-prezentacji-pwr/szablon-prezentacji-zgodny-z-ksiega-logotypu/}{dostępna}.

%\subsection{}

%\end{document}





\begin{document}

\begin{frame}[plain]
 \maketitle
\end{frame}

\ifpdf
 \tableofcontents
\fi

\section{Terminologia}

Jest pewien problem z~terminologią, a~zwłaszcza z~bardzo sobie bliskimi pojęciami. W~dokumentacji beamera używane są pojęcia:
\begin{itemize}
 \item
       frame,
 \item
       slide,
 \item
       overlay,
 \item
       page,
 \item
       hangout.
\end{itemize}
Trzeba to jakoś przetłumaczyć na Polski.

\begin{frame}[fragile]
 \frametitle<presentation>{Terminologia}
 \begin{itemize}
  \item
        strona (\emph{page}) — plik PDF składa się z~wielu stron i~widzimy tylko jedną z~nich na ekranie;
  \item
        slajd (\emph{frame}) — podstawowa jednostka logiczna informacji wyświetlanej na ekranie; składać się na nią \textbf{może} kilka \textbf{nakładek} (\textbf{warstw}),
  \item
        nakładka/warstwa (\emph{overlay}) — z~punktu widzenia pliku PDF będzie to jedna \textbf{strona} zawierająca, najczęściej, przyrostowo dodawane informacje składające się na \textbf{slajd}.

        \mode<article>{Instrukcje przyrostowego dodawania informacji do slajdu opisuje dokumentacja pakietu beamer \cite[Rozdział~9]{Beamer}.}
 \end{itemize}
 Występuje tez pojęcie „folia” (\emph{transparency}).

 Pojęcie \emph{hangout} tłumaczę jako „materiały informacyjne”.
\end{frame}
Materiały informacyjne to, po prostu, wszystkie slajdy wydrukowane po 2, 4,… na kartce papieru.

Rozróżnienie to będzie zwłaszcza istotne w~przypadku produkowaniu materiałów informacyjnych czy drukowania folii.

\section{Wprowadzenie}

Od chwili opracowania pierwszej wersji logotypu Politechniki Wrocławskiej \cite{siw2004} upłynęło sporo czasu — \difftoday{2004}{10}{21}. Prawdę mówiąc nie przez wszystkich został on przyjęty jednakowo entuzjastycznie. Bardzo wiele osób krytykowało pewne jego „przesztywnienie”. I~o~ile bardzo restrykcyjne zasady używania znaku Politechniki  Wrocławskiej uznać należy za zasadne, to już prezentacja nie dawała żadnych możliwości wykazania się inwencją. Z~drugiej strony — ciągle uważam, że znacznie ważniejsza od formy jest treść.

Pierwszym „objawem” buntu było pojawienie się „rektorskiej” (jak się dowiedziałem) wersji prezentacji. Pierwotnie używana ona była, między innymi, do prezentowania rocznych Sprawozdań Rektora (rys.~\ref{rektorska}).

\begin{frame}
 \frametitle<presentation>{Szablon „rektorski”}
 \begin{figure}
  \mode<presentation>{
   \includegraphics[width=\textwidth,height=.7\textheight,keepaspectratio]{rektorska}
  }
  \mode<article>{
   \centering\includegraphics[width=.5\textwidth]{rektorska}
  }
  \caption{Wygląd prezentacji z~„rektorską” wersją szablonu}\label{rektorska}
 \end{figure}
\end{frame}

Szablon ten nigdy nie był oficjalnie ogłoszony, a~dziś jako „poprzednia wersję szablonu”\footnote{\url{https://pwr.edu.pl/uczelnia/informacje-ogolne/materialy-promocyjne/logotyp}} wskazywana jest wersja oryginalna. Natomiast dało się zauważyć, że wiele osób z~tej wersji szablonu korzystało (a nawet ciągle korzysta — co i~rusz pojawiają się takie prezentacje).

Z drugiej strony daje się również zauważyć, że całkiem oficjalne prezentacje (nawet osób z~Kierownictwa) nie zawsze korzystają z~obowiązującego szablonu. Ale Kierownictwu wolno więcej.

W roku 2016 została ogłoszona nowa wersja Systemu Identyfikacji Wizualnej \cite{siw2016}. Prezentowana jest ona na stronach Uczelni \cite{Logotyp}. Podstawowa zmiana wiąże się ze zmianą angielskiej nazwy Politechniki Wrocławskiej z~\emph{Wrocław Uniwersity of Technology} na \emph{Wrocław University of Science and Technology}.

Kolejne zmiany związane były z~uzyskaniem przez Politechnikę Wrocławską prawa do używania logo „HR Excellence in Research”\footnote{\url{https://pwr.edu.pl/uczelnia/europejska-karta-naukowca/}.}. Logo trafiło zarówno na papier firmowy jak i~na stronę tytułową prezentacji.

\begin{frame}{Logo HR Excellence in Research}
 \includegraphics<presentation>[width=\textwidth,height=.7\textheight,keepaspectratio]{Hr_p1.pdf}
 \includegraphics<article>[width=.2\textwidth]{Hr_p1}
\end{frame}

Zmieniły się też szablony prezentacji. Wybrano niezbyt łatwy do odwzorowania w~\LaTeX{}u szablon: tło nie jest jednolicie białe, tylko w~kratkę, która zmienia swoje natężenie w~różnych miejscach. Ale zasadniczo szablon jest podobny do ,,rektorskiego'', tylko występuje w~dwu wariantach.

Jeden nazywać będę „poziomym” (poziomy pasek z~logo uczelni), a~drugi pionowym.

\begin{frame}
 \frametitle<presentation>{Tła nowego szablonu w~wersji polskiej i~angielskiej}
 \mode<article>{
  \begin{figure}
   \includegraphics[width=.24\textwidth]{image1_p1_pl}
   \includegraphics[width=.24\textwidth]{image1_p2_pl}
   \includegraphics[width=.24\textwidth]{image1_p1_en}
   \includegraphics[width=.24\textwidth]{image1_p2_en}

   \includegraphics[width=.24\textwidth]{image2_p1_pl}
   \includegraphics[width=.24\textwidth]{image2_p2_pl}
   \includegraphics[width=.24\textwidth]{image2_p1_en}
   \includegraphics[width=.24\textwidth]{image2_p2_en}
   \caption{Tła strony tytułowej (pierwszy rząd) i~kolejnych stron nowego szablonu}
  \end{figure}
 }
 \mode<presentation>{
  \framesubtitle<1>{tytułowa, horizontal, polski}
  \framesubtitle<2>{tytułowa, vertical, polski}
  \framesubtitle<3>{tytułowa, horizontal, angielski}
  \framesubtitle<4>{tytułowa, vertical, angielski}
  \framesubtitle<5>{normalna, horizontal, polski}
  \framesubtitle<6>{normalna, vertical, polski}
  \framesubtitle<7>{normalna, horizontal, angielski}
  \framesubtitle<8>{normalna,vertical, angielski}
  \transduration<1>{1}
  \transdissolve<1>[duration=1]
  \transduration<2>{1}
  \transdissolve<2>[duration=1]
  \transduration<3>{1}
  \transdissolve<3>[duration=1]
  \transduration<4>{1}
  \transdissolve<4>[duration=1]
  \transduration<5>{1}
  \transdissolve<5>[duration=1]
  \transduration<6>{1}
  \transdissolve<6>[duration=1]
  \transduration<7>{1}
  \transdissolve<7>[duration=1]
  \only<1>{\includegraphics[width=\textwidth,height=.7\textheight,keepaspectratio]{image1_p1_pl}}
  \only<2>{\includegraphics[width=\textwidth,height=.7\textheight,keepaspectratio]{image1_p2_pl}}
  \only<3>{\includegraphics[width=\textwidth,height=.7\textheight,keepaspectratio]{image1_p1_en}}
  \only<4>{\includegraphics[width=\textwidth,height=.7\textheight,keepaspectratio]{image1_p2_en}}
  \only<5>{\includegraphics[width=\textwidth,height=.7\textheight,keepaspectratio]{image2_p1_pl}}
  \only<6>{\includegraphics[width=\textwidth,height=.7\textheight,keepaspectratio]{image2_p2_pl}}
  \only<7>{\includegraphics[width=\textwidth,height=.7\textheight,keepaspectratio]{image2_p1_en}}
  \only<8>{\includegraphics[width=\textwidth,height=.7\textheight,keepaspectratio]{image2_p2_en}}
 }
\end{frame}

\section{Jak instalować}\label{instalacja}

Najlepiej by było, gdyby pakiet stał się częścią archiwum \href{https://ctan.org/}{CTAN}. Nie bardzo mam ochotę próbować dostosowywać pakiet do (wysokich) standardów tam panujących. Pozostaniemy zatem (przynajmniej na razie) przy instalacji ręcznej.

W każdym razie plik archiwum zgodny jest ze strukturą TDS~\cite{tds} i~wystarczy pliki rozpakować.

\begin{frame}[fragile]
 \frametitle<presentation>{Jak instalować}
 \only<presentation>{Należy pliki ręcznie skopiować we \textbf{właściwe} miejsce.}

 Właściwe miejsce zdefiniowane jest przez zmienne:
 \begin{itemize}
  \item
        TEXMFLOCAL (dla plików dostępnych dla wszystkich użytkowników)
        \begin{itemize}
         \item \path{/usr/local/texlive/texmf-local} (standardowo Unix/Linux)
         \item \path{%SystemDrive%\texlive\texmf-local} (Windows)
        \end{itemize}
        oraz
  \item
        TEXMFHOME (dla plików dostępnych dla aktualnego użytkownika); kartoteka texmf w~kartotece \path{$HOME} (Unix, Linux) lub \path (Windows).
  \item
        Jeszcze inaczej jest w~przypadku MiKTeXa. Opisuje to dokumentacja\only<article>{~\cite{miktex}}. Kluczową aplikacją będzie Miktex Options (w~skrócie mo).
 \end{itemize}
\end{frame}

Można też wszystkie pliki z~podkartoteki \lstinline|tex/latex/pwr| skopiować do kartoteki, w~której tworzony jest dokument. Tej metody jednak nie polecam.

\begin{frame}[fragile]
 \frametitle{Instalacja}
 \begin{enumerate}
  \item Linux: Aby ściągnąć plik archiwum trzeba sięgnąć pod \href{https://kmim.wm.pwr.edu.pl/myszka/logotyp/nowy_szablon.tar.xz}{ten adres}. Najprościej będzie wykonać następujące polecenia:
        \ifpdf
         \begin{lstlisting}[language=bash]
cd /tmp
wget https://kmim.wm.pwr.edu.pl/myszka/logotyp/nowy_szablon.tar.xz
mkdir -p ~/texmf
cd ~/texmf
tar xJf /tmp/nowy_szablon.tar.xz
\end{lstlisting}
        \else
         \begin{verbatim}
cd /tmp
wget https://kmim.wm.pwr.edu.pl/myszka/logotyp/nowy_szablon.tar.xz
mkdir -p ~/texmf
cd ~/texmf
tar xJf /tmp/nowy_szablon.tar.xz
\end{verbatim}
        \fi
  \item Windows: Aby ściągnąć plik archiwum trzeba sięgnąć pod \href{https://kmim.wm.pwr.edu.pl/myszka/logotyp/nowy_szablon.zip}{ten adres}. Później postępujemy zgodnie z~wymaganiami używanego do dekompresji i~rozpakowywania programu. Kiedyś\only<article>{\footnote{Ale ja od dawna nie używam Windows i~nawet nie mam komputera, żeby to sprawdzić!}} po kliknięciu prawym klawiszem myszy na archiwum zip pojawiało się coś podobnego do „Wyodrębnij tutaj”.
 \end{enumerate}
\end{frame}

\subsection{Wersja oszczędna}

Oprócz tego istnieje „wersja oszczędna” szablonu zawierająca wszystko za wyjątkiem plików źródłowych i~dokumentacji szablonu. Plik można pobrać \href{https://kmim.wm.pwr.edu.pl/myszka/logotyp/nowy_szablon_maly.tar.xz}{tu (wyłącznie wersja tar.xz)}.

\subsection{Używane pakiety}

Potrzebne będą „standardowe” pakiety potrzebne do pracy z~beamerem. Są to:
\begin{enumerate}
 \item
       ifxetex
 \item
       oberdiek
 \item
       pgf
 \item
       amsfonts
 \item
       amsmath
 \item
       beamer
 \item
       etoolbox
 \item
       geometry
 \item
       graphics
 \item
       hyperref
 \item
       enumerate
 \item
       xcolor
 \item
       extsizes (gdy żądamy innych niż standardowe rozmiary fontów).
\end{enumerate}
\begin{frame}
 \frametitle{Dodatkowe pakiety}

 Oprócz pakietów standardowych \only<presentation>{(niezbędnych do pracy z~beamerem)} nie są potrzebne żadne pakiety dodatkowe.

 W~przypadku gdy zechcemy użyć innych niż standardowe fontów potrzebne mogą być pakiety je instalujące, na przykład:
 \begin{itemize}
  \item
        carlito — gdy chcemy naśladować standardowy font PowerPointa: Calibri\only<article>{ (patrz również rozdział~\ref{lab:fonty})}.
 \end{itemize}
\end{frame}

\section{Jak używać}

\hologo{pdfLaTeX}:

\begin{frame}[fragile]
 \frametitle<presentation>{Jak używać?}
 \framesubtitle{\hologo{pdfLaTeX}}
 Użycie szablonu jest bardzo proste:
 \ifpdf
  \begin{lstlisting}  
\documentclass[ ]{beamer}
\usepackage[utf8]{inputenc}
\usepackage[T1]{fontenc}
\usepackage{carlito}
\usetheme[horizontal=true]{NewPwr}
\end{lstlisting}
 \else
  \begin{verbatim}  
\documentclass[ ]{beamer}
\usepackage[utf8]{inputenc}
\usepackage[T1]{fontenc}
\usepackage{carlito}
\usetheme[horizontal=true]{NewPwr}
\end{verbatim}
 \fi
\end{frame}

\hologo{LuaLaTeX}:

\begin{frame}[fragile]
 \frametitle<presentation>{Jak używać?}
 \framesubtitle{\hologo{LuaLaTeX}}
 Użycie szablonu jest równie proste:
 \ifpdf
  \begin{lstlisting}  
\documentclass[ ]{beamer}
\usepackage{fontspec}
\setsansfont{Carlito}[Numbers=OldStyle]
\usetheme[horizontal=true]{NewPwr}
\end{lstlisting}
 \else
  \begin{verbatim}  
\documentclass[ ]{beamer}
\usepackage{fontspec}
\setsansfont{Carlito}[Numbers=OldStyle]
\usetheme[horizontal=true]{NewPwr}
\end{verbatim}
 \fi
\end{frame}

W przypadku \hologo{LuaLaTeX}a trzeba jednak dokładnie przestudować dokumentację tego programu, żeby poprawnie korzystać z~fontów (i wszystkiego co one oferują).

Nie powinno być wększych problemów podczas stosowana \hologo{XeLaTeX}a.  Ale (w tej wersji dokumentacji) nie będę o~tym pisał.

%\end{document}

\begin{frame}[fragile,allowframebreaks]
 \frametitle<presentation>{Jak używać}
 \framesubtitle{Opcje szablonu PwrNew}
 W~poleceniu \lstinline|\usetheme[ ]{NewPwr}|, w~nawiasach kwadratowych podajemy opcje szablonu. Można wybrać język i~tło.
 \begin{enumerate}
  \item
        Język wybieramy podając opcję \lstinline|lang=en| lub \lstinline|lang=pl|, gdy jej nie podamy — wybrany będzie język polski.
  \item
        Rodzaj tła (pasek poziomy lub pasek pionowy) podajemy deklarując \lstinline|horizontal=true| albo \lstinline|vertical=true|; można skrócić do samego \hypertarget{pasekpoziomy}{\lstinline|horizontal|}, \hypertarget{pasekpionowy}{\lstinline|vertical|}.
        Gdy nie podamy — będzie \lstinline|vertical|.
        %        \mode<article>{
        %        \begin{figure}
        %        \caption{aaaa}
        %        \end{figure}
        %        }
  \item
        Ponieważ gdzieniegdzie funkcjonuje jeszcze wersja ,,rektorska'' szablonu, można użyć (niezalecanej) opcji \lstinline|rektor=true| lub, po prostu \lstinline|rektor|\only<article>{\footnote{Uwaga! Nie znalazłem nigdzie angielskiej wersji „Rektorskiego” szablonu/prezentacji.}}. Szablon występuje wyłącznie w~wersji polskojęzycznej i~dla ekranów o~\lstinline|aspectratio=43|.
  \item
        Uważam, że nie zawsze musi być prezentowane logo HR Excellence in Research. Władze uczelni są innego zdania%
        \only<article>{\footnote{W piśmie z~dnia 31 lipca 2017, Rektor Jasieńko pisze tak: „Proszę również o~umieszczanie logo HR […] we wszystkich prezentacjach […]”.}}%
        \ i~w~związku z~tym domyślnie logo \textbf{jest umieszczane}.   Można je wyłączyć używając opcji \lstinline|hr=false|.
        \only<article>{\end{enumerate}}
 \mode<article>{
  „Standardowe” szablony, na każdej stronie — z~wyjątkiem tytułowej — zawierają numer strony. Niektórzy tego nie lubią. Inni się tego domagają. Osobiście nie mam zdania. Zatem powinno to trafić do opcji.
  
  Natomiast w beamerze jest pewien problem z numeracją: Ze względu na to, że każdy slajd zbudowany może być z kilku warstw (nakładek) numer strony PDF zazwyczaj różni się od numeru kolejnego slajdu.
 }
 \only<article>{\begin{enumerate}[resume]}
  \item Opcja \lstinline|pagenumbers=frame| włącza numerowanie slajdów; na każdym slajdzie będzie umieszczony jego numer. Opcja \lstinline|pagenumbers=page|  lub \lstinline|pagenumbers=true| lub  lub \lstinline|pagenumbers| umieszcza na slajdzie numer strony. (Standardowo numerowanie jest wyłączone!) Strona tytułowa prezentacji nie jest numerowana!
  \item Od „zawsze”, standardowo, na dole każdego slajdu beamer umieszczał symbole nawigacji. Ponieważ, mało kto z~tego korzysta — są one wyłączone. Można je włączyć umieszczając wśród opcji szablonu \lstinline|navigation| lub \lstinline|navigation=true|.
 \end{enumerate}
\end{frame}

Tak na marginesie, beamer oferuje możliwość włączenia następujących informacji (\cite[Rozdział~8.2.1]{Beamer}): 
\begin{itemize}
 \item
       numer strony (polecenie \lstinline|\insertpagenumber|),
 \item
       numer „ramki” (\emph{frame}) \lstinline|\insertframenumber|,
 \item
       numer pierwszej strony aktualnej ramki \lstinline|\insertframestartpage|,
 \item
       numer ostatniej strony aktualnej ramki \lstinline|\insertframeendpage|,
 \item
       numer ostatniej strony dokumentu \lstinline|\insertdocumentendpage|.
\end{itemize}

\begin{frame}
 \frametitle<presentation>{Jak używać}
 \framesubtitle{Opcje klasy beamer}
 Warto również korzystać z~dodatkowych parametrów klasy beamer. Najważniejsze z~nich to:
 \begin{enumerate}
  \item
        \hypertarget{fontsize}{Rozmiar czcionki} definiowany jako:  \alert{8pt}, \alert{9pt}, 10pt, 11pt, 12pt, \alert{14pt}, \alert{17pt}, \alert{20pt}. Rozmiar 17pt to standardowy rozmiar prezentacji PowerPoint i~Impress.
        \mode<article>{Rozmiary oznaczone  kursywą wymagają zainstalowania pakietu extsize.}
  \item
        \hypertarget{proporcje}{Proporcje} ekranu ustala opcja \lstinline|aspectratio|. Standardowo \lstinline|aspectratio=43| (co oznacza proporcje 4:3). Inne dostępne wartości to 1610, 169, 149, 141, 54, 32.

        \mode<article>{
         Znaczenie parametrów jest następujące:
         \begin{description}
          \item [1610] ekran o~proporcjach $16\times10$,
          \item [169] ekran o~proporcjach $16\times9$,
          \item [149] proporcje $14\times9$,
          \item [141] ekran o~proporcjach $\sqrt{2}\times1$ czyli $1{,}41\times1$
          \item [54] ekran o~proporcjach $5\times4$,
          \item [43] (\textbf{standardowo!}) ekran o~proporcjach $4\times3$,
          \item [32] proporcje $3\times2$.
         \end{description}
        }
  \item
        Normalnie zawartość slajdu jest centrowana pionowo. Można to zmienić globalnie używając opcji \lstinline|c| (standard, centrowane) lub \lstinline|t| — umieszczana od góry slajdu.
        \mode<article>{
         Sposób rozmieszczania zawartości można ustalać odrębnie dla każdego slajdu, korzystając z~opcji środowiska \lstinline|frame|: \lstinline|t|, \lstinline|c| lub \lstinline|b|.
        }
  \item
        Inne, warte uwagi parametry, to: \hyperlink{sec:handout}{handout} i~\hyperlink{sec:trans}{trans}.
 \end{enumerate}
\end{frame}

\subsection{Personalizacja}

Pakiet beamer pozwala na bardzo „intensywną” personalizację slajdów i~wyglądu prezentacji. Warto zapoznać się z~dokumentacją \cite{Beamer} (nawet jeżeli liczy ona 250 stron).

\begin{frame}
 \frametitle<presentation>{Personalizacja}
 \begin{enumerate}
  \item
        „Pasek” \hyperlink{pasekpionowy}{pionowy}/\hyperlink{pasekpoziomy}{poziomy},
  \item
        \hyperlink{fonty}{Fonty},
  \item
        Podstawowa \hyperlink{fontsize}{wielkość liter},
  \item
        \hyperlink{efekty}{Efekty przejścia},
  \item
        \hyperlink{proporcje}{Proporcje obrazu},
  \item
        \hyperlink{tytul}{Slajd tytułowy}.
 \end{enumerate}
\end{frame}

\subsection{Fonty}\label{lab:fonty}

Szablon sugeruje użycie fontu Calibri\footnote{Nie jest to nigdzie napisane, ale tak jest ustawiony font domyślny prezentacji (co, zapewne, jest cechą wbudowaną programu PowePoint).}. Najbliższym jego odpowiednikiem w~\LaTeX{}u jest font Carlito.
Zatem możemy zainstalować pakiet \href{https://www.ctan.org/pkg/carlito}{carlito} i~użyć go w~prezentacji.
Alternatywą może być użycie fontu Iwona lub Kurier\footnote{\url{http://jmn.pl/kurier-i-iwona/}}.

Warto też wspomnieć, że w przypadku gdy font Calibri jest zainstalowany w systemie\footnote{Niestety pakiet ttf-mscorefonts-installer nie zawiera tego fontu.} (co zapewne jest standardem systemu Windows) można z niego korzystać kompilując ptrezentację z użyciem LuaLaTeXa.

\begin{frame}[fragile,allowframebreaks]
 \frametitle<presentation>{Fonty}
 \hypertarget{fonty}{\null}
 \begin{enumerate}
  \item Standard.
        Gdy nie zdefiniujemy nic, użyty zostanie standardowy font bezszeryfowy. Jest on bardzo jasny.\\<all>
        \includegraphics[width=.8\textwidth]{default}
  \item Latin Modern.
        \begin{lstlisting}
  \usepackage{lmodern}
  \end{lstlisting}
        W~zasadzie to samo co standardowy, ale wygląda znacznie lepiej.\\<all>
        \includegraphics[width=.8\textwidth]{lmodern}
  \item Carlito \only<presentation>{(darmowy odpowiednik fontu Calibri)}
        \begin{lstlisting}
 \usepackage{carlito}
\end{lstlisting}
        \includegraphics[width=.8\textwidth]{carlito}
        \begin{lstlisting}
 \usepackage[lining]{carlito}
\end{lstlisting}
        \includegraphics[width=.8\textwidth]{carlito_lining}

        \mode<article>{Font Carlito standardowo oferuje cyfry \textbf{nautyczne} (zwane czasami „zepsutymi”).
         Jeżeli zdecydowanie nam nie pasują — należy w~poleceniu dodać dodatkowy parametr (w nawiasach kwadratowych) \lstinline|lining|.}
  \item Font Iwona
        \begin{lstlisting}
 \usepackage{iwona}
\end{lstlisting}
        \includegraphics[width=.8\textwidth]{iwona}
        \only<presentation>{\pagebreak}
  \item Font Kurier
        \begin{lstlisting}
 \usepackage{kurier}
\end{lstlisting}
        \includegraphics[width=.8\textwidth]{kurier}
  \item Trebuchet (był to podstawowy font poprzedniego szablonu).
        Jeżeli ktoś ma go zainstalowanego (i korzystał z~poprzedniego szablonu bez problemów), można użyć go poleceniem:
        \begin{lstlisting}
\renewcommand{\sfdefault}{jtrr}
\end{lstlisting}
        \includegraphics[width=.8\textwidth]{trebuchet}
 \end{enumerate}
 Ale nie są to jedyne możliwości.
\end{frame}

Ja, osobiście bardzo lubię fonty \href{https://www.ctan.org/pkg/iwona}{Iwona} i~\href{https://www.ctan.org/pkg/kurier}{Kurier}, więc bardzo często zamiast fontu Carlito używam właśnie ich.

Doświadczenie uczy, że w~prezentacjach lepiej sprawdzają się fonty bezszeryfowe (\emph{sans serif}) można użyć praktycznie każdego fontu dostępnego w~systemie \hologo{LaTeX}. Na stronie \url{http://www.tug.dk/FontCatalogue/sansseriffonts.html} znajduje się katalog dostępnych fontów. Jeżeli zrezygnować z~\hologo{pdfLaTeX}a na rzecz \hologo{LuaLaTeX}a lub \hologo{XeLaTeX}a — można użyć praktycznie każdego współczesnego fontu dostępnego w~systemie. Warto jednak poczytać dokumentację pakietu \href{https://ctan.org/pkg/fontspec}{fontspec} \cite{Fontspec}.

\subsection{Efekty „przejścia”}

Efekt „przejścia” (\emph{transition effect}) slajdów są (według mnie) czymś złym.
Zajmują czas (nawet jak tylko dwie sekundy), podczas procesu znikania jednego slajdu i~pojawiania się drugiego zawartość jest niedostępna lub nieczytelna, odwracają wreszcie uwagę od zasadniczego tematu.
No, ale są.
Również w~beamerze można z~nich korzystać. Definiuje się je na poziomie warstw lub slajdów.

\begin{frame}
 \frametitle<presentation>{Efekty ,,przejścia''}
 \framesubtitle{Transition effects}
 \transduration<1->{1}
 \transblindshorizontal<2>[duration=1]
 \transblindsvertical<3>[duration=1]
 \transboxin<4>[duration=1]
 \transboxout<5>[duration=1]
 \transcover<6>[duration=1]
 \transdissolve<7>[duration=1]
 \transfade<8>[duration=1]
 \transglitter<9>[duration=1]
 \hypertarget{efekty}{Beamer} dostarcza następujący zestaw efektów:
 \begin{itemize}
  \item<2>
        \textbf{blindshorizontal}
        Show the slide as if horizontal blinds were pulled away.
  \item<3>
        \textbf{blindsvertical}
        Show the slide as if vertical blinds were pulled away.
  \item<4>
        \textbf{boxin}
        Show the slide by moving to the center from all four sides.
  \item<5>
        \textbf{boxout}
        Show the slide by showing more and more of a~rectangular area that is centered on the slide center.
  \item<6>
        \textbf{cover}
        Show the slide by covering the content that was shown before.
  \item<7>
        \textbf{dissolve}
        Show the slide by slowly dissolving what was shown before.
  \item<8>
        \textbf{fade}
        Show the slide by slowly fading what was shown before.
  \item<9>
        \textbf{glitter}
        Show the slide with a~glitter effect that sweeps in the specified direction.
 \end{itemize}
\end{frame}
\begin{frame}
 \frametitle<presentation>{Efekty ,,przejścia'' cd}
 \transduration<1->{1}
 \transpush<1>[duration=1]
 \transsplitverticalin<2>[duration=1]
 \transsplitverticalout<3>[duration=1]
 \transsplithorizontalin<4>[duration=1]
 \transsplithorizontalout<5>[duration=1]
 \transwipe<6>[duration=1]
 %\transreplace<3>[duration=1]
 \begin{itemize}
  \item<1>
        \textbf{push}
        Show the slide by pushing what was shown before off the screen using the new content.
  \item<2>
        \textbf{replace}
        Replace the previous slide directly (default behaviour).
  \item<3>
        \textbf{splitverticalin}
        Show the slide by sweeping two vertical lines from the sides inward.
  \item<4>
        \textbf{splitverticalout}
        Show the slide by sweeping two vertical lines from the center outward.
  \item<5>
        \textbf{splithorizontalin}
        Show the slide by sweeping two horizontal lines from the sides inward.
  \item<6>
        \textbf{splithorizontalout}
        Show the slide by sweeping two horizontal lines from the center outward.
  \item<7>
        \textbf{wipe}
        Show the slide by sweeping a~single line in the specified direction, thereby “wiping out” the previous contents.
 \end{itemize}
\end{frame}

% \begin{frame}
% \frametitle<beamer>{Efekty ,,przejścia''}
% \only<beamer>{Beamer dostarcza następujący zestaw efektów:}
% \begin{itemize}
%\item
% \textbf{splitverticalin} 
%Show the slide by sweeping two vertical lines from the sides inward.
%\item
% \textbf{splitverticalout} 
%Show the slide by sweeping two vertical lines from the center outward.
%\item
% \textbf{splithorizontalin} 
%Show the slide by sweeping two horizontal lines from the sides inward.
%\item
% \textbf{splithorizontalout} 
%Show the slide by sweeping two horizontal lines from the center outward.
%\item
% \textbf{wipe} 
% Show the slide by sweeping a~single line in the specified direction, thereby “wiping out” the previous contents.
%\end{itemize}
%\end{frame}

\begin{frame}[fragile]
 \frametitle{Efekty przejścia}
 \framesubtitle{Uwagi}
 \begin{itemize}
  \item
        Pamiętać natomiast trzeba, że nie każdy program używany do wyświetlania plików PDF wszystkie efekty realizuje.
        Można mieć pewność chyba tylko w~przypadku oryginalnego Adobe Acrobat Readera.
  \item
        Efekt przejścia został zaimplementowany w~taki sposób, że można go wskazać dla wybranych nakładek. Polecenia włączające efekt przejścia dokładnie opisane są w~dokumentacji pakietu beamer\only<article>{~\cite{Beamer}}.
        Tworzy się je przez dodanie do nazwy efektu przedrostka: \lstinline|\trans|. Całe polecenie ma postać:\\
        \lstinline|\trans|\emph{efekt}\lstinline|<|\emph{specyfikacja slajdów}\lstinline|>[|\emph{dodatkowe parametry}\lstinline|]|.
  \item
        \emph{dodatkowe parametry} to, na przykład czas trwanie efektu: \lstinline|duration=1|
 \end{itemize}
\end{frame}

\subsection{Uwagi}

Można \href{https://pwr.edu.pl/uczelnia/informacje-ogolne/materialy-promocyjne/logotyp/}{znaleźć} przygotowany (przez \href{https://pwr.edu.pl/uczelnia/informacje-ogolne/organizacja-uczelni/index,dzial-informacji-i-promocji.html}{Dział Informacji i~Promocji}) materiał zawierający \href{https://kmim.wm.pwr.edu.pl/myszka/wp-content/uploads/sites/2/2017/12/prezentacje_instrukcje.pdf}{przykładowe slajdy} i~możliwości rozmieszczenia tekstu oraz grafik  w~nowych szablonach.
Niektóre efekty bardzo łatwo jest osiągnąć w~beamerze, inne są trudniejsze do uzyskania. W~szczególności modyfikacje slajdu tytułowego wymagają modyfikacji szablonu.

\subsection{Modyfikacje slajdu tytułowego}

\begin{frame}[fragile]
 \frametitle<presentation>{Slajd tytułowy}
 \hypertarget{tytul}{Modyfikacje} slajdu tytułowego są bardzo ograniczone. Standardowo mogą się tam pojawić następujące pola:
 \begin{columns}
  \begin{column}{.5\textwidth}
   \begin{itemize}
    \item
          autor (\lstinline|\author{}|),
    \item
          tytuł (\lstinline|\title{}|),
    \item
          podtytuł (\lstinline|\subtitle{}|),
    \item
          data (\lstinline|\date{}|),
    \item
          instytucja (\lstinline|\institute{}|),
    \item
          grafika (\lstinline|\titlegraphics{}|),
   \end{itemize}
   \only<article>{Slajd tytułowy przedstawia rys.~\ref{title}.}
  \end{column}
  \begin{column}{.48\textwidth}
   \only<article>{\begin{figure}}
     \includegraphics[width=.6\textwidth]{metadane}

     \only<article>{
     \caption{Wygląd slajdu tytułowego}\label{title}
    \end{figure}
   }
  \end{column}
 \end{columns}
\end{frame}



\subsection{Metadane}

\begin{frame}[fragile]
 \frametitle<presentation>{Metadane}
 Zawartość pól:
 \begin{itemize}
  \item
        \lstinline|\title{}| oraz \lstinline|\subtitle{}|,
  \item
        \lstinline|\author{}|,
  \item
        \lstinline|\keywords{}|,
  \item
        \lstinline|\subject{}|,
 \end{itemize}
 automatycznie trafia do odpowiednich pól pliku PDF zawierających metadane. Generalnie metadanie mogą ułatwić zadanie wyszukiwarkom czy poprawnie zdefiniowć autora tekstów umieszczonych w~Internecie.

 Z~umieszczania metadanych w~pliku PDF można zrezygnować:
 \begin{lstlisting}
\documentclass[usepdftitle=false]{beamer}
\end{lstlisting}
\end{frame}

Tak na marginesie: pakiet \hyperref{https://ctan.org/pkg/pdfprivacy}{pdfprivacy} pozwala ma usunięcie wielu innych metadanych zapisywanych standardowo do pliku PDF.

\section{Dokumentacja}

Całą dokumentacja znajduje się w~podkatalogu (patrz rozdział~\ref{instalacja}) \path{doc/pwr}. Składa się na nią:
\begin{frame}
 \frametitle<presentation>{Dokumentacja}
 \only<presentation>{Na dokumentację składa się:}
 \begin{itemize}
  \item
        plik \href{https://kmim.wm.pwr.edu.pl/myszka/wp-content/uploads/sites/2/2017/12/nowy_szablon_doc_article.pdf}{{nowy\_szablon\_doc\_article.pdf}}\only<article>{ (zawierający treść tej strony)};
  \item
        plik \href{https://kmim.wm.pwr.edu.pl/myszka/wp-content/uploads/sites/2/2017/12/nowy_szablon_doc_beamer.pdf}{nowy\_szablon\_doc\_beamer.pdf} będący wersją „prezentacyjną” tej dokumentacji\only<presentation>{ (czyli właśnie to)};
  \item
        pliki źródłowe dokumentacji;
  \item
        przykładowa, prosta prezentacja \href{https://kmim.wm.pwr.edu.pl/myszka/wp-content/uploads/sites/2/2017/12/NewPwr-example.pdf}{NewPwr-example.pdf} wraz ze źródłami;
        \only<presentation>{\item dodatkowym bonusem jest plik \lstinline|oficyna_url.bst| czyli szablon  „\textbf{prawie} zgodny” z~wymaganiami Oficyny Wydawniczej PWr.}
 \end{itemize}
 \only<presentation>{Dokumentacja znajduje się w~podkatalogu \path{doc/pwr}.}
\end{frame}


\section{Materiały informacyjne (handout)}

Aby przygotować materiały informacyjne (tzw. handout) wystarczy wybrać ich format (dwie albo cztery slajdy na kartkę) i~zmodyfikować nieco źródło.
\begin{frame}[fragile]
 \frametitle<presentation>{Materiały informacyjne}
 \framesubtitle<presentation>{(handout)}
 \hypertarget{sec:handout}{}
 \begin{lstlisting}
\documentclass[handout]{beamer}
\usepackage{pgfpages}
\pgfpagesuselayout{2 on 1}[a4paper,
                           border shrink=5mm]
\end{lstlisting}
 może też być
 \begin{lstlisting}
\pgfpagesuselayout{4 on 1}[a4paper,landscape,
                           border shrink=5mm]
\end{lstlisting}
 Standardowo materiały informacyjne zostaną wygenerowane używając „pełnej wersji” slajdów (po uwzględnieniu wszystkich informacji „przyrostowych”).
\end{frame}
Można eksperymentować z~innymi opcjami oszczędnościowymi i~umieszczać więcej slajdów na jednej stronie.

Generalna zasada podczas tworzenia materiałów informacyjnych jest taka, że zostaje umieszczona w~nich „ostateczne” wersja slajdu (czyli po uwzględnieniu wszystkich przyrostów). Można jednak zdecydować inaczej — używając parametru handout w~definicji przyrostu.

\section{Folie}

Pakiet beamer może również być użyty do tworzenia folii (transparencies). Nie bardzo wiem w~jakiej sytuacji może się to dziś przydać — ale istnieje taka możliwość.

\begin{frame}[fragile]
 \frametitle<presentation>{Folie}
 \framesubtitle{trans}
 \hypertarget{sec:trans}{}
 \begin{lstlisting}
\documentclass[trans,17pt,aspectratio=43]{beamer}
\end{lstlisting}
 \begin{itemize}
  \item
        Standardowo folie zostaną wygenerowane używając „pełnej wersji” slajdów (po uwzględnieniu wszystkich informacji „przyrostowych”).
  \item
        Mądrze korzystając ze wszystkich możliwości oferowanych przez „kompilację warunkową”  można w~ten sposób wygenerować „lżejszą” (mniej atramentożerną) wersję prezentacji.
 \end{itemize}

\end{frame}

Działanie opcji \lstinline|trans| jest zbliżone do działania opcji \lstinline|hangout|: użyte będą „ostateczne” wersje slajdów (czyli po uwzględnieniu wszystkich przyrostów).

%\section{Drugi ekran}
%
%Second screen

\section{Tłumaczenia}

Pakiet beamer już od pewnego czasu używa narzędzia bardzo podobnego do narzędzi używanych w „profesjonalnych” pakietach oprogramowania, a służącego do łatwiejszego tworzenia dokumentów wielojęzycznych.

Pakiet nazywa się \href{https://ctan.org/pkg/translator}{translator}. Osobiście nie wiem, czy to dobry pomysł, czy zły, ale tak jest.

W szczególności korzystanie ze środowisk: \lstinline|example|, \lstinline|theorem|, \lstinline|definition|, czy poleceń \lstinline|\partpage|, \lstinline|\sectionpage|, \lstinline|\subsectionpage|,…

%
%\section{Narzędzia pomocnicze}
%
%\subsection{pdfpc}
%
%\url{https://pdfpc.github.io/}
%
%\subsection{pympress}
%
%\url{https://github.com/Cimbali/pympress}
%
%\subsection{PDF Presenter}
%
%\url{http://pdfpresenter.sourceforge.net/}
%




\section{Do zrobienia}

\begin{frame}[fragile,allowframebreaks]
 \frametitle<presentation>{Do zrobienia}
 \begin{enumerate}
  \item
        Szczegółowe:
        \begin{itemize}
         \item[\done]
               Numery slajdów
         \item[\done]
               Numer strony na stronie tytułowej
         \item[\emptybox]
               Poprawne pozycjonowanie ilustracji wstawianej poleceniem \lstinline|\titlegraphics| dla wersji pionowej
         \item[\emptybox]
               Personalizacja strony tytułowej
         \item[\emptybox]
               Wygląd bloków (zwłaszcza alertblock) i~twierdzeń, dowodów,… % powiązać z~kolorkiem szarym z~szablonu (dla poziomego?)
         \item[\done]
               Slajdy plain (bez tła)
         \item[\emptybox]
               \emph{Progress bar?}
               \item [\emptybox] Zoom
               \item [\emptybox] Kompresja slajdów
        \end{itemize}
  \item Ogólne:
        \begin{itemize}
         \item[\emptybox]
               Notatki
         \item[\emptybox]
               Drugi ekran
         \item[\emptybox]
               Narzędzia (Linux)
               \item[\emptybox] 
               Animacje — ogólne uwagi i przykłady
        \end{itemize}
        \mode<presentation>{\pagebreak}
  \item Zgłoszone na kursie uwagi/błędy:
        \begin{itemize}
         \item[\emptybox] Kolorystyka spisu treści (Agenda)
         \item[\done] Numeracja sekcji w spisie treści
          \item[\done] Raczej numer slajdu niż numer strony PDF
        \end{itemize}
  \item Zauważone błędy (niedoróbki):
        \begin{itemize}
         \item[\done]
               Numeracja slajdów (slajd 11???)
        \end{itemize}
 \end{enumerate}
\end{frame}

%\subsection{Komentarze}
%
%W maju byłem uczestnikiem organizowanego na Poltechnice Wrocławskiej w ramach projektu Innowacyjna Uczelnia --- Innowacyjny Nauczyciel szkolenia \LaTeX{} tworzenie profesjonalnej prezentacji. Zgłoszono mi tam pewne propozycje ulepszenia szablonu.

\section{Uwagi}

\subsection{Struktura archiwum/plików w~pakiecie}

{
 \footnotesize
 \lstinputlisting{tree.txt}
}

\subsection{Pliki dodatkowe}

\subsubsection{Szablon bibliografii zgodny z~wymaganiami Oficyny Wydawniczej PWr}

W pakiecie znajduje się również styl bibliografii \path{oficyna\_url.bst}. Powstał om podczas składania różnych publikacji wydawanych przez Oficynę Wydawniczą PWr i~\textbf{znacznym stopniu} spełnia wymagania edytorskie. (Choć te, w~zależności od Redaktora potrafią się trochę różnić, czy może inaczej: różni się odporność Redaktorów na drobne niezgodności.)

Plik można pobrać również \href{http://kmim.wm.pwr.edu.pl/myszka/wp-content/uploads/sites/2/2018/11/oficyna_url.bst}{osobno}.

\subsubsection{Testy}

W pod-kartotece z~dokumentacją znajduje się pod-kartoteka beamer-benchmark. W~niej znajdują się pliki zaczerpnięte z~repozytorium \href{https://github.com/louisstuart96/beamer-benchmark}{beamer-benchmark} \href{https://github.com/louisstuart96}{Luisa Stuarta}. Pozwalają one „testować” różne elementy szablonu.

\subsection{Stara wersja szablonu…}

…jest ciągle \href{https://kmim.wm.pwr.edu.pl/myszka/projekty/szablon-prezentacji-pwr/szablon-prezentacji-zgodny-z-ksiega-logotypu/}{dostępna}.

%\subsection{}

%\end{document}



