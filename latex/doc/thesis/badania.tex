\chapter{Badania eksperymentalne - wyniki i wnioski}
\label{ch:badania_eksperymentalne}

W niniejszym rozdziale prezentujemy wyniki oraz wnioski wynikające z przeprowadzonych badań eksperymentalnych. Zestawienie rezultatów eksperymentów zostało dokonane w sposób zrozumiały i klarowny, z odpowiednim wsparciem graficznym w postaci wykresów i tabel. Wnioski są bezpośrednio powiązane z wynikami badań oraz są zgodne z założonymi celami projektu.

\section{Metodologia badań}
\label{sec:metodologia_badan}

W tej sekcji szczegółowo omawiamy metody oraz podejście zastosowane podczas eksperymentów. Zostaną przedstawione narzędzia, parametry konfiguracyjne oraz procedura testowa, które razem tworzą ramy metodyczne naszego badania.

\subsection{Konfiguracja środowiska}
\label{subsec:konfiguracja_srodowiska}

Do eksperymentów wykorzystano następujące ustawienia konfiguracyjne środowiska:
\begin{verbatim}
    > ./main.py -m 'gpt-4-1106-preview' Vulnerable-Code-Snippets/
\end{verbatim}

\subsection{Procedura testowa}
\label{subsec:procedura_testowa}

Zaprojektowana procedura testowa miała na celu dokładną weryfikację funkcjonalności programu oraz ocenę jego skuteczności w wykrywaniu i naprawie podatności. Kryteria testowe zostały dobrane w sposób umożliwiający kompleksową analizę:
\begin{itemize}
    \item \textbf{Kryterium 1}: Dokładność identyfikacji podatności.
    \item \textbf{Kryterium 2}: Skuteczność proponowanych napraw.
    \item \textbf{Kryterium 3}: Efektywność czasowa analizy.
\end{itemize}

Procedura testowa przebiegała według następujących etapów:
\begin{enumerate}
    \item Selekcja i przygotowanie danych testowych.
    \item Analiza statyczna kodu z wykorzystaniem narzędzi AI.
    \item Przeprowadzenie testów funkcjonalnych oraz testów bezpieczeństwa.
    \item Analiza i interpretacja wyników.
\end{enumerate}

\subsection{Wyniki działania programu - przykład}
\label{subsec:wyniki_dzialania_programu}

Wyniki działania programu, prezentowane na konsoli oraz dokumentowane w plikach raportów, zapewniają bezpośredni wgląd w proces analizy kodu:
% Tutaj należy wstawić odpowiednią zawartość raportu w formacie Markdown, o ile jest dostępna

\markdownInput{../raports/Authentication_Bypass_20240120120144_raport.md} 




\section{Badania na zbiorze \textit{snoopysecurity/Vulnerable-Code-Snippets}}
\label{sec:badania_na_zbiorze_snoopysecurity}

Analiza zbioru \textit{snoopysecurity/Vulnerable-Code-Snippets} dostarczyła istotnych informacji na temat specyfiki podatności i skuteczności ich wykrywania przez system. Zbiór ten, zawierający 184 pliki źródłowe o łącznej liczbie 41831 tokenów, stanowił reprezentatywną próbkę dla naszych eksperymentów.

Eksperymenty przeprowadzono z wykorzystaniem poniższych parametrów:
\begin{verbatim}
    > ./main.py -m 'gpt-4-1106-preview' Vulnerable-Code-Snippets/
\end{verbatim}


\begin{verbatim}
2024-01-20 12:54:36: Welcome to gptester: the Static Code Analysis Agent
2024-01-20 12:54:36: I will now begin scanning: Vulnerable-Code-Snippets/, name: Vulnerable-Code-Snippets
2024-01-20 12:54:36: Beginning scan...
2024-01-20 12:54:36: Found 184 files to scan
2024-01-20 12:54:36: Tokens inside the directory: 41831
2024-01-20 12:54:36: Beginning code analysis...
2024-01-20 12:54:36: Using model: gpt-4-1106-preview
\end{verbatim}



\subsection{Interpretacja wyników}
\label{subsec:interpretacja_wynikow}

Interpretacja wyników eksperymentów ujawniła istotne spostrzeżenia dotyczące możliwości wykorzystania modeli AI w celu poprawy bezpieczeństwa kodu. Analiza wykazała, że ...

% Tutaj należy dodać interpretację wyników w kontekście wykresów i tabel (jeśli takowe zostały przygotowane)

\section{Wnioski}
\label{sec:wnioski}

Na podstawie przeprowadzonych badań eksperymentalnych udało się zweryfikować założenia dotyczące efektywności wykorzystania modeli AI w procesie identyfikacji i naprawy podatności w kodzie źródłowym. Główne wnioski to:

% Tutaj należy wstawić wnioski z badania