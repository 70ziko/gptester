\chapter{Zbiory danych i ich przygotowanie}
\label{ch:zbiory_danych}

W kontekście niniejszego rozdziału dokonano prezentacji oraz analizy zbiorów danych, które zostały wykorzystane w procesie testowania programu \textbf{gptester} oraz badania skuteczności dużych modeli językowych (LLM). Szczegółowo opisany został proces przygotowania i przetwarzania tych zbiorów danych, co ma kluczowe znaczenie dla efektywności analizy statycznej kodu i kalibracji narzędzia.


\section{Przegląd wykorzystanych zbiorów danych}
\label{sec:przeglad_zbiorow}

Następująca sekcja zawiera omówienie źródeł danych, ich specyfikacji oraz roli, jaką odgrywają w kontekście projektu. Analiza ta obejmuje zarówno otwarte zbiory danych, repozytoria kodu, jak i bazy danych podatności.


\begin{itemize}
    \item \textbf{snoopysecurity/Vulnerable-Code-Snippets}: Repozytorium w serwisie Github zawierające zbiór fragmentów kodu zawierających luki bezpieczeństwa. Fragmenty pobrane z różnych wpisów na blogach, książek, zasobów itp. 
    Zbiór w głównej mierze używany do testowania implementacji. Niektóre fragmenty kodu zawierają wskazówki w nazwach/komentarzach. Ewentualne naruszenie praw autorskich niezamierzone.\\ \url{https://github.com/snoopysecurity/Vulnerable-Code-Snippets}

    % \item \textbf{DiverseVul}: \url{https://arxiv.org/abs/2304.00409} Opis kolejnego zbioru danych, jego charakterystyka i zastosowanie w kontekście `gptester`.
    % \item \textbf{CVEfixes}: \url{https://github.com/secureIT-project/CVEfixes}
    \item \textbf{OWASP VulnerableApp}: Aplikacja webowa zawierająca wiele podatności, używana do testowania narzędzi do testowania bezpieczeństwa aplikacji webowych. \\ \url{https://github.com/SasanLabs/VulnerableApp}
    % Dodaj więcej zbiorów danych w razie potrzeby
\end{itemize}

\section{Proces przygotowania danych}
\label{sec:proces_przygotowania_danych}

Dobrane przeze mnie zbiory danych zostały tak, by nie trzeba było dostosowywać programu do konkretnego formatu. Oznacza to, że wskazane repozytoria zawierają przykłady kodu zapisane w plikach.
\subsection{snoopysecurity/Vulnerable-Code-Snippets} 
Repozytorium zawiera wiele plików z przykładami kodu, które mogą zawierać błędy bezpieczeństwa. Pliki te zostały pobrane z różnych źródeł, takich jak blogi, książki, zasoby itp. Pliki te zawierają często komentarze lub nazwy zmiennych, które wskazują na potencjalne błędy bezpieczeństwa. Pozwala to nam na izolację problemu identyfikacji podatności od generowania kodu. W pierwszej kolejności badania zostały przeprowadzone bez wprowadzania zmian w kodzie, aby ocenić skuteczność modeli językowych w korekcji błędów bezpieczeństwa. W kolejnym kroku, w celu przebadania zdolności do identyfikowania błędów, zostały wprowadzone zmiany w kodzie, takie jak usunięcie komentarzy, zmienienie nazw zmiennych, itp. W ten sposób można było sprawdzić, czy modele językowe są w stanie wykryć błędy bezpieczeństwa, gdy mają więcej informacji na temat kodu.

\subsection{OWASP VulnerableApp}
Aplikacja webowa zawierająca wiele podatności, używana w testach narzędzi do testowania bezpieczeństwa aplikacji webowych. Zawiera wiele przykładów kodu, które mogą zawierać błędy bezpieczeństwa. Dzięki użyciu w badaniu przykładu z prawdziwej aplikacji, można było sprawdzić, czy modele językowe są w stanie wykryć błędy bezpieczeństwa w prawdziwym kodzie.

\section{Wyzwania i ograniczenia}
\label{sec:wyzwania_i_ograniczenia}

Głównym wyzwaniem prezentowanym przez użyte przeze mnie próbki badawcze wynikają z ich charakteru. Zbiór danych Vulnerable-Code-Snippets nie jest reprezentatywne dla rzeczywistych aplikacji, a jedynie zawiera przykłady kodu, które mogą zawierać błędy bezpieczeństwa. W przypadku większości przykładów kodu, nie jest możliwe uruchomienie go bez posiadania kodu całego projektu, co utrudnia ewaluację. W związku z powyższym niektóre skrawki kodu zostały obudowane w aplikacje webową, natomiast inne pominięte. Nie każdy skrawek kodu w repozytorium jest wycięty z aplikacji webowej, te przykłady zostały uwzględnione w badaniach i sprawiały najmniej problemów. 

W przypadku OWASP VulnerableApp, aplikacja jest w pełni uruchamialna i możliwe jest badanie funkcjonalności programu na przykładzie rzeczywistym. Trudnością jest natomiast wczesna wersja 'gptester', w której nie została jeszcze wprowadzona funkcjonalność aktualizowania bazy kodu za pomocą funkcji git. Powoduje to konieczność ręcznego scalania zmian w kodzie z nowymi wersjami aplikacji, co jest czasochłonne i utrudnia badania. Testy bezpieczeństwa zostaną przeprowadzone za pomocą skanerów podatności, przede wszystkim OWASP ZAP, który jest zoptymalizowany do wykrywania podatności w aplikacji OWASP VulnerableApp, co pomoże w faktycznej ocenie skuteczności wprowadzonych korekt.

\subsection{Ograniczenia zbiorów danych}
Ograniczająca jest również liczba przykładów kodu w zbiorach danych. Wpływa to 

W przypadku repozytorium Vulnerable-Code-Snippets, niektóre przykłady kodu są bardzo podobne do siebie, co może wpływać na wyniki badań. W przypadku OWASP VulnerableApp, aplikacja zawiera wiele podatności, ale nie wszystkie są wykorzystywane w badaniach. W związku z powyższym, wyniki badań mogą być niepełne i nieodpowiednie do wyciągnięcia wniosków.

\section{Podsumowanie}
Podsumowujemy, jak przygotowanie i analiza zbiorów danych wpłynęła na projekt `gptester` i jakie wnioski można z tego wyciągnąć.

