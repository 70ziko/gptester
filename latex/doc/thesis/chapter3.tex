\chapter{Zbiory danych i ich przygotowanie}
\label{ch:zbiory_danych}

W tym rozdziale skupimy się na omówieniu zbiorów danych wykorzystanych w projekcie `gptester`, a także na procesie ich przygotowania i przetwarzania. Zbiory danych odgrywają kluczową rolę w procesie analizy statycznej kodu, pozwalając na dokładne testowanie i kalibrację narzędzia.

\section{Przegląd wykorzystanych zbiorów danych}
\label{sec:przeglad_zbiorow}

Opisujemy tutaj źródła danych, ich charakterystykę oraz znaczenie dla projektu. Może to obejmować otwarte zbiory danych, repozytoria kodu, bazy danych podatności itp.

\begin{itemize}
    \item \textbf{snoopysecurity/Vulnerable-Code-Snippets}: Repozytorium w serwisie Github zawierające zbiór fragmentów kodu zawierających luki w Internecie. Fragmenty pobrane z różnych wpisów na blogach, książek, zasobów itp. 
    Zbiór w głównej mierze używany do testowania implementacji. Niektóre fragmenty kodu zawierają wskazówki w nazwach/komentarzach. Ewentualne naruszenie praw autorskich niezamierzone.\\ \url{https://github.com/snoopysecurity/Vulnerable-Code-Snippets}

    \item \textbf{DiverseVul}: \url{https://arxiv.org/abs/2304.00409} Opis kolejnego zbioru danych, jego charakterystyka i zastosowanie w kontekście `gptester`.
    \item \textbf{CVEfixes}: \url{https://github.com/secureIT-project/CVEfixes}
    % Dodaj więcej zbiorów danych w razie potrzeby
\end{itemize}

\section{Proces przygotowania danych}
\label{sec:proces_przygotowania_danych}

Tutaj opisujemy, jak zbiory danych zostały przygotowane do użycia w projekcie. To obejmuje procesy takie jak czyszczenie danych, transformacja, ewentualne wzbogacanie oraz metody ich selekcji.

\subsection{Czyszczenie i normalizacja danych}
Opisujemy, jakie kroki zostały podjęte, aby dane były spójne i wolne od błędów, które mogłyby wpłynąć na wyniki analizy.

\subsection{Transformacja danych}
Omawiamy wszelkie przekształcenia danych, które były konieczne, takie jak konwersja formatów, zmiana struktury danych itp.

\subsection{Metody selekcji i wzbogacania danych}
Opisujemy, jak wybrano dane do analizy i czy były one w jakiś sposób wzbogacane (np. poprzez dodanie etykiet, użycie dodatkowych źródeł danych).

\section{Wyzwania i ograniczenia}
\label{sec:wyzwania_i_ograniczenia}

W tej sekcji omawiamy wyzwania, które napotkano podczas pracy z danymi, oraz ograniczenia zbiorów danych, które mogą wpłynąć na wyniki projektu.

\subsection{Problemy z jakością danych}
Omawiamy wszelkie napotkane problemy z jakością danych i jak wpłynęły one na proces analizy.

\subsection{Ograniczenia zbiorów danych}
Opisujemy ograniczenia zbiorów danych, takie jak ich rozmiar, zakres, reprezentatywność itp., i jak mogą one wpłynąć na ogólne wnioski projektu.

\section{Podsumowanie}
Podsumowujemy, jak przygotowanie i analiza zbiorów danych wpłynęła na projekt `gptester` i jakie wnioski można z tego wyciągnąć.

\end{chapter}
