
\chapter*{Wprowadzenie}

Niniejsza praca inżynierska nosi tytuł "Zastosowanie dużych modeli językowych do wykrywania i naprawiania błędów bezpieczeństwa i podatności w kodzie aplikacji webowych". Celem tej pracy jest zbadanie, jak skutecznie zaawansowane modele językowe, takie jak GPT-3.5, GPT-4 oraz w w przyszłych iteracjach badania modele otwarto-źródłowe między innymi Mistral 7B i Falcon-7B-instruct, mogą być wykorzystane do automatycznego wykrywania i naprawy błędów bezpieczeństwa w kodzie zarówno aplikacji webowych jak i lokalnych. 

W tym celu zostanie opracowane i zaimplementowane narzędzie do statycznej analizy kodu, które będzie wykorzystywać modele językowe do detekcji podatności i naprawy błędów.
Narzędzie to zostanie przetestowane i porównane z innymi rozwiązaniami, takimi jak Snyk, które oferują podobne funkcjonalności, a także z tradycyjnymi skanerami podatności.

W pracy zostaną przedstawione wyniki badań, które mają na celu odpowiedzieć na pytanie, czy modele językowe mogą być wykorzystane do tego celu, oraz jak skuteczne są one w porównaniu z innymi rozwiązaniami.
W ramach pracy zostaną również zbadane ograniczenia i wyzwania związane z wykorzystaniem tych technologii w kontekście cyberbezpieczeństwa.

\section*{Pytania badawcze}
W ramach pracy stawiam następujące pytania badawcze:
\begin{enumerate}
    \item Czy duże modele językowe mogą być wykorzystane do wykrywania i naprawiania błędów bezpieczeństwa w kodzie aplikacji webowych?
    \item Jak skuteczne są te modele w porównaniu z innymi rozwiązaniami?
    \item W jakim stopniu metody wzbogacania generacji (RAG) i uczenia się w kontekście (in-context learning) mogą poprawić skuteczność tych modeli?
    \item Jakie są ograniczenia i wyzwania związane z wykorzystaniem tych technologii w kontekście cyberbezpieczeństwa?
\end{enumerate}
\newpage
\section*{Hipotezy}
Hipotezy pracy to:
\begin{enumerate}
    \item Duże modele językowe, dzięki swojej zdolności do analizy i generowania kodu, mogą skutecznie identyfikować i naprawiać błędy bezpieczeństwa w kodzie źródłowym.
    \item Mimo obiecującego potencjału, modele te mogą napotykać ograniczenia, szczególnie w bardziej złożonych i specyficznych scenariuszach związanych z cyberbezpieczeństwem.
\end{enumerate}

\section*{Uzasadnienie tytułu}
Tytuł pracy został dobrany tak, aby odzwierciedlał główny obszar zainteresowania badawczego, jakim jest wykorzystanie nowoczesnych technologii językowych w celu poprawy bezpieczeństwa aplikacji webowych. 
W kontekście rosnącej zależności od cyfrowych rozwiązań, temat ten zyskuje na znaczeniu, oferując nowe perspektywy i podejścia do zagadnień bezpieczeństwa.
Tytuł można skrócić do \textbf{''Zastosowanie dużych modeli językowych w statycznej analizie kodu''}, ponieważ tak nazywa się problem odnajdywania i korekcji błędów w kodzie źródłowym. 
Korpus badawczy pracy został rozszerzony względem tytułu o projekty open-source aplikacji natywnych i desktopowych oraz wycinki błędnego kodu i poprawnego kodu. 

\section*{Omówienie literatury naukowej i stopnia jej przydatności}
Podstawę teoretyczną pracy stanowi literatura naukowa skupiająca się na dużych modelach językowych oraz ich zastosowaniu w cyberbezpieczeństwie. Szczególną uwagę poświęcono artykułowi ''Can OpenAI Codex and Other Large Language Models Help Us Fix Security Bugs?'', który posłużył jako punkt wyjścia dla badań. 

Praca ta ma na celu kontynuację i poszerzenie zakresu tych badań, wykorzystując literaturę naukową jako fundament do eksploracji nowych możliwości w zakresie analizy i naprawy błędów w kodzie.
Różnica między tą pracą, a literaturą naukową polega na tym, że praca skupia się na praktycznym zastosowaniu modeli językowych w statycznej analizie kodu, podczas gdy literatura naukowa skupia się na badaniu możliwości Sztucznej Inteligencji w tym zakresie.

\section*{Cel pracy}
Głównym celem pracy jest zbadanie skuteczności wykorzystania dużych modeli językowych do wykrywania i naprawiania błędów bezpieczeństwa i podatności w kodzie źródłowym aplikacji webowych. 
W tym kontekście można wyróżnić następujące cele pośrednie:
\begin{itemize}
    \item Opracowanie praktycznego rozwiązania do statycznej analizy kodu dla aplikacji webowych oraz lokalnych.
    \item Badanie skuteczności dużych modeli językowych w wykrywaniu podatności i luk bezpieczeństwa.
\end{itemize}
\section*{Zakres pracy}
Zakres pracy obejmuje:
\begin{itemize}
    \item Analizę istniejącej literatury i badań, w szczególności artykułu 'Can OpenAI Codex and Other Large Language Models Help Us Fix Security Bugs?'.
    \item Projekt i implementację narzędzia do statycznej analizy kodu opartego na modelach OpenAI.
    \item Przygotowanie zbiorów danych i przykładów z kodem zawierającym potencjalne podatności.
    \item Testowanie i porównanie skuteczności z innymi rozwiązaniami, np. oferowanymi przez firmę Snyk.
    \item Analiza wyników i formułowanie wniosków.
\end{itemize}