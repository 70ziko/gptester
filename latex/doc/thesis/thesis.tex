% !TEX encoding = UTF-8 Unicode 
%
% Use:
% magister / inzynier - for master thesis or engineering thesis
% druk / archiwum - for print version or archive version
% en - to translate template into english
% examples:
%\documentclass[inzynier,druk,en] - master thesis, print version, english
%\documentclass[magister,druk,en]{dyplom}
%\documentclass[magister,druk]{dyplom}

\documentclass[inzynier,druk]{dyplom}

\usepackage{multicol}
\usepackage{minted}
\usepackage{pdflscape}
\usepackage{graphicx}
\usepackage{float}
\usepackage{subcaption}
\usepackage{amsmath}
\usepackage{amsfonts}

\usepackage[utf8]{inputenc}
\usepackage{hyperref}

% Maximum section's depth.
\setcounter{secnumdepth}{4}

% Listings settings
\setminted{breaklines, 
frame=lines,           
framesep=3mm,          
baselinestretch=1.1,   
fontsize=\small,       
linenos              % line numbering
}

\usepackage{listings}
\usepackage{xcolor}
\usepackage{markdown}

% Define Markdown style
% \lstdefinestyle{mymarkdown}{
%     fontsize=\tiny,
%     backgroundcolor=\color{pink}, % choose the background color
%     frame=single, % adds a frame around the code
%     framesep=4pt, % tweak the padding
%     breaklines=true, % sets automatic line breaking
%     breakatwhitespace=true, % sets if automatic breaks should only happen at whitespace
%     postbreak=\mbox{\textcolor{red}{$\hookrightarrow$}\space}, % for breaking lines
% }

\usepackage{lipsum}

\faculty{Wydział informatyki i Telekomunikacji}                   % Uncomment if applicable
\fieldofstudy{Cyberbezpieczeńśtwo (CBE)}                          
\author{Patryk Fidler}
\title{Zastosowanie dużych modeli językowych do wykrywania i naprawiania błędów bezpieczeństwa i podatności w kodzie aplikacji webowych}
\supervisor{Dr. hab. inż. Maciej Piasecki}
% \consultant{Consultant's name}               % Uncomment if applicable
\specialisation{Bezpieczeństwo danych (CBD)}                         % Uncomment if applicable
\keywords{modele językowe, Sztuczna Inteligencja, statyczna analiza kodu}	% 3-5 keywords  

\begin{document}
\maketitle

\abstract{
% Polish abstract 
Praca inżynierska zatytułowana "Zastosowanie dużych modeli językowych do wykrywania i naprawiania błędów bezpieczeństwa i podatności w kodzie aplikacji webowych" koncentruje się na zastosowaniu zaawansowanych modeli językowych, takich jak GPT-3.5, GPT-4, a w przyszłości także modeli otwartoźródłowych, takich jak Mistral, do automatycznego wykrywania i naprawiania błędów bezpieczeństwa w kodzie oprogramowania i aplikacji webowych.

Motywacja tej pracy wynika z rosnącej roli dużych modeli językowych (LLM) w różnych dziedzinach, w tym w cyberbezpieczeństwie. W kontekście tych działań, badana jest możliwość wykorzystania tych modeli do wykrywania i naprawiania podatności takich jak XSS\footnote{Cross-Site Scripting}, SQL Injection\footnote{Iniekcja SQL}, CSRF\footnote{Cross-Site Request Forgery}, Buffer Overflow\footnote{Przepełnienie bufora} i tym podobne.

Punktem wyjścia dla pracy dyplomowej jest artykuł napisany w 2021 roku 
"Can OpenAI Codex and Other Large Language Models Help Us Fix Security Bugs?" \cite{codex-fix-security-bugs}
- Hammond Pearce, Benjamin Tan, Baleegh Ahmad, Ramesh Karri, Brendan Dolan-Gavitt \url{https://arxiv.org/pdf/2112.02125v1.pdf}. Autorzy podkreślają znaczący potencjał tych modeli, a niniejsza praca ma na celu kontynuację tych badań i poszerzenie ich zakresu.

Planowane działania obejmują zaprojektowanie oraz implementację praktycznego narzędzia dla programistów do statycznej analizy kodu, przygotowanie zbioru danych zawierającego podatne kod źródłowy, testowanie zdolności detekcji błędów przez modele językowe OpenAI, oraz porównanie tych wyników z istniejącymi rozwiązaniami.

Szczególny nacisk zostanie położony na wykorzystanie technik uczenia się w kontekście oraz generowanie wspomagane odnajdywaniem danych (RAG), które mogą pomóc w udoskonaleniu detekcji i wyników, nawet przy ograniczonych zasobach. 
Praca przewiduje implementację autonomicznego agenta AI zdolnego do analizy kodu, wykonania testów bezpieczeństwa i podejmowania decyzji na podstawie wyników tych testów i kontekstu.

W przyszłości badane będą możliwości detekcji błędów przez otwarte modele językowe, a w dalszej perspektywie, możliwości specjalizacji modeli w zakresie cyberbezpieczeństwa za pomocą dostrajania \footnote{fine-tuning}. Wszystkie te działania mają na celu nie tylko badanie, ale także poprawienie możliwości LLM w kontekście cyberbezpieczeństwa.

Głównym celem pracy jest zaproponowanie praktycznych rozwiązań, które mogą pomóc programistom w tworzeniu bardziej bezpiecznych aplikacji.

}{
% Abstract translated into English
The engineering thesis titled "Application of Large Language Models for Detecting and Repairing Security Errors and Vulnerabilities in Web Application Code" focuses on the use of advanced language models such as GPT-3.5, GPT-4, and in the future, open-source models like Mistral, for automatically detecting and repairing security errors in software code and web applications.

The motivation for this work stems from the growing role of large language models (LLMs) in various fields, including cybersecurity. In this context, the possibility of using these models to detect and repair vulnerabilities such as Cross-Site Scripting (XSS), SQL Injection, Cross-Site Request Forgery (CSRF), Buffer Overflow, and similar issues is explored.

The starting point for the thesis is the 2021 article "Can OpenAI Codex and Other Large Language Models Help Us Fix Security Bugs?" by Hammond Pearce, Benjamin Tan, Baleegh Ahmad, Ramesh Karri, Brendan Dolan-Gavitt. The authors highlight the significant potential of these models, and this work aims to continue their research and expand its scope.

Planned activities include the design and implementation of a practical tool for developers for static code analysis, preparation of a dataset containing vulnerable source code, testing the error detection capabilities of OpenAI language models, and comparing these results with existing solutions.

Special emphasis will be placed on the use of context-aware learning techniques and Retrieval-Augmented Generation (RAG), which can help improve detection and results, even with limited resources. The work anticipates the implementation of an autonomous AI agent capable of analyzing code, performing security tests, and proposing code fixes on the results of these tests and context.

Future research will explore the error detection capabilities of open language models, and in the longer term, the potential for specializing models in cybersecurity through fine-tuning. All these activities aim not only to study but also to improve the capabilities of LLMs in the context of cybersecurity.

The main goal of the thesis is to propose practical solutions that can help developers create more secure applications.


}

\tableofcontents


\chapter*{Wprowadzenie}

Ostatnie lata przyniosły niebywały rozwój w dziedzinie Sztucznej Inteligencji, a w szczególności w obszarze dużych modeli językowych (LLM), takich jak GPT (Generative Pre-trained Transformer) od OpenAI, BERT (Bidirectional Encoder Representations from Transformers) od Google czy Transformerów od innych wiodących instytucji badawczych i firm technologicznych. Niezwykle obiecujące są także inicjatywy otwarto-źródłowe, które umożliwiają społeczności naukowej i technologicznej trenowanie oraz dostosowywanie modeli Sztucznej Inteligencji do specyficznych potrzeb i zastosowań. Postęp ten, zauważalny szczególnie na przełomie ostatnich dwóch lat, zrewolucjonizował wiele aspektów technologii, wprowadzając znaczące innowacje w przetwarzaniu języka naturalnego (NLP), generowaniu tekstu i kodu, rozumieniu kontekstu oraz interakcji maszyna-człowiek.

Warto zauważyć, że wiodące zespoły badawcze i firmy technologiczne, mające dostęp do najwyższej jakości danych i znacznych zasobów obliczeniowych, utrzymują przewagę nad rozproszonymi grupami działającymi na ograniczonych zasobach. Ta centralizacja zasobów i danych sprzyja szybkiemu rozwojowi i komercjalizacji dużych modeli językowych jako produktów i usług, co dodatkowo napędza ich popularność i umożliwia gromadzenie coraz większej ilości danych do treningu. Z kolei dla społeczności otwarto-źródłowej, mimo ogromnego potencjału i wkładu w rozwój technologii, dogonienie rozwiniętych komercyjnie modeli stanowi znaczące wyzwanie, które prawdopodobnie będzie wymagało dodatkowego czasu i zasobów. Obecnie na rynku bezspornie dominuje najpotężniejszy i najnowszy model własnościowy od OpenAI nazwany GPT-4-turbo, który oprócz dużych zdolności jest rozbudowany o funkcje niedostępne w innych produktach, jak dostęp do narzędzi i funkcji, które pozwalają zwiększyć możliwości Sztucznej Inteligencji w zastosowaniach autonomicznych i praktycznych. 

Zastosowanie LLM w cyberbezpieczeństwie otwiera drogę do bardziej dynamicznych i inteligentnych systemów detekcji błędów, które mogą adaptować się do ewoluujących zagrożeń i technik ataku. Poprzez analizę kodu w poszukiwaniu wzorców znanych podatności, a następnie generowanie propozycji ich naprawy, LLM mogą znacząco przyspieszyć proces zapewniania bezpieczeństwa aplikacji webowych, minimalizując ryzyko eksploatacji przez potencjalnych atakujących. Co więcej, zdolność LLM do uczenia się w kontekście (in-context learning) oraz wykorzystania metod wzbogacania generacji (retrieval-augmented generation - RAG) umożliwia tworzenie bardziej precyzyjnych i efektywnych rozwiązań, dostosowanych do specyfiki danego problemu bezpieczeństwa.

W kontekście niniejszej pracy inżynierskiej, rola LLM w cyberbezpieczeństwie zostanie poddana szczegółowej analizie, z uwzględnieniem zarówno potencjału, jak i wyzwań związanych z ich zastosowaniem. Badanie to ma na celu nie tylko ocenę skuteczności LLM w wykrywaniu i naprawianiu błędów bezpieczeństwa, ale także zrozumienie, w jaki sposób te zaawansowane narzędzia mogą być integrowane z istniejącymi procesami deweloperskimi i systemami zapewniania jakości, w celu stworzenia bardziej zabezpieczonych i odpornych na ataki aplikacji webowych.


\section*{Pytania badawcze}
W ramach pracy postawione zostają następujące pytania badawcze:
\begin{enumerate}
    \item Czy duże modele językowe mogą być wykorzystane do wykrywania i naprawiania błędów bezpieczeństwa w kodzie aplikacji webowych?
    \item Jak skuteczne są te modele w porównaniu z innymi rozwiązaniami?
    \item W jakim stopniu metody wzbogacania generacji (RAG) i uczenia się w kontekście (in-context learning) mogą poprawić skuteczność tych modeli?
    \item Jakie są ograniczenia i wyzwania związane z wykorzystaniem tych technologii w kontekście cyberbezpieczeństwa?
\end{enumerate}

\section*{Hipotezy}
\begin{enumerate}
    \item Duże modele językowe, dzięki swojej zdolności do analizy i generowania kodu, mogą skutecznie identyfikować i naprawiać błędy bezpieczeństwa w kodzie źródłowym.
    \item Mimo obiecującego potencjału, modele te mogą napotykać ograniczenia, szczególnie w bardziej złożonych i specyficznych scenariuszach związanych z cyberbezpieczeństwem.
\end{enumerate}

\section*{Uzasadnienie tytułu}
Tytuł pracy został dobrany tak, aby odzwierciedlał główny obszar zainteresowania badawczego, jakim jest wykorzystanie nowoczesnych technologii językowych w celu poprawy bezpieczeństwa aplikacji webowych. 
W kontekście rosnącej zależności od cyfrowych rozwiązań, temat ten zyskuje na znaczeniu, oferując nowe perspektywy i podejścia do zagadnień bezpieczeństwa.
Tytuł można skrócić do \textbf{''Zastosowanie dużych modeli językowych w statycznej analizie kodu''}, ponieważ tak nazywa się problem odnajdywania i korekcji błędów w kodzie źródłowym. 
Korpus badawczy pracy został rozszerzony względem tytułu o projekty open-source aplikacji natywnych i desktopowych oraz wycinki błędnego kodu i poprawnego kodu. 

\section*{Omówienie literatury naukowej i stopnia jej przydatności}
Podstawę teoretyczną pracy stanowi literatura naukowa skupiająca się na dużych modelach językowych oraz ich zastosowaniu w cyberbezpieczeństwie. Szczególną uwagę poświęcono artykułowi ''Can OpenAI Codex and Other Large Language Models Help Us Fix Security Bugs?'', który posłużył jako punkt wyjścia dla badań. 

Praca ta ma na celu kontynuację i poszerzenie zakresu tych badań, wykorzystując literaturę naukową jako fundament do eksploracji nowych możliwości w zakresie analizy i naprawy błędów w kodzie.
Różnica między tą pracą, a literaturą naukową polega na tym, że praca skupia się na praktycznym zastosowaniu modeli językowych w statycznej analizie kodu, podczas gdy literatura naukowa skupia się na badaniu możliwości Sztucznej Inteligencji w tym zakresie.

\section*{Cel pracy}
Głównym celem pracy jest zbadanie skuteczności dużych modeli językowych w wykrywaniu i naprawie błędów bezpieczeństwa i podatności w kodzie źródłowym aplikacji webowych. 
W tym kontekście można wyróżnić następujące cele pośrednie:
\begin{itemize}
    \item Opracowanie praktycznego rozwiązania do statycznej analizy kodu dla aplikacji webowych oraz lokalnych.
    \item Badanie skuteczności dużych modeli językowych w wykrywaniu podatności i luk bezpieczeństwa.
\end{itemize}
\section*{Zakres pracy}
Zakres pracy obejmuje:
\begin{itemize}
    \item Analizę istniejącej literatury i badań, w szczególności artykułu 'Can OpenAI Codex and Other Large Language Models Help Us Fix Security Bugs?'.
    \item Projekt i implementację narzędzia do statycznej analizy kodu opartego na modelach OpenAI.
    \item Przygotowanie zbiorów danych i przykładów z kodem zawierającym potencjalne podatności.
    \item Testowanie i porównanie skuteczności z innymi rozwiązaniami
    \item Analiza wyników i formułowanie wniosków.
\end{itemize}


\addcontentsline{toc}{chapter}{Analiza istniejącej literatury oraz dotychczasowych badań}
\chapter{Analiza istniejącej literatury oraz dotychczasowych badań}

\section{Can OpenAI Codex and Other Large Language Models Help Us Fix Security Bugs? - \scriptsize\textit{Hammond Pearce, Benjamin Tan, Baleegh Ahmad, Ramesh Karri, Brendan Dolan-Gavitt}}
% https://arxiv.org/pdf/2112.02125v1.pdf

\subsection{Metodyka}
W badaniu "Czy OpenAI Codex i inne duże modele językowe mogą pomóc nam naprawić błędy bezpieczeństwa?"\cite{codex-fix-security-bugs} \url{https://arxiv.org/pdf/2112.02125v1.pdf} napisanym przez Hammond'a Pearce'a, Benjamina Tana, Baleegh'a Ahmad, Ramesh'a Karri oraz Brendan'a Dolan-Gavitt, autorzy skupili się na wykorzystaniu dużych modeli językowych (LLM) do naprawy podatności w kodzie w sposób zero-shot. Badanie koncentrowało się na projektowaniu monitów skłaniających LLM do generowania poprawionych wersji niebezpiecznego kodu. Przeprowadzono eksperymenty na szeroką skalę, obejmujące różne komercyjne modele LLM oraz lokalnie wytrenowany model.

\subsection{Wyniki}
Wyniki wykazały, że LLM mogą skutecznie naprawić 100\% syntetycznie wygenerowanych scenariuszy oraz 58\% podatności w historycznych błędach rzeczywistych projektów open-source. Odkryto, że różne sposoby formułowania informacji kluczowych w monitach wpływają na wyniki generowane przez modele. Zauważono, że wyższe temperatury generowania kodu przynoszą lepsze wyniki dla niektórych typów podatności, ale gorsze dla innych. 

Tak dobrych wyników niestety nie należy interpretować dosłownie, ponieważ z racji, że badanie przeprowadzono na reprezentatywnej próbie, autorzy nie byli w stanie ręcznie sprawdzać poprawności każdej naprawy i wykorzystali w tym celu istniejące narzędzia statycznej analizy kodu, takie jak CodeQL. W związku z powyższym, aby ocenić rzeczywistą skuteczność LLM w naprawianiu podatności, potrzebne są dalsze badania.
\section{Examining Zero-Shot Vulnerability Repair with Large Language Models - \scriptsize\textit{Hammond Pearce, Benjamin Tan, Baleegh Ahmad, Ramesh Karri, Brendan Dolan-Gavitt}}
% https://arxiv.org/pdf/2112.02125.pdf


W pracy naukowej pt. "Examining Zero-Shot Vulnerability Repair with Large Language Models"\cite{zero-shot-vuln-repair} \url{https://arxiv.org/pdf/2112.02125.pdf}, autorzy przedłużają swoje badania nad potencjałem wykorzystania Large Language Models (LLM) w kontekście naprawy podatności w kodzie źródłowym. Niniejsze badanie koncentruje się na wyzwaniach związanych z generowaniem funkcjonalnie adekwatnego kodu w realistycznych warunkach aplikacyjnych. Rozszerzając zakres swoich wcześniejszych prac, autorzy skupiają się na bardziej skomplikowanych przypadkach użycia LLM, eksplorując ich zdolność do efektywnego i efektywnego adresowania złożonych problemów związanych z bezpieczeństwem oprogramowania.

Podstawowe pytania badawcze były następujące:
\begin{enumerate}
    \item Czy LLM mogą generować bezpieczny i funkcjonalny kod do naprawy podatności?
    \item Czy zmiana kontekstu w komentarzach wpływa na zdolność LLM do sugerowania poprawek?
    \item Jakie są wyzwania przy używaniu LLM do naprawy podatności w rzeczywistym świecie?
    \item Jak niezawodne są LLM w generowaniu napraw?
\end{enumerate}

Eksperymenty potwierdziły, że choć LLM wykazują potencjał, ich zdolność do generowania funkcjonalnych napraw w rzeczywistych warunkach jest ograniczona. Wyzwania związane z inżynierią promptów i ograniczenia modeli wskazują na potrzebę dalszych badań i rozwoju w tej dziedzinie.

\section{Różnice między obecną pracą a istniejącą literaturą}

W przeciwieństwie do dotychczasowych badań skoncentrowanych głównie na teoretycznym potencjale dużych modeli językowych (LLM) w kontekście zero-shot, niniejsza praca dyplomowa podejmuje kroki w kierunku praktycznego zastosowania tych technologii. Główną różnicą jest tutaj zastosowanie metod takich jak Retrieval Augmented Generation (RAG) oraz in-context learning, co przesuwa nasze podejście w stronę kontekstu few-shot. 

\begin{itemize}
    \item \textbf{Zastosowanie Metod RAG i In-context Learning:} W odróżnieniu od tradycyjnych podejść zero-shot, które polegają na generowaniu odpowiedzi bez uprzedniego dostosowania modelu do specyficznego zadania, moja praca wykorzystuje RAG i uczenie się w kontekście, aby lepiej dostosować modele do konkretnych scenariuszy związanych z bezpieczeństwem kodu. Te metody pozwalają na bardziej precyzyjną analizę i naprawę błędów w kodzie.
    
    \item \textbf{Praktyczne Zastosowanie Modeli Językowych:} Podczas gdy większość istniejących badań skupia się na badaniu możliwości SI w teorii, ta praca koncentruje się na praktycznym zastosowaniu modeli językowych do wykrywania i naprawiania błędów bezpieczeństwa w kodzie. Przez to podejście, praca ta dostarcza bezpośrednich, aplikatywnych rozwiązań, które mogą być wykorzystane w rzeczywistych środowiskach programistycznych.
\end{itemize}

Takie podejście pozwala nie tylko na zrozumienie teoretycznego potencjału LLM, ale także na ocenę ich praktycznej przydatności w realnych scenariuszach związanych z cyberbezpieczeństwem. Znacząco poszerza to zakres badań w dziedzinie wykorzystania sztucznej inteligencji do poprawy bezpieczeństwa aplikacji, dostarczając nowych perspektyw i rozwiązań.

\chapter{Metodyka rozwiązania}

W niniejszej pracy dyplomowej zastosowano szereg metod i środków, aby zaimplementować narzędzie do statycznej analizy kodu oraz zbadać i ocenić potencjał dużych modeli językowych w kontekście wykrywania i naprawiania błędów bezpieczeństwa w kodzie źródłowym aplikacji.

\begin{table}[H]
    \begin{adjustwidth}{-2cm}{-2cm}  % Zmniejszenie marginesów z obu stron
        \centering
    \begin{tabular}{|>{\bfseries}p{2.7cm}|p{5cm}|>{\bfseries}p{2.5cm}|p{5cm}|}
    \hline
    \multicolumn{4}{|c|}{\textbf{Metody i Środki}} \\
    \hline
    \textbf{Metoda} & \small{Opis} & \textbf{Środek} & \small{Opis} \\
    \hline
    \textbf{Zero-shot learning} & \small{Metoda uczenia maszynowego pozwalająca modelom wykonywać zadania bez wcześniejszego treningu, opierając się na zdolności do rozumienia i generalizacji.} & \textbf{Modele językowe GPT-3.5, GPT-4} & \small{Zaawansowane modele AI OpenAI do generowania tekstu i odpowiadania na zapytania.} \\
    \hline
    \textbf{Prompt engineering} & \small{Projektowanie promptów w celu uzyskania trafnych odpowiedzi od AI.} & \textbf{OpenAI Assistant API} & \small{API umożliwiające integrację modeli językowych w aplikacjach.} \\
    \hline
    \textbf{In-context learning} & \small{Uczenie się i dostosowywanie modeli AI na podstawie informacji zawartych w kontekście zapytań.} & \textbf{Zbiory danych z kodem} & \small{Zestawy danych z przykładami kodu zawierającymi błędy, używane do trenowania narzędzi do wykrywania podatności.} \\
    \hline
    \textbf{Retrieval Augmented Generation} & \small{Technika łącząca generowanie treści z wyszukiwaniem informacji, wspomagana przez OpenAI Assistant API.} & \textbf{Projekty open-source zawierające podatności} & \small{Publiczne projekty zawierające błędy bezpieczeństwa, używane w testowaniu aplikacji oraz ocenie skuteczności LLM.} \\
    \hline
    \textbf{Analiza porównawcza} & \small{Ocena różnych technik lub systemów poprzez porównanie.} & \textbf{Statyczne testy podatności} & \small{Narzędzia analizy statycznej kodu, np. CodeQL.} \\
    \hline
    \textbf{Programowanie obiektowe i funkcyjne} & \small{Dwa paradygmaty programowania, koncentrujące się odpowiednio na obiektach i funkcjach.} & \textbf{Rozwiązania komercyjne, np. Snyk} & \small{Narzędzia AI do zarządzania bezpieczeństwem oprogramowania.} \\
    \hline
    & & \textbf{Python 3.12} & \small{Najnowsza wersja języka Python z zaawansowanymi funkcjami.} \\
    \hline
    & & \textbf{Biblioteki: openai, asyncio} & \small{Biblioteki Pythona dla integracji z OpenAI i programowania asynchronicznego.} \\
    \hline
    & & \textbf{Komputer osobisty} & \small{Urządzenie do tworzenia i testowania oprogramowania.} \\
    \hline
    \end{tabular}
\end{adjustwidth}
    \caption{Metody i środki wykorzystane w projekcie i badaniu.}
    \label{tab:methods_tools}
\end{table}

    

Metody i środki te zostały wybrane, aby zapewnić efektywne i wszechstronne podejście do analizy i naprawy kodu. Generacja wspomagana pobieraniem danych (RAG ang. Retrieval Augmented Generation) 
oraz uczenie się w kontekście(in-context learning) umożliwiają efektywną analizę i generowanie kodu. 
Z kolei analiza porównawcza pozwala na ocenę skuteczności różnych modeli i podejść. 
Wykorzystanie modeli językowych GPT-3.5 i GPT-4, środowiska Ollama, oraz innych narzędzi i zasobów, zapewnia solidną bazę do przeprowadzenia kompleksowych testów i analiz.




\chapter{Projekt oraz implementacja rozwiązania}

\section{Wstęp}
W ramach pracy inżynierskiej opracowano kompleksowe narzędzie \textbf{gptester}, które wykorzystuje zaawansowane modele językowe do analizy statycznej kodu.
Narzędzie to wykorzystuje domyślnie model GPT-4 do generowania raportów na temat jakości kodu oraz proponowania napraw, ze szczególnym uwzględnieniem bezpieczeństwa kodu.

\section{Architektura systemu}
\subsection{Ogólny opis}
\textbf{GPTester} jest programem napisanym w języku Python, wykorzystującym model GPT-4 (lub GPT-3.5-turbo) dostarczony przez OpenAI.
Jest zaprojektowany do uruchamiania z linii poleceń, a wyniki jego pracy są zapisywane w pliku formatu markdown oraz do osobnego katalogu z plikami wynikowymi zawierającymi poprawiony kod.
Praktyczną funkcjonalność zapewnia wykorzystanie metod git, takich jak tworzenie i zapisywanie łatek (ang. patch), co umożliwia wygodne wprowadzanie poprawek do kodu oraz kontrolę wersji i łatwe scalanie kodu.

\subsection{Schemat blokowy}
\label{subsec:schemat_blokowy}

\begin{landscape}
\begin{figure}[p]
    \centering
    \includegraphics[width=\linewidth]{img/gptester.drawio.png}
    \caption{Schemat blokowy działania aplikacji \textit{gptester}}
    \label{fig:schemat_blokowy}
\end{figure}
\end{landscape}

\subsubsection{Dane wejściowe i wstępne przetwarzanie}
\begin{figure}[h]
    \centering
    % Adjust the clip and trim parameters as needed: {left bottom right top}
    \includegraphics[clip, trim=0cm 4cm 29cm 6cm, width=0.9\linewidth]{img/gptester.drawio.png}
    \caption{dane wejściowe w schemacie blokowym}
    \label{fig:przyciety_obrazek}
\end{figure}
Proces rozpoczyna się od dostarczenia \texttt{oryginalnego kodu z błędami}. Następnie jeśli użytkownik przygotował środowisko dla CodeQL zostanie ono wykorzystane do wstępnej analizy, a dane wynikowe zostaną wykorzystane do wzbogacenia poleceń(promptów). 
W przeciwnym wypadku polecenie(prompt) zostanie skreowany na podstawie danych posiadanych. 

\subsubsection{Generacja testów funkcjonalnych}
\begin{figure}[h]
    \centering
    % Adjust the clip and trim parameters as needed: {left bottom right top}
    \includegraphics[clip, trim=0cm 0cm 0cm 13cm, width=0.9\linewidth]{img/gptester.drawio.png}
    \caption{Część schematu opisująca proces generatora testów funkcjonalnych}
    \label{fig:przyciety_obrazz}
\end{figure}

Jeżeli testy funkcjonalne dla naszego kodu nie zostały zapewnione, zostaną one wygenerowane za pomocą osobnego agenta. 
Jest to funkcja nadal testowana, która w przyszłości ma za zadanie generować testy funkcjonalne dla kodu, który ich nie posiada. 
Problemem jest duża ilość kodu dla średnich i dużych projektów, dlatego zalecane jest dostarczenie własnego modułu testów funkcjonalnych.

\subsection{Wzbogacanie zapytań\footnote{ang. prompt}}
\begin{figure}[h]
    \centering
    % Adjust the clip and trim parameters as needed: {left bottom right top}
    \includegraphics[clip, trim=3cm 11cm 3cm 0cm, width=0.9\linewidth]{img/gptester.drawio.png}
    \caption{Część schematu opisująca proces RAG}
    \label{fig:RAG-schemat}
\end{figure}

Dla uzyskania jak najlepszych wyników oraz zapewnienia najnowszej dostępnej wiedzy na temat podatności wykorzystywane zostaje RAG, aby wzbogacić monit o dodatkowe informacje. Są to informacje otrzymane z CodeQL, które zostają użyte do semantycznego wyszukania powiązanych wpisów w bazie danych CVE. 
Jeżeli CodeQL nie został użyty, monit zostaje wzbogacony o informacje z bazy CVE, które zostały wyszukane semantycznie w wektorowej bazie wiedzy. W tej chwili o użyciu danych z bazy wiedzy decyduje wybrany model OpenAI, dzięki nowym możliwością API. 
Nowe możliwości API jak własne narzędzia dla LLM, code interpreter (interpreter kodu) oraz semantic search (semantyczne wyszukiwanie) zostały wprowadzone 06.11.2023r.
Sprawia to, że niniejszy własny kod do wzbogacania monitów jest niepotrzebny, ale w przyszłości może zostać użyty do wzbogacania monitów o dodatkowe informacje dla modeli otwartoźródłowych. 
\begin{listing}
    \begin{minted}{python}  
def get_embedding(text, model="text-embedding-ada-002"): # alternatively use code-embedding-ada
    text = str(text)
    text = text.replace("\n", " ")
    if len(text) != 0:
        return openai.Embedding.create(input=[text], model=model)["data"][0]["embedding"]
    return [0.0] * 1536

    \end{minted}
    \caption{Kod tworzący reprezentację wektorową tekstu za pomocą API OpenAI, domyślnie `text-embedding-ada-002', (models.py)} 
    \label{listing:vector_embedding}
\end{listing}
    
\begin{listing}
    \begin{minted}{python}  
    def relevance_for(self, query: str) -> float:
        embedding = get_embedding(query)
        task = get_embedding(self.name)
        score = cosine(task, embedding)
        return score
    \end{minted}
    \caption{Kod porównujący semantyczną odległość (models.py)} 
    \label{listing:vector_relevance}
\end{listing}
    
Przedstawione zostały rzeczywiste skrawki kodu znajdujące się w projekcie, natomiast w testach oraz podczas działania na modelach komeryjnych OpenAI używane są funkcje dostępne za pomocą API.

\definecolor{light-blue}{rgb}{0.8,0.85,1}

\subsubsection{Tworzenie monitów i interakcja z LLM}
\begin{figure}[h]
    \centering
    % Adjust the clip and trim parameters as needed: {left bottom right top}
    \includegraphics[clip, trim=3cm 10cm 3cm 0cm, width=0.9\linewidth]{img/gptester.drawio.png}
    \caption{Część schematu opisująca proces RAG}
    \label{fig:przyciety_obraz}
\end{figure}
Dla każdej części kodu badanego projektu opdowiednio mieszczącej się w oknie kontekstu dla modeli, tworzony jest \textbf{PROMPT} (monit), który jest następnie przetwarzany przez `\textbf{LLM (GPT-4, \ldots)}`. 
W tym celu wykorzywane są \textbf{Prompt templates} (szablony monitów) oraz semantycznie wyszukane skrawki z \textbf{Vector DB with CVEs and documentation} (wektorowa baza danych z CVE oraz dokumentacją).

Tak spreparowane zapytanie jest następnie zadane modelowi językowemu, który identyfikuje podatności oraz proponuje poprawki.

\subsubsection{Generowanie kodu i scalanie}
\begin{figure}[h]
    \centering
    % Adjust the clip and trim parameters as needed: {left bottom right top}
    \includegraphics[clip, trim=12cm 6cm 6cm 0cm, width=0.9\linewidth]{img/gptester.drawio.png}
    \caption{Czarna skrzynka - LLM (Large Language Model)}
    \label{fig:przyciety_obraz-generowanie}
\end{figure}
LLM generuje \texttt{uzupełnienie/generacje kodu} oraz raport, które są następnie łączone \textbf{(code merging)} w potencjalnie \texttt{naprawiony program}. Proces ten wykorzystuje również \textbf{narzędzia}
(code interpreter, knowledge retrieval, file writing, running tests, \dots), aby ułatwić obróbkę wyników oraz wzbogacić generację.

\subsubsection{Testy i weryfikacja}
\begin{figure}[h]
    \centering
    % Adjust the clip and trim parameters as needed: {left bottom right top}
    \includegraphics[clip, trim=25cm 2cm 0cm 4cm, width=0.9\linewidth]{img/gptester.drawio.png}
    \caption{Czarna skrzynka - LLM (Large Language Model)}
    \label{fig:przyciety_obraz2}
\end{figure}
W ramach procesu weryfikacyjnego, naprawiony kod jest poddawany \textbf{testom bezpieczeństwa} oraz \textbf{testom funkcjonalnym}, mając na celu zapewnienie, że wprowadzone poprawki nie generują nowych defektów oraz że aplikacja funkcjonuje zgodnie z założeniami. W literaturze źródłowej, na której opiera się niniejsza praca inżynierska, do realizacji testów bezpieczeństwa stosowane są narzędzia takie jak CodeQL, czy ASAN/UBSAN, które umożliwiają wykrycie podatności w kodzie. Proces ten jest zautomatyzowany, co jest kluczowe dla przeprowadzenia badań w skali naukowej. Jednakże, z powodu specyfiki integracji CodeQL w ramach realizowanego projektu, jego wykorzystanie do testów nie jest rekomendowane, gdyż nie zapewnia wiarygodności wyników. W związku z tym, testy bezpieczeństwa są przeprowadzane manualnie w celu potwierdzenia prawidłowości działania systemu, podczas gdy testy funkcjonalne realizowane są w sposób automatyczny. 
Zaleca się wskazanie dedykowanego modułu testowego dla skanowanego projektu.
\subsubsection{Dokumentacja i raportowanie}
Wyniki pracy \textbf{gptester} są dokumentowane w raporcie o błędach, po czym raport jest zapisywany w plikach markdown, a poprawione pliki z kodem w katalogu `fixed` z odpowiednim znacznikiem czasowym. W przyszłości poprawki będą wprowadzane do bazy kodu za pomocą git patch.

\subsubsection{Podsumowanie}
Diagram blokowy przedstawia kompleksowy proces analizy i naprawy kodu, który jest silnie zależny od danych wejściowych (kod źródłowy i testy bezpieczeństwa), zaawansowanych algorytmów przetwarzania (duże modele językowe) oraz dokładności w generowaniu poprawek kodu i ich weryfikacji. Cały proces jest automatyzowany, a weryfikacja jest możliwa na otrzymanych wynikach.


\section{Implementacja oraz użycie}
\subsection{Środowisko programistyczne i wymagania}

Projekt \textbf{gptester} został opracowany w środowisku programistycznym Python, z wykorzystaniem modelu GPT-4 dostarczonego przez OpenAI. Proces konfiguracji środowiska rozpoczyna się od przygotowania odpowiedniego środowiska Pythona i zainstalowania niezbędnych zależności.

Wymagania wstępne:
\begin{itemize}
    \item Python w wersji >3.7 – Język programowania wykorzystany do napisania `gptester`.
    \item Dostęp do internetu – Niezbędny do pobrania zależności i interakcji z modelem GPT-4 przez API OpenAI.
\end{itemize}

Instalacja zależności:
\begin{listing}
    \begin{minted}{bash}
pip install -r requirements.txt
\end{minted}
\end{listing}

Plik `requirements.txt` zawiera wszystkie niezbędne biblioteki Pythona wymagane do działania \textbf{gptester}. 
Instalacja zależności jest prosta i może być wykonana w terminalu lub wirtualnym środowisku Pythona, co jest zalecane w celu uniknięcia konfliktów z istniejącymi pakietami.

Do poprawnego działania wspieranego CodeQL należy zainstalować narzędzie CodeQL-CLI, które jest dostępne na stronie \url{https://codeql.github.com/}
\subsection{Uruchomienie programu}

\begin{listing}
    \begin{minted}{bash}
        cd gptester
        python main.py -h
    \end{minted}
\end{listing}
lub 
\begin{listing}
    \begin{minted}{bash}
        cd gptester
        chmod +x main.py
        ./main.py --help
    \end{minted}
\end{listing}

\subsection{Funkcje programu}

Program \textbf{gptester} został zaprojektowany jako wszechstronne narzędzie do analizy statycznej kodu, korzystając z zaawansowanych modeli językowych do wykrywania i naprawiania błędów bezpieczeństwa w kodzie. Wersja \texttt{assistant-0.5.2-beta} programu oferuje szereg funkcji, które są dostępne za pomocą argumentów linii poleceń, umożliwiając szeroką konfigurację i dostosowanie do specyficznych potrzeb analizy.

\begin{itemize}
    \item \textbf{-h, --help}: Wyświetla pomoc programu, prezentując dostępne opcje i ich krótki opis, co ułatwia szybkie zrozumienie możliwości programu.
    \item \textbf{-v, --verbose}: Aktywuje tryb szczegółowych informacji, dzięki czemu \texttt{gptester} prezentuje dodatkowe informacje o każdym etapie przetwarzania, co jest przydatne do debugowania i głębszej analizy działania.
    \item \textbf{-f, --format}: Określa format pliku wyjściowego rezultatu, dla najnowszej wersji programu domyślnie jest to csv, ale może zostać zmieniony na markdown, jak domyślnie w starszych wersjach.
    \item \textbf{-m MODEL, --model MODEL}: Umożliwia wybór modelu językowego do analizy kodu, z domyślnym ustawieniem na "gpt-4-1106-preview". Pozwala na dostosowanie narzędzia do specyficznych wymagań projektu poprzez wybór innego dostępnego modelu.
    \item \textbf{-r, --retrieval}: Włącza Generację Wspomaganą Odnajdywaniem (Retrieval Augmented Generation - RAG) dla analizy kodu, zwiększając precyzję identyfikacji podatności przez odwołanie się do zewnętrznych baz wiedzy.
    \item \textbf{-k FILE\_TO\_KNOW, --file-to-know FILE\_TO\_KNOW}: Pozwala na wskazanie pliku, który ma być załączony do bazy wiedzy, co może pomóc w poprawie kontekstu analizy. Domyślnie ustawione na bazę danych CVE w formacie CSV.
    \item \textbf{-i IGNORE, --ignore IGNORE}: Umożliwia podanie ścieżki do pliku .gitignore, aby zignorować określone pliki podczas skanowania, co pomaga skupić się na kluczowych elementach kodu.
    \item \textbf{-p PATCH\_FILE, --patch-file PATCH\_FILE}: Określa ścieżkę dla wygenerowanego pliku z poprawkami, umożliwiając łatwe śledzenie i aplikowanie sugerowanych zmian w kodzie.
    \item \textbf{-o OUTPUT, --output OUTPUT}: Określa ścieżkę i nazwę pliku, do którego zostaną zapisane wyniki analizy. Pozwala na elastyczne zarządzanie dokumentacją wyników.
    \item \textbf{-t TESTS, --tests TESTS}: Wskazuje ścieżkę do testów funkcjonalnych, które mają być wykonane na projekcie. Ułatwia integrację z istniejącymi procedurami testowymi i pozwala na automatyczne generowanie testów, jeśli nie zostaną dostarczone.
    \item \textbf{-c, --codeql}: Włącza integrację z CodeQL, zaawansowanym narzędziem do analizy kodu. Wymaga zainstalowania narzędzia CodeQL-CLI i jest przeznaczone do wykrywania bardziej złożonych podatności.
    \item \textbf{--command COMMAND}: Umożliwia określenie polecenia budowania projektu, co jest niezbędne do prawidłowej integracji z CodeQL, szczególnie gdy w projekcie nie ma pliku cmake lub podobnego. Domyślnie ustawione na "make".
    \item \textbf{--language LANGUAGE}: Pozwala na określenie języka programowania projektu dla analizy CodeQL, z domyślnym ustawieniem na "cpp". Umożliwia dostosowanie analizy do różnych środowisk programistycznych.
\end{itemize}
\begin{figure}[H]
    \centering
    \includegraphics[width=\linewidth]{img/gptester-help.png}
    \caption{Wiadomość pomocnicza aplikacji \textit{gptester}}
    \label{fig:gptester-help}
\end{figure}

Przykład użycia z pełną konfiguracją:

\begin{listing}
    \begin{minted}{bash}
./main.py /ścieżka/do/projektu --verbose --format csv --model "gpt-4-1106-preview" --output "moj_raport.md" --tests "/ścieżka/do/testów" --codeql --command "mvn -B -DskipTests -DskipAssembly" --language "java"
\end{minted}
\end{listing}

W powyższym przykładzie, gptester analizuje kod znajdujący się w podanej ścieżce, z włączonym trybem szczegółowych informacji, korzystając z modelu GPT-4, zapisując wyniki do określonego pliku raportu, wykonując testy funkcjonalne, integrując z CodeQL, używając polecenia \texttt{mvn -B -DskipTests -DskipAssembly} do budowy projektu w języku Java. Budowa projektu jest wymagana przez CodeQL, dlatego ten argument jest wykorzystywany jedynie przy integracji z CodeQL. CdodeQL musi być zainstalowany i skonfigurowany w środowisku, aby analiza mogła się powieść, a podana komenda powinna budować projekt w trybie czystym\footnote{ang. clean build}, aby przez kompilator niewykorzystywane były artefakty z poprzednich budowań.

\section{Integracja z CodeQL}
\label{sec:integracja_codeql}

Integracja GPTester z CodeQL znacznie rozszerza jego funkcjonalność analizy statycznej kodu. CodeQL, opracowany przez Microsoft, a dokładnie GitHub, to zaawansowane narzędzie do semantycznej analizy kodu, które umożliwia wykrywanie złożonych podatności i błędów bezpieczeństwa.

\textbf{Główne cechy integracji z CodeQL:}
\begin{itemize}
    \item \textbf{Zaawansowana Analiza Bezpieczeństwa}: CodeQL przekształca kod źródłowy w zapytywalną formę, co pozwala na przeprowadzenie głębokich analiz w poszukiwaniu subtelnych luk bezpieczeństwa.
    \item \textbf{Wsparcie Dla Wielu Języków}: Obsługa różnych języków programowania przez CodeQL, takich jak C++, Java, Python, co jest wykorzystywane przez `gptester` do analizy różnorodnych projektów.
    \item \textbf{Konfiguracja Procesu Budowania}: Możliwość dostosowania polecenia budowania projektu za pomocą opcji \texttt{--command}, niezbędna w przypadku braku pliku konfiguracyjnego jak cmake w katalogu głównym.
    \item \textbf{Elastyczność Analizy}: Użytkownik może wybrać między szybkimi analizami a bardziej dogłębnymi badaniami, co umożliwia dostosowanie procesu do konkretnych wymagań projektu.
    \item \textbf{Automatyzacja Wykrywania Podatności}: CodeQL automatyzuje proces wykrywania podatności, zwiększając skuteczność i efektywność analizy bezpieczeństwa kodu.
\end{itemize}

Integracja z CodeQL czyni \textbf{gptester} narzędziem nie tylko do wykrywania błędów syntaktycznych i strukturalnych, ale także do efektywnego identyfikowania subtelniejszych problemów bezpieczeństwa, które mogą umknąć podczas standardowych analiz.
Do działania CodeQL niezbędne jest zainstalowanie CodeQL CLI, które można pobrać ze strony GitHub.

\section{Modele językowe}
\label{sec:modele_jezykowe}

W ramach projektu \textbf{gptester} zaimplementowano zaawansowane modele językowe dostarczone przez OpenAI, w szczególności GPT-4, które odegrały kluczową rolę w procesie analizy statycznej kodu. Modele te wykorzystują techniki uczenia maszynowego i sztucznej inteligencji do generowania odpowiedzi na podstawie dostarczonych danych.

\subsection{Wykorzystanie Modeli Językowych}
Modele językowe w projekcie \textbf{gptester} są wykorzystywane do identyfikacji i sugerowania potencjalnych napraw błędów w kodzie źródłowym. Proces ten opiera się na zaawansowanej analizie kontekstu i semantyki kodu, co pozwala na precyzyjne wykrywanie nawet subtelnych podatności.

\subsection{Dostępne Modele}
Projekt integruje różne wersje modeli GPT, z dominującą rolą GPT-4, który charakteryzuje się wyższą zdolnością do zrozumienia złożonych zapytań i generowania bardziej precyzyjnych odpowiedzi. Dostępność innych modeli, takich jak GPT-3.5, zapewnia elastyczność w doborze narzędzia w zależności od specyficznych wymagań analizy. 

\subsubsection{Modele otwartoźródłowe}
W przyszłości wdrożone zostaną także modele otwartoźródłowe, za pomocą narzędzia Ollama. Dostępne będą między innymi: Llama2, GPT-J, Mistral, Falcon, czy jakikolwiek model dostępny w repozytoriach Ollama. Narzędzie to pozwala na łatwe pobieranie modeli, konteneryzowane uruchamianie, a także dostęp za pomocą API. Zapewni to jeszcze większą elastyczność i dostosowanie do potrzeb projektu.

\subsection{Inżynieria Poleceń (Promptów)}
Kluczowym aspektem wykorzystania modeli językowych jest inżynieria promptów, czyli proces projektowania i optymalizowania zapytań w celu uzyskania jak najbardziej trafnych odpowiedzi od modelu. W projekt \textbf{gptester} zaimplementowano zestaw specjalnie opracowanych promptów, które są dostosowane do identyfikacji różnych rodzajów błędów i podatności w kodzie.

Polecenie systemowe używane dla agenta odpowiedzialnego za identyfikację i naprawę błędów, zapisującego wyniki w formacie csv wygląda następująco:

\begin{framed}
\scriptsize
\begin{verbatim}
You are a top-tier security specialist. You have been tasked with finding vulnerabilities and 
security bugs in a program. You will be given either a code snippet, a whole codefile or 
multiple files and optionally a description of errors found by CodeQL. 

You must find all the security vulnerabilities, list them and append all of them to the file 
"GPTest.csv" in the following format:
Vulnerability Name; File Name; Line Number; Severity(impact/risk); Description; Solution;

Next you will write the fixed code to a file with the same name as the original file in 
a new folder called "fixed". Always write the whole file with complete functional code. 
Always write the new file without asking for confirmation or more context, even when you 
only have a snippet of code write it to a new file.

write_file = {
    "name": "write_file",
    "description": "Writes content to a specified file.",
    "parameters": {
        "type": "object",
        "properties": {
            "filename": {"type": "string"},
            "content": {"type": "string"}
        },
        "required": ["filename", "content"]
    }
}

append_to_csv = {
    "name": "append_to_csv",
    "description": "Appends a row to an existing CSV file using a semicolon as the delimiter.",
    "parameters": {
        "type": "object",
        "properties": {
            "file_path": {
                "type": "string",
                "description": "The path to the CSV file to which the row will be appended."
            },
            "row_data": {
                "type": "array",
                "description": "A list of values to be formatted into a CSV row and appended 
                                to the file.",
                "items": {
                    "type": "string"
                }
            }
        },
        "required": ["file_path", "row_data"]
    }
}
\end{verbatim}
\end{framed}
\normalsize


Można zauważyć, że prompty są złożone z dwóch części: pierwsza część to opis zadania, które ma wykonać model, a druga część to opis funkcji, która ma zostać użyta do zapisania plików. W ten sposób model jest w stanie zrozumieć kontekst zadania oraz wykonać odpowiednie operacje. Użycie funkcji oraz zwrócenie odpowiedzi przez tę funkcję jest zaimplementowane w ai/assistant.py oraz tools.py \ref{lst:api_openai}. Lokalizacja zapisywanych plików jest zmieniana przez kod i niezależna od modelu.

Prompt dla innych punktów końcowych API będzie wyglądał inaczej. Niezbędna jest wówczas implementacja parsera dla odpowiedzi od modelu językowego, aby wyodrębnić odpowiednie informacje i zapisać do plików.

\subsection{Integracja z API OpenAI}
\label{sec:integracja_openai}
Komunikacja z modelami językowymi odbywa się za pośrednictwem API OpenAI, co umożliwia wykorzystanie najnowszych osiągnięć w dziedzinie sztucznej inteligencji bez konieczności posiadania zasobów obliczeniowych do lokalnego trenowania modeli. 

Można wyróżnić trzy główne sposoby komunikacji z API OpenAI:
\begin{enumerate}
    \item \textbf{Completion(Komplecja/Uzupełnienie)}: Punkt końcowy API zakończenia otrzymał ostateczną aktualizację w lipcu 2023 r. i ma inny interfejs niż nowy punkt końcowy dopełnienia czatu\footnote{ang. Chat Completion}. Zamiast danych wejściowych będących listą komunikatów, danymi wejściowymi jest dowolny ciąg tekstowy zwany poleceniem\footnote{ang. prompt}.

    Przykładowe wywołanie starszego interfejsu API Completions wygląda następująco:
    \begin{listing}
        \begin{minted}{python}
from openai import OpenAI
client = OpenAI()

response = client.completions.create(
  model="gpt-3.5-turbo-instruct",
  prompt="Write a tagline for an ice cream shop."
)
        \end{minted}
    \end{listing}
    \item \textbf{ChatCompletion(Komplecja/Uzupełnienie dialogowe)}: Modele czatu przyjmują listę wiadomości jako dane wejściowe i zwracają wiadomość wygenerowaną przez model jako dane wyjściowe. Chociaż format czatu został zaprojektowany tak, aby ułatwić wieloturowe rozmowy, jest równie przydatny w przypadku zadań jednoturowych bez żadnej rozmowy.

    Przykładowe wywołanie interfejsu API Chat Completions wygląda następująco:
    \begin{listing}[H]
        \begin{minted}{python}
from openai import OpenAI
client = OpenAI()

response = client.chat.completions.create(
  model="gpt-3.5-turbo",
  messages=[
    {"role": "system", "content": "You are a helpful assistant."},
    {"role": "user", "content": "Who won the world series in 2020?"},
    {"role": "assistant", "content": "The Los Angeles Dodgers won the World Series in 2020."},
    {"role": "user", "content": "Where was it played?"}
  ]
)
            \end{minted}
        \end{listing}
        \item \textbf{Assistant(Asystent)}: Punkt końcowy API Asystentów pozwala na tworzenie asystentów AI w ramach własnych aplikacji. Asystent posiada instrukcje i może wykorzystywać modele, narzędzia oraz wiedzę do odpowiadania na zapytania użytkowników. API Asystentów obecnie obsługuje trzy typy narzędzi: Interpreter Kodu, Pobieranie oraz Wywoływanie Funkcji. W przyszłości planujemy udostępnić więcej narzędzi stworzonych przez OpenAI oraz umożliwić dostarczanie własnych narzędzi na naszej platformie.
        
        Przykładowe wywołanie interfejsu API Asystentów wygląda następująco:
        \begin{listing}
            \begin{minted}{python}
assistant = client.beta.assistants.create(
    name="Math Tutor",
    instructions="You are a personal math tutor. Write and run code to answer math questions.",
    tools=[{"type": "code_interpreter"}],
    model="gpt-4-1106-preview"
)
            \end{minted}
        \end{listing}
\end{enumerate}

\subsection{Implementacja w projekcie}
Kod użyty w projekcie został dostosowany do nowego interfejsu Assistant API, który jest zgodny z najnowszymi wersjami modeli językowych. W projekcie znajduje się także kod wykorzystujący starsze interfejsy API, który może być użyty w przypadku starszych wersji modeli językowych. Ten kod oraz własna implementacja bazy wektorowej wynika z daty wprowadzenia punktu końcowego API asystantów, który został wprowadzony 06.11.2023r.
Kod użyty w projekcie znajduje się w pliku `ai/assistant.py` i wygląda następująco\footnote{wersja programu: \texttt{assistant-0.2.2-beta}, w nowszej wersji znajduje się większa ilość kodu, który nie jest istotny dla niniejszej pracy, jak nadpisywanie lokalizacji plików}:
\newgeometry{top=0.5cm, left=2cm,right=2cm, bottom=3cm}

\begin{listing}
    \begin{minted}[fontsize=\scriptsize]{python}
class Assistant():
    def __init__(self, role: str, name: str = "Assistant", model: str = 'gpt-3.5-turbo-1106', iol: IOlog = None, tools = None, messages = None) -> None:
        self.name = name
        self.iol = iol
        self.instructions = role
        self.assistant = client.beta.assistants.create(
            name=name,
            instructions=self.instructions,
            model=model,
            tools=tools if tools else [{"type": "code_interpreter"}, {"type": "retrieval"}]
        )
        self.thread = client.beta.threads.create(messages = messages)
    # pominięte metody przekształcania wiadomości, pozwalające na łatwą zmianę punktów końcowych API bez modyfikowania funkcji definiujących agentów. Inne punkty końcowe korzystają z osobnych klas

    async def next(self, messages: list[dict[str, str]]=None, prompt=None, directory: str = 'fixed'):
        if messages:
            self.messages_to_thread(messages)
        if prompt:
            self.fuser(self, prompt)
        try:
            run = client.beta.threads.runs.create(
                thread_id=self.thread.id,
                assistant_id=self.assistant.id,
                model=self.assistant.model if self.assistant.model else "gpt-4-1106-preview",
                instructions=self.instructions
            )
            # Polling mechanism to see if runStatus is completed
            run_status = client.beta.threads.runs.retrieve(thread_id=self.thread.id, run_id=run.id)
            while run_status.status != "completed":
                await asyncio.sleep(2)  # Sleep for 2 seconds before polling again
                run_status = client.beta.threads.runs.retrieve(thread_id=self.thread.id, run_id=run.id)
                tool_outputs = []
                # Check if there is a required action
                if run_status.required_action and run_status.required_action.type == "submit_tool_outputs":
                    for tool_call in run_status.required_action.submit_tool_outputs.tool_calls:
                        name = tool_call.function.name
                        arguments = json.loads(tool_call.function.arguments)
                        if "filename" in arguments and self.name == "debug_agent": 
                            filename = os.path.basename(arguments["filename"])
                            timestamp = datetime.datetime.now().strftime('%Y-%m-%d %H:%M:%S')
                            arguments["filename"] = os.path.join(directory, f'fixed_{timestamp}', filename)
                        # Check if the function exists in the tools module
                        if hasattr(tools, name):
                            function_to_call = getattr(tools, name)
                            response = await function_to_call(**arguments)
                            # Collect tool outputs
                            tool_outputs.append({"tool_call_id": tool_call.id, "output": response})
                # Submit tool outputs back
                if tool_outputs:
                    client.beta.threads.runs.submit_tool_outputs(
                        thread_id=self.thread.id,
                        run_id=run.id,
                        tool_outputs=tool_outputs
                    )
                if run_status.status == "failed":
                    raise Exception(f"Run failed with reason: {run_status.last_error}")
            # Get the last assistant message from the messages list
            messages = client.beta.threads.messages.list(thread_id=self.thread.id)
            response = [message for message in messages if message.run_id == run.id and message.role == "assistant"][-1]
            if response:
                self.iol.log(f"{response.content[0].text.value} \n")
        except TypeError:
            self.iol.log(f"TypeError: run[-1][\"content\"]: {run[-1]['content']}")
        return messages
    \end{minted}
    \caption{Kod używany do komunikacji z API OpenAI (ai/assistant.py), wersja programu: assistant-0.2.2-beta}
    \label{lst:api_openai}
\end{listing}

\restoregeometry

Przy używaniu punktu końcowego ChatCompletion niezbędna jest implementacja parsera, aby możliwe było wyodrębnienie wiadomości oraz zapisywanie do plików. Taki parser znajduje się w pliku \textit{utils/chat\_to\_files.py}. Przewaga ChatCompletion nad Assistant jest taka, że są to stabilniejsze metody z takimi funkcjami jak przesyłanie strumieniowe, czy ustalane parametry generowania komplecji, które nie są dostępne w Assistant. 

\section{Definicja Agenta AI}
Agent AI w kontekście projektu \textbf{gptester} definiuje się jako zaawansowany system komputerowy, który wykorzystuje techniki sztucznej inteligencji i uczenia maszynowego do automatyzacji zadań związanych w tym przypadku z analizą statyczną kodu źródłowego. Agent ten jest zaprogramowany do samodzielnego podejmowania decyzji na podstawie dostarczonych mu danych, mając na celu identyfikację i naprawianie błędów oraz podatności bezpieczeństwa w kodzie.

\subsection{Cechy Charakterystyczne}

Agent AI charakteryzuje się następującymi cechami:

\begin{itemize}
    \item \textbf{Autonomia}: Możliwość samodzielnego działania bez bezpośredniej interwencji człowieka, opierając się na zasadach i algorytmach AI.
    \item \textbf{Interaktywność}: Umiejętność komunikacji z użytkownikami lub innymi systemami w celu wymiany informacji i wykonania zadań.
    \item \textbf{Narzędzia}: Wykorzystanie różnorodnych narzędzi i funkcji do analizy i generowania rozwiązań.
    \item \textbf{Pamięć długotrwała}: Możliwość zapamiętywania informacji i wykorzystywania ich w przyszłych zadaniach, Zaimplementowana dzięki technice generowania wspomaganego pobieraniem (RAG), w tym celu możliwe są także inne rozwiązania. Nie wykorzystywana w projekcie.
\end{itemize}

\subsection{Zastosowanie w \textbf{gptester}}

W projekcie \textbf{gptester}, agent AI odgrywa kluczową rolę w:

\begin{itemize}
    \item \textbf{Wykrywaniu błędów}: Automatyczne identyfikowanie błędów w kodzie źródłowym.
    \item \textbf{Generowaniu napraw}: Proponowanie rozwiązań naprawczych dla wykrytych problemów.
    \item \textbf{Testowaniu}: Automatyzowanie procesu testowania kodu, w tym pisania testów i ich wykonywania.
\end{itemize}

Dzięki swojej zaawansowanej konfiguracji i integracji z modelami językowymi GPT-4, agenci AI w \textbf{gptester} pozwalają na nowoczesne podejście do analizy statycznej kodu, zapewniając łatwość w implementacji oraz wysoką skuteczność i efektywność w wykrywaniu oraz naprawianiu błędów programistycznych.

\section{Konfiguracja Agentów AI}
\label{subsec:konfiguracja_agentow}

Projekt \textbf{gptester} wykorzystuje zaawansowanych agentów AI, które są kluczowymi elementami w procesie analizy statycznej kodu. Konfiguracja tych agentów obejmuje szereg parametrów i narzędzi, które są niezbędne do ich efektywnego działania.

\subsection{Parametry Konfiguracyjne}

Każdy agent AI jest inicjalizowany z określonymi parametrami konfiguracyjnymi, które definiują jego zachowanie i funkcjonalność:

\begin{itemize}
    \item \textbf{Instrukcje (Rola/Prompt Systemowy)}: Określa podstawowy zakres działania agenta, np. debugowanie lub testowanie. Jest to odpowiednik polecenia systemowego (system prompt) w interakcji z modelami za pomocą punktu końcowego `ChatCompletion' w OpenAI API.
    \item \textbf{Nazwa}: Unikalna nazwa agenta, używana do identyfikacji i logowania.
    \item \textbf{Model Językowy}: Wskazuje na model AI używany przez agenta, domyślnie ustawiony na GPT-4 w najnowszej wersji.
    \item \textbf{Narzędzia}: Zestaw narzędzi, które agent może wykorzystywać podczas analizy.
\end{itemize}

Niestety, w punkcie końcowym API Asystentów nie ma możliwości przekazania parametrów konfigurujących komplecji, czy generacji, dlatego nie można przekazać parametrów takich jak `temperature' czy `max\_tokens'. W przyszłości, jeśli API Asystentów zostanie rozwinięte, będzie można przekazać te parametry i zbadać ich wpływ.
\subsection{Narzędzia i Funkcje}

Agent AI korzysta z różnorodnych narzędzi i funkcji, które wspierają jego działanie w różnych scenariuszach:

\begin{itemize}
    \item \textbf{Code Interpreter}: Narzędzie do interpretacji, analizy oraz wykonywania kodu źródłowego udostępniony przez OpenAI.
    \item \textbf{Retrieval}: Mechanizm wyszukiwania i odzyskiwania informacji z zewnętrznych źródeł udostępniony przez OpenAI.
    \item \textbf{Funkcje Własne}: Takie jak `write\_file', `run\_tests' i 'append\_to\_csv', które umożliwiają zapisywanie treści do plików i wykonanie testów.
    \item \textbf{Integracja z CodeQL}: W przypadku agenta ``debug\_agent'', integracja z CodeQL pozwala na głębszą analizę bezpieczeństwa kodu.
\end{itemize}


Każdy agent AI w \textbf{gptester} jest zaprojektowany w taki sposób, aby był elastyczny i mógł być łatwo dostosowany do zmieniających się wymagań projektowych, co umożliwia szerokie zastosowanie w różnych scenariuszach analizy kodu.

% \subsection{Przykładowa konfiguracja Agenta AI w projekcie}
% W niniejszym podrozdziale prezentuję przykład implementacji agenta AI, który został zaprojektowany do wykonywania zaawansowanych analiz kodu źródłowego oraz generowania odpowiednich rozwiązań naprawczych. Poniższy fragment kodu ilustruje wykorzystanie zestawu narzędzi i funkcji, które umożliwiają agentowi AI efektywne przetwarzanie i interpretację kodu, a także sugerowanie optymalnych metod naprawy wykrytych błędów. \\
% \vfill

% \begin{listing}
%     \begin{minted}[fontsize=\scriptsize]{python}
% async def test_agent(input: str, test: str, iol: IOlog = None, model: str = 'gpt-4-1106-preview', directory: str = 'tests') -> str:
%     """An agent used to test the supplied project
%     Capabililties: 
%         - Running tests for the project
%         - writing missing test files """
%     write_file_json = {
%         "name": "write_file",
%         "description": "Writes content to a specified file.",
%         "parameters": {
%             "type": "object",
%             "properties": {
%                 "filename": {"type": "string"},
%                 "content": {"type": "string"}
%             },
%             "required": ["filename", "content"]
%         }
%     }
%     run_tests_json = {
%         "name": "run_tests",
%         "description": "Executes test commands for specified programming languages using subprocesses.",
%         "parameters": {
%             "type": "object",
%             "properties": {
%                 "language": {
%                 "type": "string",
%                 "enum": ["cpp", "java", "python", "ruby", "php"]
%                 },
%                 "executable": {
%                 "type": "string",
%                 "description": "Path to the executable for C++ tests or additional command parameters for other languages."
%                 }
%             },
%             "required": ["language"]
%         }
%     }
%     tools=[
%         {"type": "code_interpreter"},
%         {"type": "retrieval"},
%         {"type": "function", "function": write_file_json},
%         {"type": "function", "function": run_tests_json},
%     ]

%     ai = Agent(role=f"{dbs.prompts['test']}", name='test_agent', iol = iol, tools=tools, model=model)
    
%     user = ai.fuser(msg=f"""The project codebase:\n{input}. User provided the following argument for tests: {test} \nRun appropriate tests for the project. If tests are not provided, please write them. Save them to a file and then run them. Use provided functions to do so.""")
    
%     messages = [user]
%     return await ai.next(messages, directory=directory)
% \end{minted}
% \caption{Implementacja testującego agenta AI (ai/agents.py)}
% \label{listing:przyklad_agent_ai}
% \end{listing}

\section{Rozwój i plany na przyszłość}
\label{sec:rozwoj_i_plany_na_przyszlosc}

Sekcja ta skupia się na omówieniu obecnego stanu projektu `gptester` oraz planowanych rozszerzeń i ulepszeń, które mają zostać wprowadzone w przyszłości. Planowane działania są zgodne z informacjami zawartymi w sekcji "In development" pliku README.md.

\subsection{Obecne osiągnięcia}
\label{subsec:obecne_osiagniecia}

Projekt `gptester` osiągnął już kilka kluczowych kamieni milowych w swoim rozwoju:

\begin{itemize}
    \item \textbf{Podstawowa funkcjonalność}: Program już teraz oferuje podstawowe funkcje analizy statycznej kodu, umożliwiając identyfikację typowych błędów i podatności.
    \item \textbf{Wykorzystanie technik RAG oraz in-context learning}: `gptester` wykorzystuje zaawansowane techniki generacji wspomaganej odzyskiwaniem danych oraz uczenia się w kontekście, co pozwala na lepsze dostosowanie modeli językowych do specyficznych zadań.
    \item \textbf{Integracja z CodeQL}: Znaczącym osiągnięciem jest wdrożenie integracji z CodeQL, co znacznie rozszerza możliwości analizy kodu, szczególnie w zakresie wykrywania złożonych błędów bezpieczeństwa. Niestety funkcja jest nadal testowana i stabilizowana.
    \item \textbf{Automatyzacja testów}: `gptester` automatyzuje proces testowania kodu, w tym pisania testów i ich wykonywania, co pozwala na efektywne sprawdzanie poprawności kodu.
    \item \textbf{Aktualizacja kodu za pomocą funkcji git patch i plików diff}: Funkcjonalność, która pozwola na automatyczne wprowadzanie poprawek do kodu źródłowego na podstawie wygenerowanych plików diff. To ulepszenie ułatwia proces naprawy kodu, umożliwiając automatyczne aplikowanie poprawek oraz interaktywne wybieranie elementów z obu wersji za pomocą systemu kontroli wersji git.
\end{itemize}

Te osiągnięcia stanowią solidną podstawę dla dalszego rozwoju i rozbudowy `gptester`, kładąc nacisk na wydajność, dokładność i wszechstronność narzędzia.

\subsection{Planowane rozszerzenia}
\label{subsec:planowane_rozszerzenia}

W ramach dalszego rozwoju, projekt `gptester` ma w planach kilka istotnych rozszerzeń i ulepszeń:

\begin{itemize}
    \item \textbf{Wsparcie dla modeli otwartoźródłowych}: Wdrożenie modeli otwartoźródłowych, takich jak GPT-J, Falcon, Mistral, czy Llama2, co pozwoli na jeszcze większą elastyczność i dostosowanie do różnych zastosowań.
    \item \textbf{Stabilizacja integracji z CodeQL}: Stabilizacja integracji z CodeQL, co pozwoli na lepsze wykrywanie błędów bezpieczeństwa i zwiększy użyteczność `gptester` w różnych projektach programistycznych.
    \item \textbf{Rozbudowa i stabilizacja modułu testowania}: Rozbudowa zestawu testów funkcjonalnych, jednostkowych i bezpieczeństwa, co pozwoli na lepsze sprawdzanie niezawodności i efektywności `gptester`. Automatyzacja testów pozwoli na skuteczniejsze wykrywanie błędów oraz ułatwi badania.
    \item \textbf{Obsługa większej ilości języków programowania dla CodeQL}: Rozszerzenie integracji z CodeQL o więcej języków programowania, co zwiększy użyteczność `gptester` w różnorodnych projektach programistycznych. Planowane jest dodanie wsparcia dla popularnych języków takich jak JavaScript, Python czy Ruby.
    \item \textbf{Poprawa znanego błędu aktualizaowania kodu systemem kontroli wersji git}: Poprawa błędu, który powoduje, że aktualizacja kodu za pomocą plików diff nie działa poprawnie w przypadku niektórych projektów. Wynika on z niepoprawnej hierarchi plików w folderze docelowym.
\end{itemize}

Te planowane rozszerzenia mają na celu nie tylko ulepszenie obecnych funkcjonalności gptester, ale również wprowadzenie nowych możliwości, które uczynią narzędzie jeszcze bardziej wszechstronnym i przydatnym w różnych kontekstach analizy kodu.
\section{Podsumowanie}
\label{sec:podsumowanie}

W niniejszym rozdziale przedstawiono szczegółowy opis projektu `gptester`, jego obecne możliwości oraz plany rozwojowe. `gptester`, jako zaawansowane narzędzie do analizy statycznej kodu, wykorzystujące model GPT-4 od OpenAI, stanowi znaczący krok naprzód w dziedzinie automatyzacji i poprawy jakości kodu źródłowego.


Podsumowując, `gptester` już teraz stanowi potężne narzędzie do analizy i poprawy kodu źródłowego, a planowane rozbudowy i ulepszenia sprawią, że będzie ono jeszcze bardziej wszechstronne i skuteczne. Projekt ten pokazuje, jak technologie AI i narzędzia do automatycznej analizy kodu mogą przyczynić się do poprawy jakości oprogramowania oraz efektywności procesów programistycznych.

\chapter{Zbiory danych i ich przygotowanie}
\label{ch:zbiory_danych}

W kontekście niniejszego rozdziału dokonano prezentacji oraz analizy zbiorów danych, które zostały wykorzystane w procesie testowania programu \textbf{gptester} oraz badania skuteczności dużych modeli językowych (LLM). Szczegółowo opisany został proces przygotowania i przetwarzania tych zbiorów danych, co ma kluczowe znaczenie dla efektywności analizy statycznej kodu i kalibracji narzędzia.


\section{Przegląd wykorzystanych zbiorów danych}
\label{sec:przeglad_zbiorow}

Następująca sekcja zawiera omówienie źródeł danych, ich specyfikacji oraz roli, jaką odgrywają w kontekście projektu. Analiza ta obejmuje zarówno otwarte zbiory danych, repozytoria kodu, jak i bazy danych podatności.


\begin{itemize}
    \item \textbf{snoopysecurity/Vulnerable-Code-Snippets}: Repozytorium w serwisie Github zawierające zbiór fragmentów kodu zawierających luki bezpieczeństwa. Fragmenty pobrane z różnych wpisów na blogach, książek, zasobów itp. 
    Zbiór w głównej mierze używany do testowania implementacji. Niektóre fragmenty kodu zawierają wskazówki w nazwach/komentarzach. Ewentualne naruszenie praw autorskich niezamierzone.\\ \url{https://github.com/snoopysecurity/Vulnerable-Code-Snippets}

    % \item \textbf{DiverseVul}: \url{https://arxiv.org/abs/2304.00409} Opis kolejnego zbioru danych, jego charakterystyka i zastosowanie w kontekście `gptester`.
    % \item \textbf{CVEfixes}: \url{https://github.com/secureIT-project/CVEfixes}
    \item \textbf{OWASP VulnerableApp}: Aplikacja webowa zawierająca wiele podatności, używana do testowania narzędzi do testowania bezpieczeństwa aplikacji webowych. \\ \url{https://github.com/SasanLabs/VulnerableApp}
    % Dodaj więcej zbiorów danych w razie potrzeby
\end{itemize}

\section{Proces przygotowania danych}
\label{sec:proces_przygotowania_danych}

Dobrane przeze mnie zbiory danych zostały tak, by nie trzeba było dostosowywać programu do konkretnego formatu. Oznacza to, że wskazane repozytoria zawierają przykłady kodu zapisane w plikach.
\subsection{snoopysecurity/Vulnerable-Code-Snippets} 
Repozytorium zawiera wiele plików z przykładami kodu, które mogą zawierać błędy bezpieczeństwa. Pliki te zostały pobrane z różnych źródeł, takich jak blogi, książki, zasoby itp. Pliki te zawierają często komentarze lub nazwy zmiennych, które wskazują na potencjalne błędy bezpieczeństwa. Pozwala to nam na izolację problemu identyfikacji podatności od generowania kodu. W pierwszej kolejności badania zostały przeprowadzone bez wprowadzania zmian w kodzie, aby ocenić skuteczność modeli językowych w korekcji błędów bezpieczeństwa. W kolejnym kroku, w celu przebadania zdolności do identyfikowania błędów, zostały wprowadzone zmiany w kodzie, takie jak usunięcie komentarzy, zmienienie nazw zmiennych, itp. W ten sposób można było sprawdzić, czy modele językowe są w stanie wykryć błędy bezpieczeństwa, gdy mają więcej informacji na temat kodu.

\subsection{OWASP VulnerableApp}
Aplikacja webowa zawierająca wiele podatności, używana w testach narzędzi do testowania bezpieczeństwa aplikacji webowych. Zawiera wiele przykładów kodu, które mogą zawierać błędy bezpieczeństwa. Dzięki użyciu w badaniu przykładu z prawdziwej aplikacji, można było sprawdzić, czy modele językowe są w stanie wykryć błędy bezpieczeństwa w prawdziwym kodzie.

\section{Wyzwania i ograniczenia}
\label{sec:wyzwania_i_ograniczenia}

Głównym wyzwaniem prezentowanym przez użyte przeze mnie próbki badawcze wynikają z ich charakteru. Zbiór danych Vulnerable-Code-Snippets nie jest reprezentatywne dla rzeczywistych aplikacji, a jedynie zawiera przykłady kodu, które mogą zawierać błędy bezpieczeństwa. W przypadku większości przykładów kodu, nie jest możliwe uruchomienie go bez posiadania kodu całego projektu, co utrudnia ewaluację. W związku z powyższym niektóre skrawki kodu zostały obudowane w aplikacje webową, natomiast inne pominięte. Nie każdy skrawek kodu w repozytorium jest wycięty z aplikacji webowej, te przykłady zostały uwzględnione w badaniach i sprawiały najmniej problemów. 

W przypadku OWASP VulnerableApp, aplikacja jest w pełni uruchamialna i możliwe jest badanie funkcjonalności programu na przykładzie rzeczywistym. Trudnością jest natomiast wczesna wersja 'gptester', w której nie została jeszcze wprowadzona funkcjonalność aktualizowania bazy kodu za pomocą funkcji git. Powoduje to konieczność ręcznego scalania zmian w kodzie z nowymi wersjami aplikacji, co jest czasochłonne i utrudnia badania. Testy bezpieczeństwa zostaną przeprowadzone za pomocą skanerów podatności, przede wszystkim OWASP ZAP, który jest zoptymalizowany do wykrywania podatności w aplikacji OWASP VulnerableApp, co pomoże w faktycznej ocenie skuteczności wprowadzonych korekt.

\subsection{Ograniczenia zbiorów danych}
Ograniczająca jest również liczba przykładów kodu w zbiorach danych. Wpływa to 

W przypadku repozytorium Vulnerable-Code-Snippets, niektóre przykłady kodu są bardzo podobne do siebie, co może wpływać na wyniki badań. W przypadku OWASP VulnerableApp, aplikacja zawiera wiele podatności, ale nie wszystkie są wykorzystywane w badaniach. W związku z powyższym, wyniki badań mogą być niepełne i nieodpowiednie do wyciągnięcia wniosków.

\section{Podsumowanie}
Podsumowujemy, jak przygotowanie i analiza zbiorów danych wpłynęła na projekt `gptester` i jakie wnioski można z tego wyciągnąć.


\chapter{Badania eksperymentalne - wyniki i wnioski}
\label{ch:badania_eksperymentalne}

W niniejszym rozdziale prezentujemy wyniki oraz wnioski wynikające z przeprowadzonych badań eksperymentalnych. Zestawienie rezultatów eksperymentów zostało dokonane w sposób zrozumiały i klarowny, z odpowiednim wsparciem graficznym w postaci wykresów i tabel. Wnioski są bezpośrednio powiązane z wynikami badań oraz są zgodne z założonymi celami projektu.

\section{Metodologia badań}
\label{sec:metodologia_badan}

W tej sekcji szczegółowo omawiamy metody oraz podejście zastosowane podczas eksperymentów. Zostaną przedstawione narzędzia, parametry konfiguracyjne oraz procedura testowa, które razem tworzą ramy metodyczne naszego badania.

\subsection{Konfiguracja środowiska}
\label{subsec:konfiguracja_srodowiska}

Do eksperymentów wykorzystano następujące ustawienia konfiguracyjne środowiska:
\begin{verbatim}
    > ./main.py -m 'gpt-4-1106-preview' Vulnerable-Code-Snippets/
\end{verbatim}

\subsection{Procedura testowa}
\label{subsec:procedura_testowa}

Zaprojektowana procedura testowa miała na celu dokładną weryfikację funkcjonalności programu oraz ocenę jego skuteczności w wykrywaniu i naprawie podatności. Kryteria testowe zostały dobrane w sposób umożliwiający kompleksową analizę:
\begin{itemize}
    \item \textbf{Kryterium 1}: Dokładność identyfikacji podatności.
    \item \textbf{Kryterium 2}: Skuteczność proponowanych napraw.
    \item \textbf{Kryterium 3}: Efektywność czasowa analizy.
\end{itemize}

Procedura testowa przebiegała według następujących etapów:
\begin{enumerate}
    \item Selekcja i przygotowanie danych testowych.
    \item Analiza statyczna kodu z wykorzystaniem narzędzi AI.
    \item Przeprowadzenie testów funkcjonalnych oraz testów bezpieczeństwa.
    \item Analiza i interpretacja wyników.
\end{enumerate}

\subsection{Wyniki działania programu - przykład}
\label{subsec:wyniki_dzialania_programu}

Wyniki działania programu, prezentowane na konsoli oraz dokumentowane w plikach raportów, zapewniają bezpośredni wgląd w proces analizy kodu:
% Tutaj należy wstawić odpowiednią zawartość raportu w formacie Markdown, o ile jest dostępna

\markdownInput{../raports/Authentication_Bypass_20240120120144_raport.md} 




\section{Badania na zbiorze \textit{snoopysecurity/Vulnerable-Code-Snippets}}
\label{sec:badania_na_zbiorze_snoopysecurity}

Analiza zbioru \textit{snoopysecurity/Vulnerable-Code-Snippets} dostarczyła istotnych informacji na temat specyfiki podatności i skuteczności ich wykrywania przez system. Zbiór ten, zawierający 184 pliki źródłowe o łącznej liczbie 41831 tokenów, stanowił reprezentatywną próbkę dla naszych eksperymentów.

Eksperymenty przeprowadzono z wykorzystaniem poniższych parametrów:
\begin{verbatim}
    > ./main.py -m 'gpt-4-1106-preview' Vulnerable-Code-Snippets/
\end{verbatim}


\begin{verbatim}
2024-01-20 12:54:36: Welcome to gptester: the Static Code Analysis Agent
2024-01-20 12:54:36: I will now begin scanning: Vulnerable-Code-Snippets/, name: Vulnerable-Code-Snippets
2024-01-20 12:54:36: Beginning scan...
2024-01-20 12:54:36: Found 184 files to scan
2024-01-20 12:54:36: Tokens inside the directory: 41831
2024-01-20 12:54:36: Beginning code analysis...
2024-01-20 12:54:36: Using model: gpt-4-1106-preview
\end{verbatim}



\subsection{Interpretacja wyników}
\label{subsec:interpretacja_wynikow}

Interpretacja wyników eksperymentów ujawniła istotne spostrzeżenia dotyczące możliwości wykorzystania modeli AI w celu poprawy bezpieczeństwa kodu. Analiza wykazała, że ...

% Tutaj należy dodać interpretację wyników w kontekście wykresów i tabel (jeśli takowe zostały przygotowane)

\section{Wnioski}
\label{sec:wnioski}

Na podstawie przeprowadzonych badań eksperymentalnych udało się zweryfikować założenia dotyczące efektywności wykorzystania modeli AI w procesie identyfikacji i naprawy podatności w kodzie źródłowym. Główne wnioski to:

% Tutaj należy wstawić wnioski z badania

% !TEX encoding = UTF-8 Unicode 
% !TEX root = praca.tex

\chapter*{Podsumowanie i wnioski}

\section*{Podsumowanie}
W pracy zbadano wykorzystanie dużych modeli językowych (LLM) w kontekście statycznej analizy kodu, skupiając się na ich zdolnościach do identyfikacji i naprawy błędów bezpieczeństwa. Rozpatrzono różne aspekty stosowania LLM, w tym ich integrację z istniejącymi narzędziami do analizy kodu, potencjał w automatyzacji procesów weryfikacji kodu oraz wyzwania związane z ich praktycznym zastosowaniem. Praca porusza również kwestie związane z ograniczeniami modeli językowych, takie jak ich zależność od złożoności danych wejściowych oraz konieczność humanitarnego nadzoru i weryfikacji wyników generowanych przez te systemy.

\section*{Odpowiedzi na pytania badawcze}
\begin{enumerate}
    \item \textbf{Wykorzystanie do wykrywania i naprawiania błędów:} LLM mogą efektywnie identyfikować i naprawiać standardowe błędy bezpieczeństwa w kodzie, jednak ich skuteczność maleje w bardziej złożonych scenariuszach.
    \item \textbf{Skuteczność w porównaniu z innymi rozwiązaniami:} LLM oferują obiecujące możliwości, ale wymagają ludzkiej ekspertyzy do weryfikacji i poprawy kodu, co sugeruje, że nie zastępują one całkowicie istniejących narzędzi, ale mogą je skutecznie uzupełniać.
    \item \textbf{Wpływ RAG i in-context learning:} Metody te mogą poprawić skuteczność LLM, jednak konieczne są dalsze badania nad optymalizacją ich zastosowania.
    \item \textbf{Ograniczenia i wyzwania:} Największe ograniczenia LLM to ograniczenia wielkości kontekstu, potencjalne halucynacje oraz niedetermistyczna natura, co może prowadzić do generowania nieoptymalnych, fałszywych a nawet niebezpiecznych sugestii.
\end{enumerate}

\section*{Weryfikacja hipotez}
\begin{enumerate}
    \item \textbf{Skuteczność w identyfikacji i naprawie błędów:} Zdolność LLM do identyfikowania podatności przerosła oczekiwania, niestety dokładne zbadanie skuteczeności sugerowanych napraw nie powiodło się. Ich zdolność do generowania funkcjonalnych napraw w rzeczywistych warunkach jest ograniczona.
    \item \textbf{Ograniczenia w złożonych scenariuszach:} Zdolność LLM do generowania funkcjonalnych napraw w rzeczywistych warunkach jest ograniczona, co potwierdza drugą hipotezę. Do poprawnego zastosowania LLM konieczna jest ludzka ekspertyza.
\end{enumerate}

\section*{Wnioski}
Duże modele językowe oferują obiecujące możliwości w analizie i naprawie kodu, jednak ich skuteczność jest ograniczona w złożonych scenariuszach cyberbezpieczeństwa. Wyniki badań podkreślają konieczność ludzkiej ekspertyzy w procesie weryfikacji i poprawy kodu. Dalsze badania są potrzebne do rozwoju metodologii i narzędzi, które pozwolą na pełniejsze wykorzystanie potencjału LLM w poprawie bezpieczeństwa aplikacji. 
\\
\textbf{Na podstawie przeprowadzonych badań, można wyciągnąć następujące wnioski:}

\begin{enumerate}
    \item LLM mogą efektywnie identyfikować podatności dla standardowych błędów w kodzie, ale ich skuteczność maleje wraz ze wzrostem złożoności zadania.
    \item Niezbadane pozostają błędy w identyfikacji podatności, pod względem zarówno fałszywie pozytywnych, jak i fałszywie negatywnych.
    \item LLM mogą być wykorzystywane do automatycznego generowania poprawek kodu, ale ich skuteczność jest ograniczona do prostych przypadków. W złożonych przypadkach konieczna jest ludzka ekspertyza, ale stosując optymalizacje przedstawionych metod oraz w dalszym ciągu rozwijając modele językowe, można zwiększyć skuteczność generowanych napraw.
    \item Modele te wymagają precyzyjnie sformułowanych zapytań i dobrze zdefiniowanych kontekstów, aby generować użyteczne wyniki. Konieczne są dodatkowe badania nad metodami wyboru i przygotowania danych wejściowych.
    \item Ograniczenia LLM, takie jak brak głębokiego zrozumienia logiki programistycznej i kontekstu biznesowego, mogą prowadzić do nieoptymalnych, niefunkcjonalnych lub niebezpiecznych sugestii. Wynika to między innymi z ograniczeń wielkości kontekstu, co uniemożliwia modelom zrozumienie złożonych zależności między kodem.
    \item Istotna jest ciągła interakcja i weryfikacja przez doświadczonych programistów, aby zapewnić bezpieczeństwo i poprawność proponowanych rozwiązań.
    \item Rozwój narzędzi wspomagających, które integrują LLM z tradycyjnymi metodami statycznej analizy kodu, może zwiększyć skuteczność wykrywania i naprawy błędów.
    \item Należy zachować ostrożność w kwestii etycznej i prawnej odpowiedzialności za błędy wprowadzone lub niezauważone przez modele językowe.
    \item Konieczne jest badanie wpływu na wydajność i jakość pracy programistów, w tym potencjalnych ryzyk związanych z nadmiernym poleganiem na automatycznych sugestiach.
\end{enumerate}


% Bibliography
\bibliographystyle{dyplom}
\bibliography{bibliography}

% Lists of figures, listings, tables
\listoffigures
\listoflistings
\listoftables

% Appendices - comment out if not applicable
\appendixpage
\appendix
\chapter{Surowe wyniki analizy kodu aplikacji NodeGoat bez podpowiedzi w kodzie w formacie Markdown przetłumaczonym na LaTex}\label{app1}

% {\large\textbf{}}
\normalsize
\begin{enumerate}
  \def\labelenumi{\arabic{enumi}.}
  \item
    \textbf{MongoClient Connection String Exposure}: The \texttt{db}
    variable likely contains the connection string to MongoDB, which
    includes credentials. This sensitive information can be exposed if the
    configuration file is not properly secured.
  \item
    \textbf{Insecure Express Session Configuration}: The session
    middleware in \texttt{server.js} uses a cookieSecret from a config
    file for signing cookies, but it does not have secure attributes such
    as \texttt{httpOnly}, \texttt{secure}, \texttt{sameSite}, as well as
    not using a store which can lead to security issues.
  \item
    \textbf{SWIG Template Auto-Escape Disabled}: In \texttt{server.js},
    the SWIG template engine is used with \texttt{autoescape:\ false},
    making it vulnerable to Cross-Site Scripting (XSS) attacks if user
    input is included in templates.
  \item
    \textbf{Use of Marked Library for Markdown Parsing}: The marked
    library is being used in \texttt{server.js} with the \texttt{sanitize}
    option set to \texttt{true}, which is good for preventing XSS.
    However, if library updates change the default behavior, it could lead
    to vulnerabilities.
  \item
    \textbf{Database Reset Script}: In \texttt{db-reset.js}, the process
    exits with a different status code in case of an error. This can lead
    to inconsistent behavior and potentially leak information about the
    database status or structure.
  \item
    \textbf{Lack of HTTPS}: The server in \texttt{server.js} is started
    with HTTP and not HTTPS, leading to the exposure of data in transit.
  \item
    \textbf{Open Docker Ports in Development}: The
    \texttt{docker-compose.yml} file shows ports open which should not be
    exposed in a production environment.
  \item
    \textbf{Hardcoded Credentials in \texttt{db-reset.js}}: Usernames and
    passwords are hardcoded in the \texttt{db-reset.js} script, which is a
    bad practice for managing credentials.
  \end{enumerate}
  
  Here are the corrected segments of code for each vulnerability found:
  
  \textbf{1. MongoClient Connection String Exposure}:
  
  \begin{framed}\scriptsize
  \begin{Highlighting}[]
  \NormalTok{// No diffs displayed as it requires infrastructural practices like environment variable management or secret management systems, which can\textquotesingle{}t be presented in source code directly.}
  \end{Highlighting}
  \end{framed}\normalsize
  
  \textbf{2. Insecure Express Session Configuration}:
  
  \begin{framed}\scriptsize 
  \begin{verbatim}
  -    app.use(session({
  -        secret: cookieSecret,
  -        saveUninitialized: true,
  -        resave: true
  -    }));
  +    app.use(session({
  +        secret: cookieSecret,
  +        saveUninitialized: true,
  +        resave: false,
  +        cookie: {
  +            httpOnly: true,
  +            secure: true,
  +            sameSite: "lax"
  +        },
  +        store: /* instance of secure session storage */
  +    }));
  \end{verbatim}
  \end{framed}
  
  \textbf{3. SWIG Template Auto-Escape Disabled}:

  \begin{framed}\scriptsize 
  \begin{verbatim}
  -   swig.setDefaults({
  -       // Autoescape disabled
  -       autoescape: false
  -   });
  +   swig.setDefaults({
  +       % Autoescape enabled
  +       autoescape: true
  +   });
  \end{verbatim}
  \end{framed}

  
  \textbf{4. Use of Marked Library for Markdown Parsing}: The current code
  has correctly set the \texttt{sanitize} option, but developers should
  remain vigilant for future library updates.
  
  \textbf{5. Database Reset Script (no explicit fix, informational only)}:
  Ensure consistent process exit codes and safeguard against information
  leakage.
  
  \textbf{6. Lack of HTTPS}:
  
  \begin{framed}\scriptsize
  \begin{verbatim}[]
-    http.createServer(app).listen(port, () => {
+   const https = require('https');
+   const fs = require('fs');
+
+   const options = {
+     key: fs.readFileSync('path/to/key.pem'),
+     cert: fs.readFileSync('path/to/cert.pem')
+   };
+
+   https.createServer(options, app).listen(port, () => {
        console.log(`Express http server listening on port ${port}`);
    });
  \end{verbatim}
  \end{framed}
  
  \textbf{7. Open Docker Ports in Development}: Adjust the
  \texttt{docker-compose.yml} configuration for production to not expose
  sensitive ports publicly.
  
  \textbf{8. Hardcoded Credentials in \texttt{db-reset.js}}:
  
  \begin{framed}\scriptsize
  \begin{verbatim}
-       "password": "Admin_123",
+       // Use environment variable or secure secrets management 
+       // to retrieve password
+       "password": process.env.ADMIN_PASSWORD,
  \end{verbatim}
  \end{framed}
  
  \begin{enumerate}
  \def\labelenumi{\arabic{enumi}.}
  \item
    \textbf{Insecure Password Storage (user-dao.js)}: Passwords should not
    be stored in plaintext. The \texttt{addUser} method currently stores
    the password received from the user directly into the database.
  


  \begin{framed}
    \scriptsize
  \begin{verbatim} % Use the 'lstlisting' environment for code highlighting
  - password %received from request param
  + password: bcrypt.hashSync(password, bcrypt.genSaltSync(8)) % Hash the password
  \end{verbatim}
  \end{framed}

  \item
    \textbf{Insecure Password Verification (user-dao.js)}: The
    \texttt{validateLogin} method uses a simple string comparison to
    validate passwords, which would only be secure if the passwords were
    properly hashed and salted before storage and then compared using a
    secure function.
  
  \begin{framed}
    \scriptsize
  \begin{verbatim}
- const comparePassword = (fromDB, fromUser) => { 
-  return fromDB === fromUser; 
-  }; 
+ const comparePassword = (fromDB, fromUser) => { 
+   return bcrypt.compareSync(fromUser, fromDB); 
+ };
  \end{verbatim}
  \end{framed}
  \item
    \textbf{Potential NoSQL Injection (research-dao.js, getBySymbol)}: The
    \texttt{getBySymbol} method creates a query without proper sanitation
    or parameterized queries, which may open up the application to NoSQL
    injection attacks.
  
  \begin{framed}
    \scriptsize
    \begin{verbatim}
-    symbol
+    symbol: symbol
+    // The searchCriteria function should return the properly constructed query object
+    db.collection('research').find(searchCriteria()).toArray(callback);
  \end{verbatim}
  \end{framed}
  \item
    \textbf{Potential NoSQL Injection (profile-dao.js, updateUser)}: The
    \texttt{updateUser} method directly uses the incoming parameter
    \texttt{userId} after parsing it as an integer. Although this reduces
    the risk, it's still a good habit to use a parameterized query.
  \begin{framed}
    \scriptsize
    \begin{verbatim}
-                    symbol
+                    symbol: symbol
+                // The searchCriteria function should return the properly constructed query object
+                db.collection('research').find(searchCriteria()).toArray(callback);
    \end{verbatim}
  \end{framed}
  \item
    \textbf{Lack of Input Validation}: Across various DAO functions, there
    is a lack of input validation to ensure that the values passed to the
    database operations do not contain malicious input.
  
    For this point, code modifications would be more extensive and not as
    straightforward to display in a patch format because proper input
    validation would need to be implemented throughout each function that
    takes external input.
  \end{enumerate}
  
  \hypertarget{vulnerability-6-nosql-injection-in-allocations}{%
  \paragraph{Vulnerability 6: NoSQL Injection in
  Allocations}\label{vulnerability-6-nosql-injection-in-allocations}}
  
  File: NodeGoat/app/data/allocations-dao.js
  
  Issue: The method \texttt{getByUserIdAndThreshold} is susceptible to
  NoSQL injection as it constructs a query using a \texttt{\$where}
  operator with user input, which can be manipulated.
  
  Vulnerability Fix:
  
  \begin{framed}
  \begin{verbatim}
this.getByUserIdAndThreshold = (userId, threshold, callback) => {
-    const parsedUserId = parseInt(userId);
-    const searchCriteria = () => {
-        if (threshold) {
-            return {
-                $where: `this.userId == ${parsedUserId} && \
                          this.stocks > '${threshold}'`
-            };
+    const parsedUserId = parseInt(userId)
+    let query = { userId: parsedUserId };
+
+    if (threshold) {
+        let numericThreshold = parseFloat(threshold);
+        if (!isNaN(numericThreshold)) {
+            query.stocks = { $gt: numericThreshold };
          }
-        return {
-            userId: parsedUserId
-        };
-    };
+    }
-    allocationsCol.find(searchCriteria()).toArray(...);
+    allocationsCol.find(query).toArray(...);
  };
  \end{verbatim}
  \end{framed}
  
  \hypertarget{vulnerability-7-insecure-configuration-in-github-workflow}{%
  \paragraph{Vulnerability 7: Insecure Configuration in Github
  Workflow}\label{vulnerability-7-insecure-configuration-in-github-workflow}}
  \\
  File: NodeGoat/.github/workflows/e2e-test.yml
  
  Issue: The configuration file uses hardcoded version \texttt{"4.0"} for
  the MongoDB Docker image, which might be outdated and contain known
  vulnerabilities.
  
  Vulnerability Fix:
  
  \begin{verbatim}
    - docker run -d -p 27017:27017 mongo:4.0
    + docker run -d -p 27017:27017 mongo:latest
    \end{verbatim}
    
  
  \hypertarget{vulnerability-8-arbitrary-redirect-in-index-route}{%
  \paragraph{Vulnerability 8: Arbitrary Redirect in Index
  Route}\label{vulnerability-8-arbitrary-redirect-in-index-route}}
  
  File: NodeGoat/app/routes/index.js
  
  Issue: The \texttt{/learn} route redirects to a user-specified URL
  without validation, which can be exploited for phishing attacks.
  
  Vulnerability Fix:
  
  \scriptsize
  \begin{verbatim}
app.get("/learn", isLoggedIn, (req, res) => {
-    return res.redirect(req.query.url);
+    const allowedUrls = ["https://trustedresource.com/learn", "https://anothertrustedsource.com/resources"];
+    const requestedUrl = req.query.url;
+    if (allowedUrls.includes(requestedUrl)) {
+        return res.redirect(requestedUrl);
+    } else {
+        return res.status(400).send("Invalid URL provided for redirection.");
+    }
    });
    \end{verbatim}
    \normalsize
  
  \hypertarget{vulnerability-9-insufficient-logging-and-monitoring-in-error-handler}{%
  \paragraph{Vulnerability 9: Insufficient Logging and Monitoring in Error
  Handler}\label{vulnerability-9-insufficient-logging-and-monitoring-in-error-handler}}
  
  File: NodeGoat/app/routes/error.js
  
  Issue: The error handling middleware logs the error message but doesn't
  notify the team or use a centralized logging system.
  
  Vulnerability Fix:
  
  \begin{framed}\scriptsize
    \begin{verbatim}
+ const {logger} = require("../log"); // Hypothetical logging module that should be created

const errorHandler = (err, req, res,next) => {
-    console.error(err.message);
-    console.error(err.stack);
+    logger.error(err.message, {stack: err.stack, req});
// ...
      };
      \end{verbatim}
  \end{framed}\normalsize
  
  \hypertarget{vulnerability-10-cross-site-scripting-xss-in-profile-data-rendering}{%
  \paragraph{Vulnerability 10: Cross-Site Scripting (XSS) in Profile Data
  Rendering}\label{vulnerability-10-cross-site-scripting-xss-in-profile-data-rendering}}
  
  File: NodeGoat/app/routes/profile.js
  
  Issue: Directly embedding user input from \texttt{doc.website} in the
  HTML without encoding it, which can lead to Cross-Site Scripting
  attacks.
  
  Vulnerability Fix:
  
  \begin{framed}\scriptsize
    \begin{verbatim}
this.displayProfile = (req, res, next) => {
    // ...
    doc.userId = userId;
-   doc.website = ESAPI.encoder().encodeForHTML(doc.website);
+   doc.website = ESAPI.encoder().encodeForURL(doc.website);
    // ...
      };
      \end{verbatim}
  \end{framed}\normalsize
  
  \hypertarget{vulnerability-11-insufficient-password-strength-validation}{%
  \paragraph{Vulnerability 11: Insufficient Password Strength
  Validation}\label{vulnerability-11-insufficient-password-strength-validation}}
  
  File: NodeGoat/app/routes/session.js
  
  Issue: The \texttt{PASS\_RE} regex allows for weak passwords that do not
  require a mix of character types.
  
  Vulnerability Fix:
  
  \begin{framed}\scriptsize
    \begin{verbatim}
const validateSignup = (...) => {
+    const PASS_RE = /^.*(?=.*\d)(?=.*[a-z])(?=.*[A-Z]).{8,18}$/;
-    const PASS_RE = /^.{1,20}$/;

-    if (!PASS_RE.test(password)) {
+    if (!PASS_RE.test(password)) {
        errors.passwordError = "Password must be 8 to 18 characters" +
-            " including numbers, lowercase and uppercase letters.";
+            " including at least one number, one lowercase letter, and one uppercase letter.";
        return false;
    }
    // ...
};
     \end{verbatim}
     
  \end{framed}\normalsize
  
  \hypertarget{analyzing-vulnerabilities}{%
  \subsubsection{Analyzing
  Vulnerabilities}\label{analyzing-vulnerabilities}}
  
  \hypertarget{vulnerability-1-remote-code-execution-in-config-file}{%
  \paragraph{Vulnerability 1: Remote Code Execution in Config
  File}\label{vulnerability-1-remote-code-execution-in-config-file}}
  
  File: NodeGoat/config/config.js
  
  Issue: User input (\texttt{finalEnv}) is used to construct a file path
  without proper validation, which could allow an attacker to traverse the
  filesystem or execute arbitrary code.
  
  Vulnerability Fix:
  
  \begin{framed}\scriptsize
    \begin{verbatim}
const validateSignup = (...) => {
+    const PASS_RE = /^.*(?=.*\d)(?=.*[a-z])(?=.*[A-Z]).{8,18}$/;
-    const PASS_RE = /^.{1,20}$/;

-    if (!PASS_RE.test(password)) {
+    if (!PASS_RE.test(password)) {
        errors.passwordError = "Password must be 8 to 18 characters" +
-            " including numbers, lowercase and uppercase letters.";
+            " including at least one number, one lowercase letter, and one uppercase letter.";
        return false;
    }
    // ...
};
     \end{verbatim}
     
  \end{framed}\normalsize
  
  \hypertarget{vulnerability-2-insecure-direct-object-references-idor-in-allocations}{%
  \paragraph{Vulnerability 2: Insecure Direct Object References (IDOR) in
  Allocations}\label{vulnerability-2-insecure-direct-object-references-idor-in-allocations}}
  
  File: NodeGoat/app/routes/allocations.js
  
  Issue: User input from \texttt{req.params.userId} is used directly to
  query the database, which could allow an unauthorized user to access
  other users' data.
  
  Vulnerability Fix:
  
  \begin{framed}\scriptsize
    \begin{verbatim}
this.displayAllocations = (req, res, next) => {
    
    const {
-        userId
+        userId: rawUserId
    } = req.params;
+
+    // Verify the user ID from the session
+    const {
+        userId: sessionUserId
+    } = req.session;
+
+    const isAuthorized = rawUserId == sessionUserId; // Use proper authorization check
+    if (!isAuthorized) {
+        return res.status(403).json({ error: "Unauthorized access" });
+    }

    const {
        threshold
    } = req.query;
// ...
      \end{verbatim}
      
  \end{framed}\normalsize
  
  \hypertarget{vulnerability-3-server-side-request-forgery-ssrf-in-research}{%
  \paragraph{Vulnerability 3: Server-Side Request Forgery (SSRF) in
  Research}\label{vulnerability-3-server-side-request-forgery-ssrf-in-research}}
  
  File: NodeGoat/app/routes/research.js
  
  Issue: The \texttt{url} and \texttt{symbol} parameters from user input
  are concatenated and used in a GET request without validation, allowing
  SSRF attacks.
  
  Vulnerability Fix:
  
  \begin{framed}\scriptsize
    \begin{verbatim}
this.displayResearch = (req, res) => {

    if (req.query.symbol) {
-       const url = req.query.url + req.query.symbol;
+       const allowedDomains = ["https://api.example.com"]; // Replace with the 
                                                           // actual domain you want to allow
+       const defaultResearchUrl = allowedDomains[0] + "/stock_info"
+       const safeSymbol = encodeURIComponent(req.query.symbol); // URI encode the symbol to avoid 
                                                                // manipulation
+       const url = defaultResearchUrl + "?symbol=" + safeSymbol;

        return needle.get(url, ...);
    }
// ...
      \end{verbatim}
      
  \end{framed}\normalsize
  
  \hypertarget{vulnerability-4-cross-site-scripting-xss-in-memos}{%
  \paragraph{Vulnerability 4: Cross-Site Scripting (XSS) in
  Memos}\label{vulnerability-4-cross-site-scripting-xss-in-memos}}
  
  File: NodeGoat/app/routes/memos.js
  
  Issue: User input \texttt{req.body.memo} is directly inserted into the
  database and later rendered without proper encoding.
  
  Vulnerability Fix:
  
  \begin{framed}\scriptsize
\begin{verbatim}
const MemosDAO = require("../data/memos-dao").MemosDAO;
const {
    environmentalScripts
} = require("../../config/config");
+const ESAPI = require("node-esapi");
+
function MemosHandler(db) {
// ...
    this.addMemos = (req, res, next) => {
+
+        // Encode memo content to avoid XSS
+        const encodedMemo = ESAPI.encoder().encodeForHTML(req.body.memo);

-        memosDAO.insert(req.body.memo, (err, docs) => {
+        memosDAO.insert(encodedMemo, (err, docs) => {
// ...
\end{verbatim}
      
  \end{framed}\normalsize
  
  \hypertarget{vulnerability-5-command-injection-in-contributions}{%
  \paragraph{Vulnerability 5: Command Injection in
  Contributions}\label{vulnerability-5-command-injection-in-contributions}}
  
  File: NodeGoat/app/routes/contributions.js
  
  Issue: Use of \texttt{eval} with user input \texttt{req.body.preTax},
  \texttt{req.body.afterTax}, and \texttt{req.body.roth}, enabling command
  injection attacks.
  
  Vulnerability Fix:
  
  \begin{framed}\scriptsize
\begin{verbatim}
  this.handleContributionsUpdate = (req, res, next) => {
  
  -        const preTax = eval(req.body.preTax);
  -        const afterTax = eval(req.body.afterTax);
  -        const roth = eval(req.body.roth);
  +        const preTax = parseFloat(req.body.preTax);
  +        const afterTax = parseFloat(req.body.afterTax);
  +        const roth = parseFloat(req.body.roth);
  
          const {
              userId
          } = req.session;
  // ...
  \end{verbatim}
      
  \end{framed}\normalsize

\end{document}
