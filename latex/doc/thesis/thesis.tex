% !TEX encoding = UTF-8 Unicode 
%
% Use:
% magister / inzynier - for master thesis or engineering thesis
% druk / archiwum - for print version or archive version
% en - to translate template into english
% examples:
%\documentclass[inzynier,druk,en] - master thesis, print version, english
%\documentclass[magister,druk,en]{dyplom}
%\documentclass[magister,druk]{dyplom}

\documentclass[magister,druk]{dyplom}

\usepackage[utf8]{inputenc}
\usepackage{hyperref}

% Maximum section's depth.
\setcounter{secnumdepth}{4}

% Listings settings
\setminted{breaklines, 
frame=lines,           
framesep=3mm,          
baselinestretch=1.1,   
fontsize=\small,       
% linenos              % line numbering
}

\usepackage{lipsum}

% \faculty{Faculty of \dots}                   % Uncomment if applicable
\fieldofstudy{Cyberbezpieczeńśtwo (CBD)}                          
\author{Patryk Fidler}
\title{Zastosowanie dużych modeli językowych do wykrywania i naprawiania błędów bezpieczeństwa i podatności w kodzie aplikacji webowych}
\supervisor{Dr. hab. inż. Maciej Piasecki}
% \consultant{Consultant's name}               % Uncomment if applicable
\specialisation{CBD}                         % Uncomment if applicable
\keywords{modele językowe, Sztuczna Inteligencja, statyczna analiza kodu}	% 3-5 keywords  

\begin{document}

\maketitle

\abstract{
% Polish abstract 
Praca inżynierska zatytułowana "Zastosowanie dużych modeli językowych do wykrywania i naprawiania błędów bezpieczeństwa i podatności w kodzie aplikacji webowych" koncentruje się na zastosowaniu zaawansowanych modeli językowych, takich jak GPT-3.5, GPT-4, a w przyszłości także modeli otwartoźródłowych, takich jak Mistral, do automatycznego wykrywania i naprawiania błędów bezpieczeństwa w kodzie oprogramowania i aplikacji webowych.

Motywacja tej pracy wynika z rosnącej roli dużych modeli językowych (LLM) w różnych dziedzinach, w tym w cyberbezpieczeństwie. W kontekście tych działań, badana jest możliwość wykorzystania tych modeli do wykrywania i naprawiania podatności takich jak XSS\footnote{Cross-Site Scripting}, SQL Injection\footnote{Iniekcja SQL}, CSRF\footnote{Cross-Site Request Forgery}, Buffer Overflow\footnote{Przepełnienie bufora} i tym podobne.

Punktem wyjścia dla pracy dyplomowej jest artykuł napisany w 2021 roku 
"Can OpenAI Codex and Other Large Language Models Help Us Fix Security Bugs?" \cite{codex-fix-security-bugs}
- Hammond Pearce, Benjamin Tan, Baleegh Ahmad, Ramesh Karri, Brendan Dolan-Gavitt \url{https://arxiv.org/pdf/2112.02125v1.pdf}. Autorzy podkreślają znaczący potencjał tych modeli, a niniejsza praca ma na celu kontynuację tych badań i poszerzenie ich zakresu.

Planowane działania obejmują zaprojektowanie oraz implementację praktycznego narzędzia dla programistów do statycznej analizy kodu, przygotowanie zbioru danych zawierającego podatne kod źródłowy, testowanie zdolności detekcji błędów przez modele językowe OpenAI, oraz porównanie tych wyników z istniejącymi rozwiązaniami.

Szczególny nacisk zostanie położony na wykorzystanie technik uczenia się w kontekście oraz generowanie wspomagane odnajdywaniem danych (RAG), które mogą pomóc w udoskonaleniu detekcji i wyników, nawet przy ograniczonych zasobach. 
Praca przewiduje implementację autonomicznego agenta AI zdolnego do analizy kodu, wykonania testów bezpieczeństwa i podejmowania decyzji na podstawie wyników tych testów i kontekstu.

W przyszłości badane będą możliwości detekcji błędów przez otwarte modele językowe, a w dalszej perspektywie, możliwości specjalizacji modeli w zakresie cyberbezpieczeństwa za pomocą dostrajania \footnote{fine-tuning}. Wszystkie te działania mają na celu nie tylko badanie, ale także poprawienie możliwości LLM w kontekście cyberbezpieczeństwa.

Głównym celem pracy jest zaproponowanie praktycznych rozwiązań, które mogą pomóc programistom w tworzeniu bardziej bezpiecznych aplikacji.

}{
% Abstract translated into English
The engineering thesis titled "Application of Large Language Models for Detecting and Repairing Security Errors and Vulnerabilities in Web Application Code" focuses on the use of advanced language models such as GPT-3.5, GPT-4, and in the future, open-source models like Mistral, for automatically detecting and repairing security errors in software code and web applications.

The motivation for this work stems from the growing role of large language models (LLMs) in various fields, including cybersecurity. In this context, the possibility of using these models to detect and repair vulnerabilities such as Cross-Site Scripting (XSS), SQL Injection, Cross-Site Request Forgery (CSRF), Buffer Overflow, and similar issues is explored.

The starting point for the thesis is the 2021 article "Can OpenAI Codex and Other Large Language Models Help Us Fix Security Bugs?" by Hammond Pearce, Benjamin Tan, Baleegh Ahmad, Ramesh Karri, Brendan Dolan-Gavitt. The authors highlight the significant potential of these models, and this work aims to continue their research and expand its scope.

Planned activities include the design and implementation of a practical tool for developers for static code analysis, preparation of a dataset containing vulnerable source code, testing the error detection capabilities of OpenAI language models, and comparing these results with existing solutions.

Special emphasis will be placed on the use of context-aware learning techniques and Retrieval-Augmented Generation (RAG), which can help improve detection and results, even with limited resources. The work anticipates the implementation of an autonomous AI agent capable of analyzing code, performing security tests, and proposing code fixes on the results of these tests and context.

Future research will explore the error detection capabilities of open language models, and in the longer term, the potential for specializing models in cybersecurity through fine-tuning. All these activities aim not only to study but also to improve the capabilities of LLMs in the context of cybersecurity.

The main goal of the thesis is to propose practical solutions that can help developers create more secure applications.


}

\tableofcontents


\chapter*{Wprowadzenie}

Ostatnie lata przyniosły niebywały rozwój w dziedzinie Sztucznej Inteligencji, a w szczególności w obszarze dużych modeli językowych (LLM), takich jak GPT (Generative Pre-trained Transformer) od OpenAI, BERT (Bidirectional Encoder Representations from Transformers) od Google czy Transformerów od innych wiodących instytucji badawczych i firm technologicznych. Niezwykle obiecujące są także inicjatywy otwarto-źródłowe, które umożliwiają społeczności naukowej i technologicznej trenowanie oraz dostosowywanie modeli Sztucznej Inteligencji do specyficznych potrzeb i zastosowań. Postęp ten, zauważalny szczególnie na przełomie ostatnich dwóch lat, zrewolucjonizował wiele aspektów technologii, wprowadzając znaczące innowacje w przetwarzaniu języka naturalnego (NLP), generowaniu tekstu i kodu, rozumieniu kontekstu oraz interakcji maszyna-człowiek.

Warto zauważyć, że wiodące zespoły badawcze i firmy technologiczne, mające dostęp do najwyższej jakości danych i znacznych zasobów obliczeniowych, utrzymują przewagę nad rozproszonymi grupami działającymi na ograniczonych zasobach. Ta centralizacja zasobów i danych sprzyja szybkiemu rozwojowi i komercjalizacji dużych modeli językowych jako produktów i usług, co dodatkowo napędza ich popularność i umożliwia gromadzenie coraz większej ilości danych do treningu. Z kolei dla społeczności otwarto-źródłowej, mimo ogromnego potencjału i wkładu w rozwój technologii, dogonienie rozwiniętych komercyjnie modeli stanowi znaczące wyzwanie, które prawdopodobnie będzie wymagało dodatkowego czasu i zasobów. Obecnie na rynku bezspornie dominuje najpotężniejszy i najnowszy model własnościowy od OpenAI nazwany GPT-4-turbo, który oprócz dużych zdolności jest rozbudowany o funkcje niedostępne w innych produktach, jak dostęp do narzędzi i funkcji, które pozwalają zwiększyć możliwości Sztucznej Inteligencji w zastosowaniach autonomicznych i praktycznych. 

Zastosowanie LLM w cyberbezpieczeństwie otwiera drogę do bardziej dynamicznych i inteligentnych systemów detekcji błędów, które mogą adaptować się do ewoluujących zagrożeń i technik ataku. Poprzez analizę kodu w poszukiwaniu wzorców znanych podatności, a następnie generowanie propozycji ich naprawy, LLM mogą znacząco przyspieszyć proces zapewniania bezpieczeństwa aplikacji webowych, minimalizując ryzyko eksploatacji przez potencjalnych atakujących. Co więcej, zdolność LLM do uczenia się w kontekście (in-context learning) oraz wykorzystania metod wzbogacania generacji (retrieval-augmented generation - RAG) umożliwia tworzenie bardziej precyzyjnych i efektywnych rozwiązań, dostosowanych do specyfiki danego problemu bezpieczeństwa.

W kontekście niniejszej pracy inżynierskiej, rola LLM w cyberbezpieczeństwie zostanie poddana szczegółowej analizie, z uwzględnieniem zarówno potencjału, jak i wyzwań związanych z ich zastosowaniem. Badanie to ma na celu nie tylko ocenę skuteczności LLM w wykrywaniu i naprawianiu błędów bezpieczeństwa, ale także zrozumienie, w jaki sposób te zaawansowane narzędzia mogą być integrowane z istniejącymi procesami deweloperskimi i systemami zapewniania jakości, w celu stworzenia bardziej zabezpieczonych i odpornych na ataki aplikacji webowych.


\section*{Pytania badawcze}
W ramach pracy postawione zostają następujące pytania badawcze:
\begin{enumerate}
    \item Czy duże modele językowe mogą być wykorzystane do wykrywania i naprawiania błędów bezpieczeństwa w kodzie aplikacji webowych?
    \item Jak skuteczne są te modele w porównaniu z innymi rozwiązaniami?
    \item W jakim stopniu metody wzbogacania generacji (RAG) i uczenia się w kontekście (in-context learning) mogą poprawić skuteczność tych modeli?
    \item Jakie są ograniczenia i wyzwania związane z wykorzystaniem tych technologii w kontekście cyberbezpieczeństwa?
\end{enumerate}

\section*{Hipotezy}
\begin{enumerate}
    \item Duże modele językowe, dzięki swojej zdolności do analizy i generowania kodu, mogą skutecznie identyfikować i naprawiać błędy bezpieczeństwa w kodzie źródłowym.
    \item Mimo obiecującego potencjału, modele te mogą napotykać ograniczenia, szczególnie w bardziej złożonych i specyficznych scenariuszach związanych z cyberbezpieczeństwem.
\end{enumerate}

\section*{Uzasadnienie tytułu}
Tytuł pracy został dobrany tak, aby odzwierciedlał główny obszar zainteresowania badawczego, jakim jest wykorzystanie nowoczesnych technologii językowych w celu poprawy bezpieczeństwa aplikacji webowych. 
W kontekście rosnącej zależności od cyfrowych rozwiązań, temat ten zyskuje na znaczeniu, oferując nowe perspektywy i podejścia do zagadnień bezpieczeństwa.
Tytuł można skrócić do \textbf{''Zastosowanie dużych modeli językowych w statycznej analizie kodu''}, ponieważ tak nazywa się problem odnajdywania i korekcji błędów w kodzie źródłowym. 
Korpus badawczy pracy został rozszerzony względem tytułu o projekty open-source aplikacji natywnych i desktopowych oraz wycinki błędnego kodu i poprawnego kodu. 

\section*{Omówienie literatury naukowej i stopnia jej przydatności}
Podstawę teoretyczną pracy stanowi literatura naukowa skupiająca się na dużych modelach językowych oraz ich zastosowaniu w cyberbezpieczeństwie. Szczególną uwagę poświęcono artykułowi ''Can OpenAI Codex and Other Large Language Models Help Us Fix Security Bugs?'', który posłużył jako punkt wyjścia dla badań. 

Praca ta ma na celu kontynuację i poszerzenie zakresu tych badań, wykorzystując literaturę naukową jako fundament do eksploracji nowych możliwości w zakresie analizy i naprawy błędów w kodzie.
Różnica między tą pracą, a literaturą naukową polega na tym, że praca skupia się na praktycznym zastosowaniu modeli językowych w statycznej analizie kodu, podczas gdy literatura naukowa skupia się na badaniu możliwości Sztucznej Inteligencji w tym zakresie.

\section*{Cel pracy}
Głównym celem pracy jest zbadanie skuteczności dużych modeli językowych w wykrywaniu i naprawie błędów bezpieczeństwa i podatności w kodzie źródłowym aplikacji webowych. 
W tym kontekście można wyróżnić następujące cele pośrednie:
\begin{itemize}
    \item Opracowanie praktycznego rozwiązania do statycznej analizy kodu dla aplikacji webowych oraz lokalnych.
    \item Badanie skuteczności dużych modeli językowych w wykrywaniu podatności i luk bezpieczeństwa.
\end{itemize}
\section*{Zakres pracy}
Zakres pracy obejmuje:
\begin{itemize}
    \item Analizę istniejącej literatury i badań, w szczególności artykułu 'Can OpenAI Codex and Other Large Language Models Help Us Fix Security Bugs?'.
    \item Projekt i implementację narzędzia do statycznej analizy kodu opartego na modelach OpenAI.
    \item Przygotowanie zbiorów danych i przykładów z kodem zawierającym potencjalne podatności.
    \item Testowanie i porównanie skuteczności z innymi rozwiązaniami
    \item Analiza wyników i formułowanie wniosków.
\end{itemize}


\addcontentsline{toc}{chapter}{Analiza istniejącej literatury oraz dotychczasowych badań}
\chapter{Analiza istniejącej literatury oraz dotychczasowych badań}

\section{Can OpenAI Codex and Other Large Language Models Help Us Fix Security Bugs? - \scriptsize\textit{Hammond Pearce, Benjamin Tan, Baleegh Ahmad, Ramesh Karri, Brendan Dolan-Gavitt}}
% https://arxiv.org/pdf/2112.02125v1.pdf

\subsection{Metodyka}
W badaniu "Czy OpenAI Codex i inne duże modele językowe mogą pomóc nam naprawić błędy bezpieczeństwa?"\cite{codex-fix-security-bugs} \url{https://arxiv.org/pdf/2112.02125v1.pdf} napisanym przez Hammond'a Pearce'a, Benjamina Tana, Baleegh'a Ahmad, Ramesh'a Karri oraz Brendan'a Dolan-Gavitt, autorzy skupili się na wykorzystaniu dużych modeli językowych (LLM) do naprawy podatności w kodzie w sposób zero-shot. Badanie koncentrowało się na projektowaniu monitów skłaniających LLM do generowania poprawionych wersji niebezpiecznego kodu. Przeprowadzono eksperymenty na szeroką skalę, obejmujące różne komercyjne modele LLM oraz lokalnie wytrenowany model.

\subsection{Wyniki}
Wyniki wykazały, że LLM mogą skutecznie naprawić 100\% syntetycznie wygenerowanych scenariuszy oraz 58\% podatności w historycznych błędach rzeczywistych projektów open-source. Odkryto, że różne sposoby formułowania informacji kluczowych w monitach wpływają na wyniki generowane przez modele. Zauważono, że wyższe temperatury generowania kodu przynoszą lepsze wyniki dla niektórych typów podatności, ale gorsze dla innych. 

Tak dobrych wyników niestety nie należy interpretować dosłownie, ponieważ z racji, że badanie przeprowadzono na reprezentatywnej próbie, autorzy nie byli w stanie ręcznie sprawdzać poprawności każdej naprawy i wykorzystali w tym celu istniejące narzędzia statycznej analizy kodu, takie jak CodeQL. W związku z powyższym, aby ocenić rzeczywistą skuteczność LLM w naprawianiu podatności, potrzebne są dalsze badania.
\section{Examining Zero-Shot Vulnerability Repair with Large Language Models - \scriptsize\textit{Hammond Pearce, Benjamin Tan, Baleegh Ahmad, Ramesh Karri, Brendan Dolan-Gavitt}}
% https://arxiv.org/pdf/2112.02125.pdf


W pracy naukowej pt. "Examining Zero-Shot Vulnerability Repair with Large Language Models"\cite{zero-shot-vuln-repair} \url{https://arxiv.org/pdf/2112.02125.pdf}, autorzy przedłużają swoje badania nad potencjałem wykorzystania Large Language Models (LLM) w kontekście naprawy podatności w kodzie źródłowym. Niniejsze badanie koncentruje się na wyzwaniach związanych z generowaniem funkcjonalnie adekwatnego kodu w realistycznych warunkach aplikacyjnych. Rozszerzając zakres swoich wcześniejszych prac, autorzy skupiają się na bardziej skomplikowanych przypadkach użycia LLM, eksplorując ich zdolność do efektywnego i efektywnego adresowania złożonych problemów związanych z bezpieczeństwem oprogramowania.

Podstawowe pytania badawcze były następujące:
\begin{enumerate}
    \item Czy LLM mogą generować bezpieczny i funkcjonalny kod do naprawy podatności?
    \item Czy zmiana kontekstu w komentarzach wpływa na zdolność LLM do sugerowania poprawek?
    \item Jakie są wyzwania przy używaniu LLM do naprawy podatności w rzeczywistym świecie?
    \item Jak niezawodne są LLM w generowaniu napraw?
\end{enumerate}

Eksperymenty potwierdziły, że choć LLM wykazują potencjał, ich zdolność do generowania funkcjonalnych napraw w rzeczywistych warunkach jest ograniczona. Wyzwania związane z inżynierią promptów i ograniczenia modeli wskazują na potrzebę dalszych badań i rozwoju w tej dziedzinie.

\section{Różnice między obecną pracą a istniejącą literaturą}

W przeciwieństwie do dotychczasowych badań skoncentrowanych głównie na teoretycznym potencjale dużych modeli językowych (LLM) w kontekście zero-shot, niniejsza praca dyplomowa podejmuje kroki w kierunku praktycznego zastosowania tych technologii. Główną różnicą jest tutaj zastosowanie metod takich jak Retrieval Augmented Generation (RAG) oraz in-context learning, co przesuwa nasze podejście w stronę kontekstu few-shot. 

\begin{itemize}
    \item \textbf{Zastosowanie Metod RAG i In-context Learning:} W odróżnieniu od tradycyjnych podejść zero-shot, które polegają na generowaniu odpowiedzi bez uprzedniego dostosowania modelu do specyficznego zadania, moja praca wykorzystuje RAG i uczenie się w kontekście, aby lepiej dostosować modele do konkretnych scenariuszy związanych z bezpieczeństwem kodu. Te metody pozwalają na bardziej precyzyjną analizę i naprawę błędów w kodzie.
    
    \item \textbf{Praktyczne Zastosowanie Modeli Językowych:} Podczas gdy większość istniejących badań skupia się na badaniu możliwości SI w teorii, ta praca koncentruje się na praktycznym zastosowaniu modeli językowych do wykrywania i naprawiania błędów bezpieczeństwa w kodzie. Przez to podejście, praca ta dostarcza bezpośrednich, aplikatywnych rozwiązań, które mogą być wykorzystane w rzeczywistych środowiskach programistycznych.
\end{itemize}

Takie podejście pozwala nie tylko na zrozumienie teoretycznego potencjału LLM, ale także na ocenę ich praktycznej przydatności w realnych scenariuszach związanych z cyberbezpieczeństwem. Znacząco poszerza to zakres badań w dziedzinie wykorzystania sztucznej inteligencji do poprawy bezpieczeństwa aplikacji, dostarczając nowych perspektyw i rozwiązań.

\chapter{Metodyka rozwiązania}

W niniejszej pracy dyplomowej zastosowano szereg metod i środków, aby zaimplementować narzędzie do statycznej analizy kodu oraz zbadać i ocenić potencjał dużych modeli językowych w kontekście wykrywania i naprawiania błędów bezpieczeństwa w kodzie źródłowym aplikacji.

\begin{table}[H]
    \begin{adjustwidth}{-2cm}{-2cm}  % Zmniejszenie marginesów z obu stron
        \centering
    \begin{tabular}{|>{\bfseries}p{2.7cm}|p{5cm}|>{\bfseries}p{2.5cm}|p{5cm}|}
    \hline
    \multicolumn{4}{|c|}{\textbf{Metody i Środki}} \\
    \hline
    \textbf{Metoda} & \small{Opis} & \textbf{Środek} & \small{Opis} \\
    \hline
    \textbf{Zero-shot learning} & \small{Metoda uczenia maszynowego pozwalająca modelom wykonywać zadania bez wcześniejszego treningu, opierając się na zdolności do rozumienia i generalizacji.} & \textbf{Modele językowe GPT-3.5, GPT-4} & \small{Zaawansowane modele AI OpenAI do generowania tekstu i odpowiadania na zapytania.} \\
    \hline
    \textbf{Prompt engineering} & \small{Projektowanie promptów w celu uzyskania trafnych odpowiedzi od AI.} & \textbf{OpenAI Assistant API} & \small{API umożliwiające integrację modeli językowych w aplikacjach.} \\
    \hline
    \textbf{In-context learning} & \small{Uczenie się i dostosowywanie modeli AI na podstawie informacji zawartych w kontekście zapytań.} & \textbf{Zbiory danych z kodem} & \small{Zestawy danych z przykładami kodu zawierającymi błędy, używane do trenowania narzędzi do wykrywania podatności.} \\
    \hline
    \textbf{Retrieval Augmented Generation} & \small{Technika łącząca generowanie treści z wyszukiwaniem informacji, wspomagana przez OpenAI Assistant API.} & \textbf{Projekty open-source zawierające podatności} & \small{Publiczne projekty zawierające błędy bezpieczeństwa, używane w testowaniu aplikacji oraz ocenie skuteczności LLM.} \\
    \hline
    \textbf{Analiza porównawcza} & \small{Ocena różnych technik lub systemów poprzez porównanie.} & \textbf{Statyczne testy podatności} & \small{Narzędzia analizy statycznej kodu, np. CodeQL.} \\
    \hline
    \textbf{Programowanie obiektowe i funkcyjne} & \small{Dwa paradygmaty programowania, koncentrujące się odpowiednio na obiektach i funkcjach.} & \textbf{Rozwiązania komercyjne, np. Snyk} & \small{Narzędzia AI do zarządzania bezpieczeństwem oprogramowania.} \\
    \hline
    & & \textbf{Python 3.12} & \small{Najnowsza wersja języka Python z zaawansowanymi funkcjami.} \\
    \hline
    & & \textbf{Biblioteki: openai, asyncio} & \small{Biblioteki Pythona dla integracji z OpenAI i programowania asynchronicznego.} \\
    \hline
    & & \textbf{Komputer osobisty} & \small{Urządzenie do tworzenia i testowania oprogramowania.} \\
    \hline
    \end{tabular}
\end{adjustwidth}
    \caption{Metody i środki wykorzystane w projekcie i badaniu.}
    \label{tab:methods_tools}
\end{table}

    

Metody i środki te zostały wybrane, aby zapewnić efektywne i wszechstronne podejście do analizy i naprawy kodu. Generacja wspomagana pobieraniem danych (RAG ang. Retrieval Augmented Generation) 
oraz uczenie się w kontekście(in-context learning) umożliwiają efektywną analizę i generowanie kodu. 
Z kolei analiza porównawcza pozwala na ocenę skuteczności różnych modeli i podejść. 
Wykorzystanie modeli językowych GPT-3.5 i GPT-4, środowiska Ollama, oraz innych narzędzi i zasobów, zapewnia solidną bazę do przeprowadzenia kompleksowych testów i analiz.



% !TEX encoding = UTF-8 Unicode 
% !TEX root = praca.tex

\chapter*{Podsumowanie i wnioski}

\section*{Podsumowanie}
W pracy zbadano wykorzystanie dużych modeli językowych (LLM) w kontekście statycznej analizy kodu, skupiając się na ich zdolnościach do identyfikacji i naprawy błędów bezpieczeństwa. Rozpatrzono różne aspekty stosowania LLM, w tym ich integrację z istniejącymi narzędziami do analizy kodu, potencjał w automatyzacji procesów weryfikacji kodu oraz wyzwania związane z ich praktycznym zastosowaniem. Praca porusza również kwestie związane z ograniczeniami modeli językowych, takie jak ich zależność od złożoności danych wejściowych oraz konieczność humanitarnego nadzoru i weryfikacji wyników generowanych przez te systemy.

\section*{Odpowiedzi na pytania badawcze}
\begin{enumerate}
    \item \textbf{Wykorzystanie do wykrywania i naprawiania błędów:} LLM mogą efektywnie identyfikować i naprawiać standardowe błędy bezpieczeństwa w kodzie, jednak ich skuteczność maleje w bardziej złożonych scenariuszach.
    \item \textbf{Skuteczność w porównaniu z innymi rozwiązaniami:} LLM oferują obiecujące możliwości, ale wymagają ludzkiej ekspertyzy do weryfikacji i poprawy kodu, co sugeruje, że nie zastępują one całkowicie istniejących narzędzi, ale mogą je skutecznie uzupełniać.
    \item \textbf{Wpływ RAG i in-context learning:} Metody te mogą poprawić skuteczność LLM, jednak konieczne są dalsze badania nad optymalizacją ich zastosowania.
    \item \textbf{Ograniczenia i wyzwania:} Największe ograniczenia LLM to ograniczenia wielkości kontekstu, potencjalne halucynacje oraz niedetermistyczna natura, co może prowadzić do generowania nieoptymalnych, fałszywych a nawet niebezpiecznych sugestii.
\end{enumerate}

\section*{Weryfikacja hipotez}
\begin{enumerate}
    \item \textbf{Skuteczność w identyfikacji i naprawie błędów:} Zdolność LLM do identyfikowania podatności przerosła oczekiwania, niestety dokładne zbadanie skuteczeności sugerowanych napraw nie powiodło się. Ich zdolność do generowania funkcjonalnych napraw w rzeczywistych warunkach jest ograniczona.
    \item \textbf{Ograniczenia w złożonych scenariuszach:} Zdolność LLM do generowania funkcjonalnych napraw w rzeczywistych warunkach jest ograniczona, co potwierdza drugą hipotezę. Do poprawnego zastosowania LLM konieczna jest ludzka ekspertyza.
\end{enumerate}

\section*{Wnioski}
Duże modele językowe oferują obiecujące możliwości w analizie i naprawie kodu, jednak ich skuteczność jest ograniczona w złożonych scenariuszach cyberbezpieczeństwa. Wyniki badań podkreślają konieczność ludzkiej ekspertyzy w procesie weryfikacji i poprawy kodu. Dalsze badania są potrzebne do rozwoju metodologii i narzędzi, które pozwolą na pełniejsze wykorzystanie potencjału LLM w poprawie bezpieczeństwa aplikacji. 
\\
\textbf{Na podstawie przeprowadzonych badań, można wyciągnąć następujące wnioski:}

\begin{enumerate}
    \item LLM mogą efektywnie identyfikować podatności dla standardowych błędów w kodzie, ale ich skuteczność maleje wraz ze wzrostem złożoności zadania.
    \item Niezbadane pozostają błędy w identyfikacji podatności, pod względem zarówno fałszywie pozytywnych, jak i fałszywie negatywnych.
    \item LLM mogą być wykorzystywane do automatycznego generowania poprawek kodu, ale ich skuteczność jest ograniczona do prostych przypadków. W złożonych przypadkach konieczna jest ludzka ekspertyza, ale stosując optymalizacje przedstawionych metod oraz w dalszym ciągu rozwijając modele językowe, można zwiększyć skuteczność generowanych napraw.
    \item Modele te wymagają precyzyjnie sformułowanych zapytań i dobrze zdefiniowanych kontekstów, aby generować użyteczne wyniki. Konieczne są dodatkowe badania nad metodami wyboru i przygotowania danych wejściowych.
    \item Ograniczenia LLM, takie jak brak głębokiego zrozumienia logiki programistycznej i kontekstu biznesowego, mogą prowadzić do nieoptymalnych, niefunkcjonalnych lub niebezpiecznych sugestii. Wynika to między innymi z ograniczeń wielkości kontekstu, co uniemożliwia modelom zrozumienie złożonych zależności między kodem.
    \item Istotna jest ciągła interakcja i weryfikacja przez doświadczonych programistów, aby zapewnić bezpieczeństwo i poprawność proponowanych rozwiązań.
    \item Rozwój narzędzi wspomagających, które integrują LLM z tradycyjnymi metodami statycznej analizy kodu, może zwiększyć skuteczność wykrywania i naprawy błędów.
    \item Należy zachować ostrożność w kwestii etycznej i prawnej odpowiedzialności za błędy wprowadzone lub niezauważone przez modele językowe.
    \item Konieczne jest badanie wpływu na wydajność i jakość pracy programistów, w tym potencjalnych ryzyk związanych z nadmiernym poleganiem na automatycznych sugestiach.
\end{enumerate}


% Bibliography
\bibliographystyle{dyplom}
\bibliography{bibliography}

% Lists of figures, listings, tables
\listoffigures
\listoflistings
\listoftables

% Appendices - comment out if not applicable
\appendixpage
\appendix
\chapter{Surowe wyniki analizy kodu aplikacji NodeGoat bez podpowiedzi w kodzie w formacie Markdown przetłumaczonym na LaTex}\label{app1}

% {\large\textbf{}}
\normalsize
\begin{enumerate}
  \def\labelenumi{\arabic{enumi}.}
  \item
    \textbf{MongoClient Connection String Exposure}: The \texttt{db}
    variable likely contains the connection string to MongoDB, which
    includes credentials. This sensitive information can be exposed if the
    configuration file is not properly secured.
  \item
    \textbf{Insecure Express Session Configuration}: The session
    middleware in \texttt{server.js} uses a cookieSecret from a config
    file for signing cookies, but it does not have secure attributes such
    as \texttt{httpOnly}, \texttt{secure}, \texttt{sameSite}, as well as
    not using a store which can lead to security issues.
  \item
    \textbf{SWIG Template Auto-Escape Disabled}: In \texttt{server.js},
    the SWIG template engine is used with \texttt{autoescape:\ false},
    making it vulnerable to Cross-Site Scripting (XSS) attacks if user
    input is included in templates.
  \item
    \textbf{Use of Marked Library for Markdown Parsing}: The marked
    library is being used in \texttt{server.js} with the \texttt{sanitize}
    option set to \texttt{true}, which is good for preventing XSS.
    However, if library updates change the default behavior, it could lead
    to vulnerabilities.
  \item
    \textbf{Database Reset Script}: In \texttt{db-reset.js}, the process
    exits with a different status code in case of an error. This can lead
    to inconsistent behavior and potentially leak information about the
    database status or structure.
  \item
    \textbf{Lack of HTTPS}: The server in \texttt{server.js} is started
    with HTTP and not HTTPS, leading to the exposure of data in transit.
  \item
    \textbf{Open Docker Ports in Development}: The
    \texttt{docker-compose.yml} file shows ports open which should not be
    exposed in a production environment.
  \item
    \textbf{Hardcoded Credentials in \texttt{db-reset.js}}: Usernames and
    passwords are hardcoded in the \texttt{db-reset.js} script, which is a
    bad practice for managing credentials.
  \end{enumerate}
  
  Here are the corrected segments of code for each vulnerability found:
  
  \textbf{1. MongoClient Connection String Exposure}:
  
  \begin{framed}\scriptsize
  \begin{Highlighting}[]
  \NormalTok{// No diffs displayed as it requires infrastructural practices like environment variable management or secret management systems, which can\textquotesingle{}t be presented in source code directly.}
  \end{Highlighting}
  \end{framed}\normalsize
  
  \textbf{2. Insecure Express Session Configuration}:
  
  \begin{framed}\scriptsize 
  \begin{verbatim}
  -    app.use(session({
  -        secret: cookieSecret,
  -        saveUninitialized: true,
  -        resave: true
  -    }));
  +    app.use(session({
  +        secret: cookieSecret,
  +        saveUninitialized: true,
  +        resave: false,
  +        cookie: {
  +            httpOnly: true,
  +            secure: true,
  +            sameSite: "lax"
  +        },
  +        store: /* instance of secure session storage */
  +    }));
  \end{verbatim}
  \end{framed}
  
  \textbf{3. SWIG Template Auto-Escape Disabled}:

  \begin{framed}\scriptsize 
  \begin{verbatim}
  -   swig.setDefaults({
  -       // Autoescape disabled
  -       autoescape: false
  -   });
  +   swig.setDefaults({
  +       % Autoescape enabled
  +       autoescape: true
  +   });
  \end{verbatim}
  \end{framed}

  
  \textbf{4. Use of Marked Library for Markdown Parsing}: The current code
  has correctly set the \texttt{sanitize} option, but developers should
  remain vigilant for future library updates.
  
  \textbf{5. Database Reset Script (no explicit fix, informational only)}:
  Ensure consistent process exit codes and safeguard against information
  leakage.
  
  \textbf{6. Lack of HTTPS}:
  
  \begin{framed}\scriptsize
  \begin{verbatim}[]
-    http.createServer(app).listen(port, () => {
+   const https = require('https');
+   const fs = require('fs');
+
+   const options = {
+     key: fs.readFileSync('path/to/key.pem'),
+     cert: fs.readFileSync('path/to/cert.pem')
+   };
+
+   https.createServer(options, app).listen(port, () => {
        console.log(`Express http server listening on port ${port}`);
    });
  \end{verbatim}
  \end{framed}
  
  \textbf{7. Open Docker Ports in Development}: Adjust the
  \texttt{docker-compose.yml} configuration for production to not expose
  sensitive ports publicly.
  
  \textbf{8. Hardcoded Credentials in \texttt{db-reset.js}}:
  
  \begin{framed}\scriptsize
  \begin{verbatim}
-       "password": "Admin_123",
+       // Use environment variable or secure secrets management 
+       // to retrieve password
+       "password": process.env.ADMIN_PASSWORD,
  \end{verbatim}
  \end{framed}
  
  \begin{enumerate}
  \def\labelenumi{\arabic{enumi}.}
  \item
    \textbf{Insecure Password Storage (user-dao.js)}: Passwords should not
    be stored in plaintext. The \texttt{addUser} method currently stores
    the password received from the user directly into the database.
  


  \begin{framed}
    \scriptsize
  \begin{verbatim} % Use the 'lstlisting' environment for code highlighting
  - password %received from request param
  + password: bcrypt.hashSync(password, bcrypt.genSaltSync(8)) % Hash the password
  \end{verbatim}
  \end{framed}

  \item
    \textbf{Insecure Password Verification (user-dao.js)}: The
    \texttt{validateLogin} method uses a simple string comparison to
    validate passwords, which would only be secure if the passwords were
    properly hashed and salted before storage and then compared using a
    secure function.
  
  \begin{framed}
    \scriptsize
  \begin{verbatim}
- const comparePassword = (fromDB, fromUser) => { 
-  return fromDB === fromUser; 
-  }; 
+ const comparePassword = (fromDB, fromUser) => { 
+   return bcrypt.compareSync(fromUser, fromDB); 
+ };
  \end{verbatim}
  \end{framed}
  \item
    \textbf{Potential NoSQL Injection (research-dao.js, getBySymbol)}: The
    \texttt{getBySymbol} method creates a query without proper sanitation
    or parameterized queries, which may open up the application to NoSQL
    injection attacks.
  
  \begin{framed}
    \scriptsize
    \begin{verbatim}
-    symbol
+    symbol: symbol
+    // The searchCriteria function should return the properly constructed query object
+    db.collection('research').find(searchCriteria()).toArray(callback);
  \end{verbatim}
  \end{framed}
  \item
    \textbf{Potential NoSQL Injection (profile-dao.js, updateUser)}: The
    \texttt{updateUser} method directly uses the incoming parameter
    \texttt{userId} after parsing it as an integer. Although this reduces
    the risk, it's still a good habit to use a parameterized query.
  \begin{framed}
    \scriptsize
    \begin{verbatim}
-                    symbol
+                    symbol: symbol
+                // The searchCriteria function should return the properly constructed query object
+                db.collection('research').find(searchCriteria()).toArray(callback);
    \end{verbatim}
  \end{framed}
  \item
    \textbf{Lack of Input Validation}: Across various DAO functions, there
    is a lack of input validation to ensure that the values passed to the
    database operations do not contain malicious input.
  
    For this point, code modifications would be more extensive and not as
    straightforward to display in a patch format because proper input
    validation would need to be implemented throughout each function that
    takes external input.
  \end{enumerate}
  
  \hypertarget{vulnerability-6-nosql-injection-in-allocations}{%
  \paragraph{Vulnerability 6: NoSQL Injection in
  Allocations}\label{vulnerability-6-nosql-injection-in-allocations}}
  
  File: NodeGoat/app/data/allocations-dao.js
  
  Issue: The method \texttt{getByUserIdAndThreshold} is susceptible to
  NoSQL injection as it constructs a query using a \texttt{\$where}
  operator with user input, which can be manipulated.
  
  Vulnerability Fix:
  
  \begin{framed}
  \begin{verbatim}
this.getByUserIdAndThreshold = (userId, threshold, callback) => {
-    const parsedUserId = parseInt(userId);
-    const searchCriteria = () => {
-        if (threshold) {
-            return {
-                $where: `this.userId == ${parsedUserId} && \
                          this.stocks > '${threshold}'`
-            };
+    const parsedUserId = parseInt(userId)
+    let query = { userId: parsedUserId };
+
+    if (threshold) {
+        let numericThreshold = parseFloat(threshold);
+        if (!isNaN(numericThreshold)) {
+            query.stocks = { $gt: numericThreshold };
          }
-        return {
-            userId: parsedUserId
-        };
-    };
+    }
-    allocationsCol.find(searchCriteria()).toArray(...);
+    allocationsCol.find(query).toArray(...);
  };
  \end{verbatim}
  \end{framed}
  
  \hypertarget{vulnerability-7-insecure-configuration-in-github-workflow}{%
  \paragraph{Vulnerability 7: Insecure Configuration in Github
  Workflow}\label{vulnerability-7-insecure-configuration-in-github-workflow}}
  \\
  File: NodeGoat/.github/workflows/e2e-test.yml
  
  Issue: The configuration file uses hardcoded version \texttt{"4.0"} for
  the MongoDB Docker image, which might be outdated and contain known
  vulnerabilities.
  
  Vulnerability Fix:
  
  \begin{verbatim}
    - docker run -d -p 27017:27017 mongo:4.0
    + docker run -d -p 27017:27017 mongo:latest
    \end{verbatim}
    
  
  \hypertarget{vulnerability-8-arbitrary-redirect-in-index-route}{%
  \paragraph{Vulnerability 8: Arbitrary Redirect in Index
  Route}\label{vulnerability-8-arbitrary-redirect-in-index-route}}
  
  File: NodeGoat/app/routes/index.js
  
  Issue: The \texttt{/learn} route redirects to a user-specified URL
  without validation, which can be exploited for phishing attacks.
  
  Vulnerability Fix:
  
  \scriptsize
  \begin{verbatim}
app.get("/learn", isLoggedIn, (req, res) => {
-    return res.redirect(req.query.url);
+    const allowedUrls = ["https://trustedresource.com/learn", "https://anothertrustedsource.com/resources"];
+    const requestedUrl = req.query.url;
+    if (allowedUrls.includes(requestedUrl)) {
+        return res.redirect(requestedUrl);
+    } else {
+        return res.status(400).send("Invalid URL provided for redirection.");
+    }
    });
    \end{verbatim}
    \normalsize
  
  \hypertarget{vulnerability-9-insufficient-logging-and-monitoring-in-error-handler}{%
  \paragraph{Vulnerability 9: Insufficient Logging and Monitoring in Error
  Handler}\label{vulnerability-9-insufficient-logging-and-monitoring-in-error-handler}}
  
  File: NodeGoat/app/routes/error.js
  
  Issue: The error handling middleware logs the error message but doesn't
  notify the team or use a centralized logging system.
  
  Vulnerability Fix:
  
  \begin{framed}\scriptsize
    \begin{verbatim}
+ const {logger} = require("../log"); // Hypothetical logging module that should be created

const errorHandler = (err, req, res,next) => {
-    console.error(err.message);
-    console.error(err.stack);
+    logger.error(err.message, {stack: err.stack, req});
// ...
      };
      \end{verbatim}
  \end{framed}\normalsize
  
  \hypertarget{vulnerability-10-cross-site-scripting-xss-in-profile-data-rendering}{%
  \paragraph{Vulnerability 10: Cross-Site Scripting (XSS) in Profile Data
  Rendering}\label{vulnerability-10-cross-site-scripting-xss-in-profile-data-rendering}}
  
  File: NodeGoat/app/routes/profile.js
  
  Issue: Directly embedding user input from \texttt{doc.website} in the
  HTML without encoding it, which can lead to Cross-Site Scripting
  attacks.
  
  Vulnerability Fix:
  
  \begin{framed}\scriptsize
    \begin{verbatim}
this.displayProfile = (req, res, next) => {
    // ...
    doc.userId = userId;
-   doc.website = ESAPI.encoder().encodeForHTML(doc.website);
+   doc.website = ESAPI.encoder().encodeForURL(doc.website);
    // ...
      };
      \end{verbatim}
  \end{framed}\normalsize
  
  \hypertarget{vulnerability-11-insufficient-password-strength-validation}{%
  \paragraph{Vulnerability 11: Insufficient Password Strength
  Validation}\label{vulnerability-11-insufficient-password-strength-validation}}
  
  File: NodeGoat/app/routes/session.js
  
  Issue: The \texttt{PASS\_RE} regex allows for weak passwords that do not
  require a mix of character types.
  
  Vulnerability Fix:
  
  \begin{framed}\scriptsize
    \begin{verbatim}
const validateSignup = (...) => {
+    const PASS_RE = /^.*(?=.*\d)(?=.*[a-z])(?=.*[A-Z]).{8,18}$/;
-    const PASS_RE = /^.{1,20}$/;

-    if (!PASS_RE.test(password)) {
+    if (!PASS_RE.test(password)) {
        errors.passwordError = "Password must be 8 to 18 characters" +
-            " including numbers, lowercase and uppercase letters.";
+            " including at least one number, one lowercase letter, and one uppercase letter.";
        return false;
    }
    // ...
};
     \end{verbatim}
     
  \end{framed}\normalsize
  
  \hypertarget{analyzing-vulnerabilities}{%
  \subsubsection{Analyzing
  Vulnerabilities}\label{analyzing-vulnerabilities}}
  
  \hypertarget{vulnerability-1-remote-code-execution-in-config-file}{%
  \paragraph{Vulnerability 1: Remote Code Execution in Config
  File}\label{vulnerability-1-remote-code-execution-in-config-file}}
  
  File: NodeGoat/config/config.js
  
  Issue: User input (\texttt{finalEnv}) is used to construct a file path
  without proper validation, which could allow an attacker to traverse the
  filesystem or execute arbitrary code.
  
  Vulnerability Fix:
  
  \begin{framed}\scriptsize
    \begin{verbatim}
const validateSignup = (...) => {
+    const PASS_RE = /^.*(?=.*\d)(?=.*[a-z])(?=.*[A-Z]).{8,18}$/;
-    const PASS_RE = /^.{1,20}$/;

-    if (!PASS_RE.test(password)) {
+    if (!PASS_RE.test(password)) {
        errors.passwordError = "Password must be 8 to 18 characters" +
-            " including numbers, lowercase and uppercase letters.";
+            " including at least one number, one lowercase letter, and one uppercase letter.";
        return false;
    }
    // ...
};
     \end{verbatim}
     
  \end{framed}\normalsize
  
  \hypertarget{vulnerability-2-insecure-direct-object-references-idor-in-allocations}{%
  \paragraph{Vulnerability 2: Insecure Direct Object References (IDOR) in
  Allocations}\label{vulnerability-2-insecure-direct-object-references-idor-in-allocations}}
  
  File: NodeGoat/app/routes/allocations.js
  
  Issue: User input from \texttt{req.params.userId} is used directly to
  query the database, which could allow an unauthorized user to access
  other users' data.
  
  Vulnerability Fix:
  
  \begin{framed}\scriptsize
    \begin{verbatim}
this.displayAllocations = (req, res, next) => {
    
    const {
-        userId
+        userId: rawUserId
    } = req.params;
+
+    // Verify the user ID from the session
+    const {
+        userId: sessionUserId
+    } = req.session;
+
+    const isAuthorized = rawUserId == sessionUserId; // Use proper authorization check
+    if (!isAuthorized) {
+        return res.status(403).json({ error: "Unauthorized access" });
+    }

    const {
        threshold
    } = req.query;
// ...
      \end{verbatim}
      
  \end{framed}\normalsize
  
  \hypertarget{vulnerability-3-server-side-request-forgery-ssrf-in-research}{%
  \paragraph{Vulnerability 3: Server-Side Request Forgery (SSRF) in
  Research}\label{vulnerability-3-server-side-request-forgery-ssrf-in-research}}
  
  File: NodeGoat/app/routes/research.js
  
  Issue: The \texttt{url} and \texttt{symbol} parameters from user input
  are concatenated and used in a GET request without validation, allowing
  SSRF attacks.
  
  Vulnerability Fix:
  
  \begin{framed}\scriptsize
    \begin{verbatim}
this.displayResearch = (req, res) => {

    if (req.query.symbol) {
-       const url = req.query.url + req.query.symbol;
+       const allowedDomains = ["https://api.example.com"]; // Replace with the 
                                                           // actual domain you want to allow
+       const defaultResearchUrl = allowedDomains[0] + "/stock_info"
+       const safeSymbol = encodeURIComponent(req.query.symbol); // URI encode the symbol to avoid 
                                                                // manipulation
+       const url = defaultResearchUrl + "?symbol=" + safeSymbol;

        return needle.get(url, ...);
    }
// ...
      \end{verbatim}
      
  \end{framed}\normalsize
  
  \hypertarget{vulnerability-4-cross-site-scripting-xss-in-memos}{%
  \paragraph{Vulnerability 4: Cross-Site Scripting (XSS) in
  Memos}\label{vulnerability-4-cross-site-scripting-xss-in-memos}}
  
  File: NodeGoat/app/routes/memos.js
  
  Issue: User input \texttt{req.body.memo} is directly inserted into the
  database and later rendered without proper encoding.
  
  Vulnerability Fix:
  
  \begin{framed}\scriptsize
\begin{verbatim}
const MemosDAO = require("../data/memos-dao").MemosDAO;
const {
    environmentalScripts
} = require("../../config/config");
+const ESAPI = require("node-esapi");
+
function MemosHandler(db) {
// ...
    this.addMemos = (req, res, next) => {
+
+        // Encode memo content to avoid XSS
+        const encodedMemo = ESAPI.encoder().encodeForHTML(req.body.memo);

-        memosDAO.insert(req.body.memo, (err, docs) => {
+        memosDAO.insert(encodedMemo, (err, docs) => {
// ...
\end{verbatim}
      
  \end{framed}\normalsize
  
  \hypertarget{vulnerability-5-command-injection-in-contributions}{%
  \paragraph{Vulnerability 5: Command Injection in
  Contributions}\label{vulnerability-5-command-injection-in-contributions}}
  
  File: NodeGoat/app/routes/contributions.js
  
  Issue: Use of \texttt{eval} with user input \texttt{req.body.preTax},
  \texttt{req.body.afterTax}, and \texttt{req.body.roth}, enabling command
  injection attacks.
  
  Vulnerability Fix:
  
  \begin{framed}\scriptsize
\begin{verbatim}
  this.handleContributionsUpdate = (req, res, next) => {
  
  -        const preTax = eval(req.body.preTax);
  -        const afterTax = eval(req.body.afterTax);
  -        const roth = eval(req.body.roth);
  +        const preTax = parseFloat(req.body.preTax);
  +        const afterTax = parseFloat(req.body.afterTax);
  +        const roth = parseFloat(req.body.roth);
  
          const {
              userId
          } = req.session;
  // ...
  \end{verbatim}
      
  \end{framed}\normalsize

\end{document}
