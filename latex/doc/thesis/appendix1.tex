\chapter{Surowe wyniki analizy kodu aplikacji NodeGoat bez podpowiedzi w kodzie w formacie Markdown przetłumaczonym na LaTex}\label{app1}

% {\large\textbf{}}
\normalsize
\begin{enumerate}
  \def\labelenumi{\arabic{enumi}.}
  \item
    \textbf{MongoClient Connection String Exposure}: The \texttt{db}
    variable likely contains the connection string to MongoDB, which
    includes credentials. This sensitive information can be exposed if the
    configuration file is not properly secured.
  \item
    \textbf{Insecure Express Session Configuration}: The session
    middleware in \texttt{server.js} uses a cookieSecret from a config
    file for signing cookies, but it does not have secure attributes such
    as \texttt{httpOnly}, \texttt{secure}, \texttt{sameSite}, as well as
    not using a store which can lead to security issues.
  \item
    \textbf{SWIG Template Auto-Escape Disabled}: In \texttt{server.js},
    the SWIG template engine is used with \texttt{autoescape:\ false},
    making it vulnerable to Cross-Site Scripting (XSS) attacks if user
    input is included in templates.
  \item
    \textbf{Use of Marked Library for Markdown Parsing}: The marked
    library is being used in \texttt{server.js} with the \texttt{sanitize}
    option set to \texttt{true}, which is good for preventing XSS.
    However, if library updates change the default behavior, it could lead
    to vulnerabilities.
  \item
    \textbf{Database Reset Script}: In \texttt{db-reset.js}, the process
    exits with a different status code in case of an error. This can lead
    to inconsistent behavior and potentially leak information about the
    database status or structure.
  \item
    \textbf{Lack of HTTPS}: The server in \texttt{server.js} is started
    with HTTP and not HTTPS, leading to the exposure of data in transit.
  \item
    \textbf{Open Docker Ports in Development}: The
    \texttt{docker-compose.yml} file shows ports open which should not be
    exposed in a production environment.
  \item
    \textbf{Hardcoded Credentials in \texttt{db-reset.js}}: Usernames and
    passwords are hardcoded in the \texttt{db-reset.js} script, which is a
    bad practice for managing credentials.
  \end{enumerate}
  
  Here are the corrected segments of code for each vulnerability found:
  
  \textbf{1. MongoClient Connection String Exposure}:
  
  \begin{framed}\scriptsize
  \begin{Highlighting}[]
  \NormalTok{// No diffs displayed as it requires infrastructural practices like environment variable management or secret management systems, which can\textquotesingle{}t be presented in source code directly.}
  \end{Highlighting}
  \end{framed}\normalsize
  
  \textbf{2. Insecure Express Session Configuration}:
  
  \begin{framed}\scriptsize 
  \begin{verbatim}
  -    app.use(session({
  -        secret: cookieSecret,
  -        saveUninitialized: true,
  -        resave: true
  -    }));
  +    app.use(session({
  +        secret: cookieSecret,
  +        saveUninitialized: true,
  +        resave: false,
  +        cookie: {
  +            httpOnly: true,
  +            secure: true,
  +            sameSite: "lax"
  +        },
  +        store: /* instance of secure session storage */
  +    }));
  \end{verbatim}
  \end{framed}
  
  \textbf{3. SWIG Template Auto-Escape Disabled}:

  \begin{framed}\scriptsize 
  \begin{verbatim}
  -   swig.setDefaults({
  -       // Autoescape disabled
  -       autoescape: false
  -   });
  +   swig.setDefaults({
  +       % Autoescape enabled
  +       autoescape: true
  +   });
  \end{verbatim}
  \end{framed}

  
  \textbf{4. Use of Marked Library for Markdown Parsing}: The current code
  has correctly set the \texttt{sanitize} option, but developers should
  remain vigilant for future library updates.
  
  \textbf{5. Database Reset Script (no explicit fix, informational only)}:
  Ensure consistent process exit codes and safeguard against information
  leakage.
  
  \textbf{6. Lack of HTTPS}:
  
  \begin{framed}\scriptsize
  \begin{verbatim}[]
-    http.createServer(app).listen(port, () => {
+   const https = require('https');
+   const fs = require('fs');
+
+   const options = {
+     key: fs.readFileSync('path/to/key.pem'),
+     cert: fs.readFileSync('path/to/cert.pem')
+   };
+
+   https.createServer(options, app).listen(port, () => {
        console.log(`Express http server listening on port ${port}`);
    });
  \end{verbatim}
  \end{framed}
  
  \textbf{7. Open Docker Ports in Development}: Adjust the
  \texttt{docker-compose.yml} configuration for production to not expose
  sensitive ports publicly.
  
  \textbf{8. Hardcoded Credentials in \texttt{db-reset.js}}:
  
  \begin{framed}\scriptsize
  \begin{verbatim}
-       "password": "Admin_123",
+       // Use environment variable or secure secrets management 
+       // to retrieve password
+       "password": process.env.ADMIN_PASSWORD,
  \end{verbatim}
  \end{framed}
  
  \begin{enumerate}
  \def\labelenumi{\arabic{enumi}.}
  \item
    \textbf{Insecure Password Storage (user-dao.js)}: Passwords should not
    be stored in plaintext. The \texttt{addUser} method currently stores
    the password received from the user directly into the database.
  


  \begin{framed}
    \scriptsize
  \begin{verbatim} % Use the 'lstlisting' environment for code highlighting
  - password %received from request param
  + password: bcrypt.hashSync(password, bcrypt.genSaltSync(8)) % Hash the password
  \end{verbatim}
  \end{framed}

  \item
    \textbf{Insecure Password Verification (user-dao.js)}: The
    \texttt{validateLogin} method uses a simple string comparison to
    validate passwords, which would only be secure if the passwords were
    properly hashed and salted before storage and then compared using a
    secure function.
  
  \begin{framed}
    \scriptsize
  \begin{verbatim}
- const comparePassword = (fromDB, fromUser) => { 
-  return fromDB === fromUser; 
-  }; 
+ const comparePassword = (fromDB, fromUser) => { 
+   return bcrypt.compareSync(fromUser, fromDB); 
+ };
  \end{verbatim}
  \end{framed}
  \item
    \textbf{Potential NoSQL Injection (research-dao.js, getBySymbol)}: The
    \texttt{getBySymbol} method creates a query without proper sanitation
    or parameterized queries, which may open up the application to NoSQL
    injection attacks.
  
  \begin{framed}
    \scriptsize
    \begin{verbatim}
-    symbol
+    symbol: symbol
+    // The searchCriteria function should return the properly constructed query object
+    db.collection('research').find(searchCriteria()).toArray(callback);
  \end{verbatim}
  \end{framed}
  \item
    \textbf{Potential NoSQL Injection (profile-dao.js, updateUser)}: The
    \texttt{updateUser} method directly uses the incoming parameter
    \texttt{userId} after parsing it as an integer. Although this reduces
    the risk, it's still a good habit to use a parameterized query.
  \begin{framed}
    \scriptsize
    \begin{verbatim}
-                    symbol
+                    symbol: symbol
+                // The searchCriteria function should return the properly constructed query object
+                db.collection('research').find(searchCriteria()).toArray(callback);
    \end{verbatim}
  \end{framed}
  \item
    \textbf{Lack of Input Validation}: Across various DAO functions, there
    is a lack of input validation to ensure that the values passed to the
    database operations do not contain malicious input.
  
    For this point, code modifications would be more extensive and not as
    straightforward to display in a patch format because proper input
    validation would need to be implemented throughout each function that
    takes external input.
  \end{enumerate}
  
  \hypertarget{vulnerability-6-nosql-injection-in-allocations}{%
  \paragraph{Vulnerability 6: NoSQL Injection in
  Allocations}\label{vulnerability-6-nosql-injection-in-allocations}}
  
  File: NodeGoat/app/data/allocations-dao.js
  
  Issue: The method \texttt{getByUserIdAndThreshold} is susceptible to
  NoSQL injection as it constructs a query using a \texttt{\$where}
  operator with user input, which can be manipulated.
  
  Vulnerability Fix:
  
  \begin{framed}
  \begin{verbatim}
this.getByUserIdAndThreshold = (userId, threshold, callback) => {
-    const parsedUserId = parseInt(userId);
-    const searchCriteria = () => {
-        if (threshold) {
-            return {
-                $where: `this.userId == ${parsedUserId} && \
                          this.stocks > '${threshold}'`
-            };
+    const parsedUserId = parseInt(userId)
+    let query = { userId: parsedUserId };
+
+    if (threshold) {
+        let numericThreshold = parseFloat(threshold);
+        if (!isNaN(numericThreshold)) {
+            query.stocks = { $gt: numericThreshold };
          }
-        return {
-            userId: parsedUserId
-        };
-    };
+    }
-    allocationsCol.find(searchCriteria()).toArray(...);
+    allocationsCol.find(query).toArray(...);
  };
  \end{verbatim}
  \end{framed}
  
  \hypertarget{vulnerability-7-insecure-configuration-in-github-workflow}{%
  \paragraph{Vulnerability 7: Insecure Configuration in Github
  Workflow}\label{vulnerability-7-insecure-configuration-in-github-workflow}}
  \\
  File: NodeGoat/.github/workflows/e2e-test.yml
  
  Issue: The configuration file uses hardcoded version \texttt{"4.0"} for
  the MongoDB Docker image, which might be outdated and contain known
  vulnerabilities.
  
  Vulnerability Fix:
  
  \begin{verbatim}
    - docker run -d -p 27017:27017 mongo:4.0
    + docker run -d -p 27017:27017 mongo:latest
    \end{verbatim}
    
  
  \hypertarget{vulnerability-8-arbitrary-redirect-in-index-route}{%
  \paragraph{Vulnerability 8: Arbitrary Redirect in Index
  Route}\label{vulnerability-8-arbitrary-redirect-in-index-route}}
  
  File: NodeGoat/app/routes/index.js
  
  Issue: The \texttt{/learn} route redirects to a user-specified URL
  without validation, which can be exploited for phishing attacks.
  
  Vulnerability Fix:
  
  \scriptsize
  \begin{verbatim}
app.get("/learn", isLoggedIn, (req, res) => {
-    return res.redirect(req.query.url);
+    const allowedUrls = ["https://trustedresource.com/learn", "https://anothertrustedsource.com/resources"];
+    const requestedUrl = req.query.url;
+    if (allowedUrls.includes(requestedUrl)) {
+        return res.redirect(requestedUrl);
+    } else {
+        return res.status(400).send("Invalid URL provided for redirection.");
+    }
    });
    \end{verbatim}
    \normalsize
  
  \hypertarget{vulnerability-9-insufficient-logging-and-monitoring-in-error-handler}{%
  \paragraph{Vulnerability 9: Insufficient Logging and Monitoring in Error
  Handler}\label{vulnerability-9-insufficient-logging-and-monitoring-in-error-handler}}
  
  File: NodeGoat/app/routes/error.js
  
  Issue: The error handling middleware logs the error message but doesn't
  notify the team or use a centralized logging system.
  
  Vulnerability Fix:
  
  \begin{framed}\scriptsize
    \begin{verbatim}
+ const {logger} = require("../log"); // Hypothetical logging module that should be created

const errorHandler = (err, req, res,next) => {
-    console.error(err.message);
-    console.error(err.stack);
+    logger.error(err.message, {stack: err.stack, req});
// ...
      };
      \end{verbatim}
  \end{framed}\normalsize
  
  \hypertarget{vulnerability-10-cross-site-scripting-xss-in-profile-data-rendering}{%
  \paragraph{Vulnerability 10: Cross-Site Scripting (XSS) in Profile Data
  Rendering}\label{vulnerability-10-cross-site-scripting-xss-in-profile-data-rendering}}
  
  File: NodeGoat/app/routes/profile.js
  
  Issue: Directly embedding user input from \texttt{doc.website} in the
  HTML without encoding it, which can lead to Cross-Site Scripting
  attacks.
  
  Vulnerability Fix:
  
  \begin{framed}\scriptsize
    \begin{verbatim}
this.displayProfile = (req, res, next) => {
    // ...
    doc.userId = userId;
-   doc.website = ESAPI.encoder().encodeForHTML(doc.website);
+   doc.website = ESAPI.encoder().encodeForURL(doc.website);
    // ...
      };
      \end{verbatim}
  \end{framed}\normalsize
  
  \hypertarget{vulnerability-11-insufficient-password-strength-validation}{%
  \paragraph{Vulnerability 11: Insufficient Password Strength
  Validation}\label{vulnerability-11-insufficient-password-strength-validation}}
  
  File: NodeGoat/app/routes/session.js
  
  Issue: The \texttt{PASS\_RE} regex allows for weak passwords that do not
  require a mix of character types.
  
  Vulnerability Fix:
  
  \begin{framed}\scriptsize
    \begin{verbatim}
const validateSignup = (...) => {
+    const PASS_RE = /^.*(?=.*\d)(?=.*[a-z])(?=.*[A-Z]).{8,18}$/;
-    const PASS_RE = /^.{1,20}$/;

-    if (!PASS_RE.test(password)) {
+    if (!PASS_RE.test(password)) {
        errors.passwordError = "Password must be 8 to 18 characters" +
-            " including numbers, lowercase and uppercase letters.";
+            " including at least one number, one lowercase letter, and one uppercase letter.";
        return false;
    }
    // ...
};
     \end{verbatim}
     
  \end{framed}\normalsize
  
  \hypertarget{analyzing-vulnerabilities}{%
  \subsubsection{Analyzing
  Vulnerabilities}\label{analyzing-vulnerabilities}}
  
  \hypertarget{vulnerability-1-remote-code-execution-in-config-file}{%
  \paragraph{Vulnerability 1: Remote Code Execution in Config
  File}\label{vulnerability-1-remote-code-execution-in-config-file}}
  
  File: NodeGoat/config/config.js
  
  Issue: User input (\texttt{finalEnv}) is used to construct a file path
  without proper validation, which could allow an attacker to traverse the
  filesystem or execute arbitrary code.
  
  Vulnerability Fix:
  
  \begin{framed}\scriptsize
    \begin{verbatim}
const validateSignup = (...) => {
+    const PASS_RE = /^.*(?=.*\d)(?=.*[a-z])(?=.*[A-Z]).{8,18}$/;
-    const PASS_RE = /^.{1,20}$/;

-    if (!PASS_RE.test(password)) {
+    if (!PASS_RE.test(password)) {
        errors.passwordError = "Password must be 8 to 18 characters" +
-            " including numbers, lowercase and uppercase letters.";
+            " including at least one number, one lowercase letter, and one uppercase letter.";
        return false;
    }
    // ...
};
     \end{verbatim}
     
  \end{framed}\normalsize
  
  \hypertarget{vulnerability-2-insecure-direct-object-references-idor-in-allocations}{%
  \paragraph{Vulnerability 2: Insecure Direct Object References (IDOR) in
  Allocations}\label{vulnerability-2-insecure-direct-object-references-idor-in-allocations}}
  
  File: NodeGoat/app/routes/allocations.js
  
  Issue: User input from \texttt{req.params.userId} is used directly to
  query the database, which could allow an unauthorized user to access
  other users' data.
  
  Vulnerability Fix:
  
  \begin{framed}\scriptsize
    \begin{verbatim}
this.displayAllocations = (req, res, next) => {
    
    const {
-        userId
+        userId: rawUserId
    } = req.params;
+
+    // Verify the user ID from the session
+    const {
+        userId: sessionUserId
+    } = req.session;
+
+    const isAuthorized = rawUserId == sessionUserId; // Use proper authorization check
+    if (!isAuthorized) {
+        return res.status(403).json({ error: "Unauthorized access" });
+    }

    const {
        threshold
    } = req.query;
// ...
      \end{verbatim}
      
  \end{framed}\normalsize
  
  \hypertarget{vulnerability-3-server-side-request-forgery-ssrf-in-research}{%
  \paragraph{Vulnerability 3: Server-Side Request Forgery (SSRF) in
  Research}\label{vulnerability-3-server-side-request-forgery-ssrf-in-research}}
  
  File: NodeGoat/app/routes/research.js
  
  Issue: The \texttt{url} and \texttt{symbol} parameters from user input
  are concatenated and used in a GET request without validation, allowing
  SSRF attacks.
  
  Vulnerability Fix:
  
  \begin{framed}\scriptsize
    \begin{verbatim}
this.displayResearch = (req, res) => {

    if (req.query.symbol) {
-       const url = req.query.url + req.query.symbol;
+       const allowedDomains = ["https://api.example.com"]; // Replace with the 
                                                           // actual domain you want to allow
+       const defaultResearchUrl = allowedDomains[0] + "/stock_info"
+       const safeSymbol = encodeURIComponent(req.query.symbol); // URI encode the symbol to avoid 
                                                                // manipulation
+       const url = defaultResearchUrl + "?symbol=" + safeSymbol;

        return needle.get(url, ...);
    }
// ...
      \end{verbatim}
      
  \end{framed}\normalsize
  
  \hypertarget{vulnerability-4-cross-site-scripting-xss-in-memos}{%
  \paragraph{Vulnerability 4: Cross-Site Scripting (XSS) in
  Memos}\label{vulnerability-4-cross-site-scripting-xss-in-memos}}
  
  File: NodeGoat/app/routes/memos.js
  
  Issue: User input \texttt{req.body.memo} is directly inserted into the
  database and later rendered without proper encoding.
  
  Vulnerability Fix:
  
  \begin{framed}\scriptsize
\begin{verbatim}
const MemosDAO = require("../data/memos-dao").MemosDAO;
const {
    environmentalScripts
} = require("../../config/config");
+const ESAPI = require("node-esapi");
+
function MemosHandler(db) {
// ...
    this.addMemos = (req, res, next) => {
+
+        // Encode memo content to avoid XSS
+        const encodedMemo = ESAPI.encoder().encodeForHTML(req.body.memo);

-        memosDAO.insert(req.body.memo, (err, docs) => {
+        memosDAO.insert(encodedMemo, (err, docs) => {
// ...
\end{verbatim}
      
  \end{framed}\normalsize
  
  \hypertarget{vulnerability-5-command-injection-in-contributions}{%
  \paragraph{Vulnerability 5: Command Injection in
  Contributions}\label{vulnerability-5-command-injection-in-contributions}}
  
  File: NodeGoat/app/routes/contributions.js
  
  Issue: Use of \texttt{eval} with user input \texttt{req.body.preTax},
  \texttt{req.body.afterTax}, and \texttt{req.body.roth}, enabling command
  injection attacks.
  
  Vulnerability Fix:
  
  \begin{framed}\scriptsize
\begin{verbatim}
  this.handleContributionsUpdate = (req, res, next) => {
  
  -        const preTax = eval(req.body.preTax);
  -        const afterTax = eval(req.body.afterTax);
  -        const roth = eval(req.body.roth);
  +        const preTax = parseFloat(req.body.preTax);
  +        const afterTax = parseFloat(req.body.afterTax);
  +        const roth = parseFloat(req.body.roth);
  
          const {
              userId
          } = req.session;
  // ...
  \end{verbatim}
      
  \end{framed}\normalsize