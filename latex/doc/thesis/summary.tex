% !TEX encoding = UTF-8 Unicode 
% !TEX root = praca.tex


\section*{Podsumowanie}
W pracy zbadano wykorzystanie dużych modeli językowych (LLM) w kontekście statycznej analizy kodu, skupiając się na ich zdolnościach do identyfikacji i naprawy błędów programistycznych. Rozpatrzono różne aspekty stosowania LLM, w tym ich integrację z istniejącymi narzędziami do analizy kodu, potencjał w automatyzacji procesów weryfikacji kodu oraz wyzwania związane z ich praktycznym zastosowaniem. Praca porusza również kwestie związane z ograniczeniami modeli językowych, takie jak ich zależność od jakości i zakresu danych treningowych oraz konieczność humanitarnej nadzoru i weryfikacji wyników generowanych przez te systemy.

\section*{Wnioski}
Duże modele językowe oferują obiecujące możliwości w analizie i naprawie kodu, jednak ich skuteczność jest ograniczona w złożonych scenariuszach cyberbezpieczeństwa. Wyniki badań podkreślają konieczność ludzkiej ekspertyzy w procesie weryfikacji i poprawy kodu. Dalsze badania są potrzebne do rozwoju metodologii i narzędzi, które pozwolą na pełniejsze wykorzystanie potencjału LLM w poprawie bezpieczeństwa aplikacji. 
\\
\textbf{Na podstawie przeprowadzonych badań, można wyciągnąć następujące wnioski:}

\begin{enumerate}
    \item LLM mogą efektywnie identyfikować i proponować naprawy dla standardowych błędów w kodzie, ale ich skuteczność maleje wraz ze wzrostem złożoności zadania.
    \item Modele te wymagają precyzyjnie sformułowanych zapytań i dobrze zdefiniowanych kontekstów, aby generować użyteczne wyniki. Konieczne są dodatkowe badania nad metodami wyboru i przygotowania danych wejściowych.
    \item Ograniczenia LLM, takie jak brak głębokiego zrozumienia logiki programistycznej i kontekstu biznesowego, mogą prowadzić do nieoptymalnych lub niebezpiecznych sugestii.
    \item Istotna jest ciągła interakcja i weryfikacja przez doświadczonych programistów, aby zapewnić bezpieczeństwo i poprawność proponowanych rozwiązań.
    \item Rozwój narzędzi wspomagających, które integrują LLM z tradycyjnymi metodami statycznej analizy kodu, może zwiększyć skuteczność wykrywania i naprawy błędów.
    \item Należy zachować ostrożność w kwestii etycznej i prawnej odpowiedzialności za błędy wprowadzone lub niezauważone przez modele językowe.
    \item Dalsze badania powinny skupić się na usprawnieniu interakcji między LLM a programistami, zwłaszcza w kontekście uczenia się z interakcji i feedbacku.
    \item Konieczne jest badanie wpływu na wydajność i jakość pracy programistów, w tym potencjalnych ryzyk związanych z nadmiernym poleganiem na automatycznych sugestiach.
\end{enumerate}
